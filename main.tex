\documentclass[12pt, a4paper,titlepage]{article}

% Packages einbinden
\usepackage[latin1]{inputenc}
\usepackage{ngerman}
\usepackage{amsmath, amssymb, amsfonts, amsthm}
\usepackage{makeidx}
\usepackage{hyperref}

% Autor, Titel und Datum
\date{\today}
\author{michael.gerhaeuser@gmx.de \and benedikt.ahrens@web.de}
\title{Gruppentheorie}

%%%%%%%%%%%%%%%%%%%%%%%%%%%%%%%%%%%%%%%%%%%%%

\makeindex

% Der erste Abschnitt soll Nummer 0 haben
\setcounter{section}{-1}

% Neue Theoremstile
\newtheoremstyle{mydefinition}
 {\topsep} % Platz vor dem Theorem
 {\topsep} % Platz nach dem Theorem
 {} % Schrift im Theorem
 {} % Einrueckung (empty = keine Einrueckung, \parindent = Normale Paragrapheneinrueckung)
 {} % Schrift in der Kopfzeile des Theorems
 {} % Punktation nach dem Theoremkopf
 {\newline} %     Space after thm head (\newline = linebreak)
 {{\bfseries\thmnumber{#2 }\thmname{#1}} \emph{\thmnote{ #3}}} % Kopfformatierung

\newtheoremstyle{myplain}
 {\topsep} % Platz vor dem Theorem
 {\topsep} % Platz nach dem Theorem
 {\itshape} % Schrift im Theorem
 {} % Einrueckung (empty = keine Einrueckung, \parindent = Normale Paragrapheneinrueckung)
 {} % Schrift in der Kopfzeile des Theorems
 {} % Punktation nach dem Theoremkopf
 {\newline} %     Space after thm head (\newline = linebreak)
 {{\bfseries\thmnumber{#2 }\thmname{#1}} \thmnote{ #3}} % Kopfformatierung

\newtheoremstyle{myremark}
 {\topsep} % Platz vor dem Theorem
 {\topsep} % Platz nach dem Theorem
 {} % Schrift im Theorem
 {} % Einrueckung (empty = keine Einrueckung, \parindent = Normale Paragrapheneinrueckung)
 {} % Schrift in der Kopfzeile des Theorems
 {} % Punktation nach dem Theoremkopf
 {\newline} %     Space after thm head (\newline = linebreak)
 {{\bfseries\thmnumber{#2 }\thmname{#1}} \emph{\thmnote{#3}}} % Kopfformatierung

\newtheoremstyle{mybeweis}
 {\topsep} % Platz vor dem Theorem
 {\topsep} % Platz nach dem Theorem
 {} % Schrift im Theorem
 {} % Einrueckung (empty = keine Einrueckung, \parindent = Normale Paragrapheneinrueckung)
 {\itshape} % Schrift in der Kopfzeile des Theorems
 {} % Punktation nach dem Theoremkopf
 {\newline} %     Space after thm head (\newline = linebreak)
 {\thmnumber{#2}\thmname{#1}\thmnote{ #3}:} % Kopfformatierung
%{\thmnumber{ #2 }\thmname{#1}\thmnote{#3}:} % Kopfformatierung

% Theoremumgebungen fuer das Skript definieren
\theoremstyle{mydefinition}
\newtheorem{definition}{Definition}[section]
\newtheorem{defundsatz}[definition]{Definition und Satz}

\theoremstyle{myplain}
\newtheorem{satz}[definition]{Satz}
\newtheorem*{satz*}{Satz}
\newtheorem{lemma}[definition]{Lemma}
\newtheorem{folgerung}[definition]{Folgerung}

\theoremstyle{myremark}
\newtheorem{beispiel}[definition]{Beispiel}
\newtheorem*{beispiel*}{Beispiel}
\newtheorem{bemerkung}[definition]{Bemerkung}
\newtheorem*{bemerkung*}{Bemerkung}

\newtheorem{aufgabe}{Aufgabe}[section]

\theoremstyle{mybeweis}
\newtheorem*{beweis}{Beweis}

% Mathematische Operatoren definieren
% Automorphismengruppen, Spezielle Gruppen
\DeclareMathOperator{\Aut}{Aut}
\DeclareMathOperator{\Inn}{Inn}
\DeclareMathOperator{\Symm}{S}
\DeclareMathOperator{\Alt}{A}
\DeclareMathOperator{\Syl}{Syl}
\DeclareMathOperator{\Di}{D}
\DeclareMathOperator{\Gl}{Gl}
\DeclareMathOperator{\PSL}{PSL}
\DeclareMathOperator{\SL}{SL}
\DeclareMathOperator{\PSU}{PSU}
\DeclareMathOperator{\Gal}{Gal}
\DeclareMathOperator{\Quat}{Q}
\DeclareMathOperator{\Mod}{M}
\DeclareMathOperator{\SD}{SD}
% Abbildungen
\DeclareMathOperator{\sign}{sign}
\DeclareMathOperator{\id}{id}
%\DeclareMathOperator{\deg}{deg}
% Mengen bezueglich Abbildungen
\DeclareMathOperator{\Img}{Im} % Bild
\DeclareMathOperator{\Ker}{Ker} % Kern
% Zentralisator, Zentrum, Normalisator
\DeclareMathOperator{\Cen}{C} % Zentralisator
\DeclareMathOperator{\Zen}{Z} % Zentrum
\DeclareMathOperator{\Norm}{N} % Normalisator
% Sonstiges
\DeclareMathOperator{\Char}{ char }
\DeclareMathOperator{\ggT}{ggT}
\DeclareMathOperator{\kgV}{kgV}
\DeclareMathOperator{\ord}{ord}



% Neue Kommandos definieren
% Klammern
\renewcommand{\l}{\left}
\renewcommand{\r}{\right}

% fuer den Index
\renewcommand{\seename}{s.}

% Neue Punkte im Inhalt einfuegen
\newcommand{\addtotoc}[1]{\addcontentsline{toc}{subsection}{#1}}

% Mathematische Operatoren
\newcommand{\nt}{\lhd} % Normalteiler (links)
\newcommand{\ntr}{\vartriangleright} % Normalteiler (rechts)
%\newcommand{\bigcupdot}{\bigcup \hspace{-0.35cm} \cdot} % Disjunkte Vereinigung
%\newcommand{\bigcupdot}{\mathop{\scriptstyle\amalg}\nolimits}
\newcommand{\bigcupdot}{\bigcup}
\newcommand{\cupdot}{\cup}
% Abkuerzungen fuer die mathbb & mathcal
\newcommand{\ZZ}{\mathbb{Z}}
\newcommand{\NN}{\mathbb{N}}
\newcommand{\PP}{\mathbb{P}}
\newcommand{\RR}{\mathbb{R}}
\newcommand{\FF}{\mathbb{F}}
\newcommand{\OOC}{\mathcal{O}}
\newcommand{\GGC}{\mathcal{G}}
\newcommand{\NNC}{\mathcal{N}}
\newcommand{\Agemo}{\text{Agemo}}

\newcommand{\spspace}{~\vspace{-6mm}}
%\newcommand{\spspace}{}

% Das eigentliche Dokument
\begin{document}
\maketitle

\begin{abstract}
\begin{center}
 Gruppentheorie nach Dr. Harald Meyer, Sommersemester 2007
\end{center}


\end{abstract}

\tableofcontents
\newpage
\section{Einf\"uhrung}
Folgendes wird als bekannt vorausgesetzt:
\begin{itemize}
 \item Definitionen: Gruppe, Untergruppe, Normalteiler, Faktorgruppe, Ordnung
 \index{Gruppe}
 \index{Untergruppe}
 \index{Normalteiler}
 \index{Faktorgruppe}
 \index{Ordnung}

 \item Untergruppenkriterien
 \item Homomorphiesatz und Isomorphies\"atze:
 \index{Homorphiesatz}
  \begin{satz}[Homomorphiesatz]
   \label{homomorphiesatz}
   $G, H$ Gruppen, $\varphi: G \to H$ Homomorphismus. Dann gilt: 
   \begin{equation*} 
    G/ H \cong \Img \varphi
   \end{equation*}
  \end{satz}
  \begin{satz}[1. Isomorphiesatz]
  \index{Isomorphies\"atze!1. Isomorphiesatz}
  \label{isomorphiesatz1}
   Sei $H\leq G, N \nt G$. Dann gilt:
   \begin{enumerate}
    \item $HN \leq G$
    \item $N \nt HN$
    \item $H \cap N \nt H$
    \item Der kanonische Homomorphismus
     \begin{equation*}
      H/H\cap N \to HN/N, \quad h\l(H\cap N\r) \mapsto hN
     \end{equation*}
     ist ein Isomorphismus.
   \end{enumerate}
  \end{satz}
  \begin{satz}[2. Isomorphiesatz]
  \index{Isomorphies\"atze!2. Isomorphiesatz}
  \label{isomorphiesatz2}
   Seien $M, N \nt G, M\leq N$. Dann gilt:
   \begin{enumerate}
    \item $N/M \nt G/M$
    \item Der kanonische Homomorphismus
     \begin{equation*}
      \l(G/M\r)/\l(N/M\r) \to G/N, \quad \l(gM\r)\l(N/M\r) \mapsto gN
     \end{equation*}
     ist ein Isomorphismus.
   \end{enumerate}
  \end{satz}
 \item Der Satz von Lagrange:
 \index{Lagrange!Satz von}
  \begin{satz}[Satz von Lagrange]
   Die Ordnung der Untergruppe teilt stets die Ordnung der Gruppe.
  \end{satz}
 \item Definition von zyklischen Gruppen, die Gruppen $\ZZ_n$.
 \index{Gruppe!zyklische}
 \index{Gruppe!$\ZZ_n$}
 \item Hauptsatz \"uber endliche abelsche Gruppen:
 %\index{Hauptsatz \"uber endliche abelsche Gruppen}
  \begin{satz}[Hauptsatz \"uber endliche abelsche Gruppen]
  \label{hauptsatz_ueber_endliche_abelsche_gruppen}
   Jede endliche abelsche Gruppe ist direktes Produkt von Gruppen von Primzahlpotenzordnung, das hei\ss{}t es gibt (nicht notwendig verschiedene) Primzahlen $p_1, \ldots, p_k$, so dass
   \begin{equation*}
    G \cong \ZZ_{p_1^{a_1}} \times \ldots \ZZ_{p_k^{a_k}}
   \end{equation*}
  \end{satz}
 \item Eigenschaften zyklischer Gruppen, zum Beispiel:
  \begin{equation*}
   \ZZ_{nm} \cong \ZZ_n \times \ZZ_m \Leftrightarrow \ggT\l(n, m\r)=1
  \end{equation*}
 \item Definition des direkten Produkts von Gruppen
 \item Definition der symmetrischen Gruppe $\Symm_n$, alternierende Gruppe $\Alt_n$
  \begin{equation*}
   \sign: \Symm_n \to \l\{1, -1 \r\}
  \end{equation*}
 \item Rechnen mit Zykeln in $\Symm_n$, Zerlegung in elementfremde Zykeln.
\end{itemize}

\section{Grundlagen}
Im ersten Kapitel werden wir verschiedene S\"atze besprechen, die evtl. auch schon bekannt sind.

Generalvoraussetzung: $G$ sei eine endliche Gruppe.

\begin{lemma}\label{1.1}\label{untergruppenkriterium}
 Seien $A, B \subset G$ (nicht notwendig Untergruppen) mit $A^{-1} = A,\quad B^{-1}=B$. Dann gilt:
 \begin{enumerate}
 \item $$|AB| = \frac{|A|\cdot |B|}{|A\cap B|},$$
  wobei $AB$ die Menge aller Produkte der Form $a\cdot b$ mit $a \in A$ und $b \in B$ ist.
 \item Seien jetzt $A, B \leq G$ Untergruppen von $G$. Dann gilt:
  $AB$ ist Untergruppe von $G$ genau dann, wenn $AB=BA$.
\end{enumerate}

\end{lemma}
\begin{beweis} \spspace
 \begin{enumerate}
 \item Seien $a_1,a_2$ aus $A$, $b_1,b_2$ aus $B$. Dann gilt:
  \begin{eqnarray*}a_{1}b_{1}=a_{2}b_{2} \Leftrightarrow a_{2}^{-1} a_1=b_2b_{1}^{-1} =:d \in A\cap B \\\Leftrightarrow \exists d \textrm{ in } A\cap B \textrm{ mit } a_1=a_2d, b=d^{-1}b_2.
  \end{eqnarray*}
  Zu einem festen Produkt $a_2b_2$ gibt es genau $|A\cap B|$ M\"oglichkeiten, Elemente $a_1 \textrm{ in } A, b_1 \textrm{ in } B$ zu w\"ahlen derart, dass $a_2b_2 = a_1b_1$. Damit ist $|AB|=\frac{|A|\cdot |B|}{|A\cap B|}$.
 \item Sei zun\"achst $AB$  Untergruppe von $G$. Dann ist $AB=(AB)^{-1}=B^{-1}A^{-1}\stackrel{A,B \leq G}{=}BA$.\\
Ist umgekehrt $AB=BA$, so folgt $(AB)(AB)=A(BA)B\stackrel{Vor.}{=}A(AB)B\\=(AA)(BB)=AB$. Nach dem Untergruppenkriterium f\"ur endliche Gruppen (\ref{1.1}) ist damit $AB$ Untergruppe von $G$.
\end{enumerate}

\end{beweis}


\begin{defundsatz}\label{1.2}\label{schabrackentapir}\spspace
\index{Automorphismengruppe}
\index{Automorphismus!innerer}
\index{Automorphismus!\"au\ss{}erer}
\index{Konjugation}
\index{Aut(G)}
\index{Inn(G)}
\index{Untergruppe!charakteristische}
 \begin{enumerate}
 \item Die Menge \emph{Aut(G)} aller Automorphismen $\alpha : G \to G$ bildet mit der Verkn\"ufpung \"uber Komposition eine Gruppe, die sog. \emph{Automorphismengruppe von $G$}.
 \item F\"ur jedes $g$ in $G$ ist die Abbildung $\varphi_g : G \to G, x \mapsto g^{-1}xg$, die sog. \emph{Konjugation mit $g$}, ein Automorphismus von $G$. Automorphismen von diesem Typ hei\ss{}en \emph{innere Automorphismen}, andere hei\ss{}en \emph{\"au\ss{}ere Automorphismen}.
\item Die Menge \emph{Inn(G)} der inneren Automorphismen bildet eine Untergruppe von $\Aut(G)$. Es gilt: $\Inn(G) \nt \Aut(G)$.
\item Eine Untergruppe $U$ von $G$ hei\ss{}t \emph{charakteristisch} (schreibe $U \Char G$), wenn $U$ invariant ist unter Automorphismen von $G$, d.h. wenn $\alpha(U)=U$ f\"ur alle $\alpha$ aus $\Aut(G)$.
\end{enumerate}

\end{defundsatz}

\begin{bemerkung*}
 Es gilt $U\nt G \Longleftrightarrow g^{-1}Ug = U$ f\"ur alle $g\in G \Longleftrightarrow \alpha(U)=U$ f\"ur alle $\alpha \in \Inn(G)$. Daher gilt $U\Char G \Longrightarrow U \nt G$.
\end{bemerkung*}





\begin{beweis}[von \ref{1.2}]  \spspace%\ref{gnk}
 \begin{enumerate}
  \item trivial
  \item $g^{-1}xg = 1 \Leftrightarrow x=gg^{-1}$. Also ist $\varphi_x$ injektiv und, da $|G|<\infty$, auch surjektiv.
    Weite ist $\varphi_g(xy)=g^{-1}xyg=g^{-1}xgg^{-1}yg=\varphi_g(x)\varphi_g(y)$. Also ist $\varphi_g$ ein Homomorphismus.
  \item Seien $h_1,h_2 \in G$. Wir zeigen\\ $\varphi_{h_{2}}\circ\varphi_{h_{1}}=\\\varphi_{h_1h_2}\in \Inn(G)$:\\
  $\varphi_{h_2}\circ\varphi_{h_1}(x)=\varphi_{h_2}(h_1^{-1}xh_1)=h_2^{-1}h_1^{-1}xh_1h_2=(h_1h_2)^{-1}x(h_1h_2)=\varphi_{h_1h_2}(x)$ Also ist $\varphi_{h_2}\circ\varphi_{h_1}=\varphi_{h_1h_2}\in \Inn(G)$ und damit $\Inn(G)\leq\Aut(G)$.\\
  Sei nun $\psi\in\Aut(G), g \in G$. Dann gilt: $(\psi^{-1}\varphi_g\psi)(x)=\psi^{-1}(g^{-1}\psi(x)g)=\psi^{-1}(g)^{-1}x\psi^{-1}(g)=\varphi_{\psi^{-1}(g)}(x)$. Also ist $\psi^{-1}\varphi_g\psi=\varphi_{\psi^{-1}(g)}\in\Inn(G)$ und damit $\Inn(G)\nt\Aut(G)$.

 \end{enumerate}

\end{beweis}




\begin{bemerkung}
 Die Formel $\varphi_{h_2}\circ\varphi_{h_1}=\varphi_{h_1h_2}$ im letzten Beweis ist nicht nur unsch\"on, sondern verursacht h\"aufig technische Schwierigkeiten: Die Komposition von Abbildungen ist f\"ur die Gruppentheorie ``verkehrt herum''. Z.B. sollte bei einem Homomorphismus $\varphi:G\to\Aut(H)$ mit $\varphi(g)=\alpha, \varphi(h)=\beta$ auch $\varphi(gh)=\alpha\beta$ gelten.\\
 Als Abhilfe vereinbart man folgende Konvention: Abbildungen, Homomorphismen werden rechts von Elementen als Exponent geschrieben, also $g^\alpha$ statt $\alpha(g)$ und entsprechend $g^{\alpha\beta}$ statt $\beta(\alpha(g))$.\\
 In der symmetrischen Gruppe lesen wir Zykel von links nach rechts und multiplizieren sie auch von links, also $(1\quad 2)(1\quad 3)=(1\quad 2\quad 3)$.
\end{bemerkung}

\begin{satz}[Korrespondenzsatz]\label{korres}\label{1.4}
 Sei $N$ Normalteiler in $G$. Der nat\"urliche Epimorphismus $v: G\to G/N$ induziert eine Bijektion $$\tilde v : \lbrace U|N \leq U \leq G\rbrace \to\lbrace V|V\leq G/N\rbrace, \quad U \longmapsto v(U)$$ zwischen der Menge der Zwischengruppen von $N$ und $G$ und der Menge der Untergruppen von $G/N$. Dabei entsprechen Normalteiler einander, es gilt also $N\leq U \nt G$ genau dann, wenn $v(U) \nt G/N$.
\end{satz}
\begin{beweis}
 Ist $U\nt G$, so ist $v(U) \leq G/N$ wieder Untergruppe, denn Homomorphismen bilden Untergruppen wieder auf Untergruppen ab. Also ist $\tilde v$ wohldefiniert.\\
 Die Umkehrabbildung ist $\tilde v^{-1}:V\to v^{-1}(V)$. Weil $v$ surjektiv ist, gilt $v(v^{-1}(V))=V$. Ist $N\leq U$, so gilt $v^{-1}(\underbrace{v(U)}_{UN/N})=UN=U$.
 Damit ist $\tilde v$ bijektiv.\\
 Sei jetzt $N\leq U \nt G$. Dann ist $x^{-1}Ux = U$ f\"ur alle $x \in G$. Hierauf $v$ angewandt liefert $v(U) = v(x^{-1}Ux)=v(x)^{-1}v(U)v(x)$. Durchl\"auft $x$ ganz $G$, so durchl\"auft $v(x)$ ganz $G/N$, also folgt $v(U) \nt G/N$. Sei $V \nt G/N$ und $U\leq G$ mit $N\leq U$ eine Untergruppe mit $v(U)=V$. Wir zeigen, dass $U$ und $x^{-1}Ux$ f\"ur jedes $x\in G$ durch $\tilde v$ auf $V$ abgebildet werden: es gilt $\tilde v(x^{-1}Ux)=v(x^{-1}Ux)=v(x)^{-1}v(U)v(x)=v(x)^{-1}Vv(x)\stackrel{V\nt G/N}{=}V=v(U)=\tilde v(U)$. Nach dem ersten Teil folgt $x^{-1}Ux=U`$. Da dies f\"ur alle $x\in G$ gilt, ist $U \nt G$.\qed
\end{beweis}

\begin{definition}\spspace
\index{Gruppe!einfache}
\index{Normalteiler!maximaler}
\index{Normalteiler!minimaler}
 \begin{enumerate}
  \item $G$ hei\ss{}t \emph{einfach}, wenn $G$ au\ss{}er $\l<1\r>$ und $G$ keine Normalteiler besitzt.
  \item Ein Normalteiler $N\nt G$ hei\ss{}t \emph{maximal}, wenn es keinen Normalteiler $M\nt G$ gibt mit $N < M < G$. Nach dem Korrespondenzsatz \ref{1.4} ist dies genau dann der Fall, wenn $G/N$ einfach ist.
  \item Ein Normalteiler $N\nt G$ hei\ss{}t \emph{minimal}, wenn es keinen Normalteiler $M\nt G$ gibt mit $\l<1\r> < M < N$.\\Beachte: $N$ muss hier nicht einfach sein, da $M\nt N\nt G \stackrel{\text{i.A.}}{\nRightarrow} M \nt G$.
 \end{enumerate}

\end{definition}

\begin{satz}[Cayley]\label{cayley}\label{1.6}
\index{Cayley!Satz von}
 Jede Gruppe der Ordnung $n$ ist isomorph zu einer Untergruppe der $\Symm_n$.
\end{satz}
\begin{beweis}
 Sei $|G| = n$ und f\"ur jedes $g\in G$ sei $r_g:G\to G, x\mapsto xg$ die \emph{Rechtstranslation um $g$}.\\
 Die Abbildung $r:G\to \Symm_G, g\mapsto r_g$ ist ein injektiver Gruppenhomomorphismus: $$x^{r(gh)}=x\cdot (gh) = (xg)h = xg^{r(h)}=(x^{r_g})^{r_h}=x^{r(g)r(h)}$$ liefert $r(gh)=r(g)r(h)$. Weil $$\Ker(r) = \lbrace g|x\cdot g = x \forall x\in G\rbrace = \l<1\r>$$ ist $r$ injektiv.\\
Mit dem Homomorphiesatz folgt $$G\cong G/\Ker(r) \cong \Inn(r) \leq \Symm_G \cong \Symm_n.$$
\end{beweis}

\begin{satz}
 Sei $G$ eine Gruppe mit $|G| = 2n$, $n$ ungerade. Dann besitzt $G$ einen Normalteiler der Ordnung $n$.
\end{satz}
\begin{beweis}
 Wir zeigen zun\"achst, dass $G$ ein Element $g$ der Ordnung 2 enth\"alt. Sei $A:=\lbrace g\in G| g=g^{-1}\rbrace$ und $B:=\lbrace g\in G| g\neq g^{-1}\rbrace$. Dann ist $G=A\amalg B$.$|B|$ ist gerade, da mit $g\in B$ auch $g^{-1}\in B, g^{-1}\neq g$. Also ist auch $|A| = 2n - |B|$ gerade. Da $1\in A$ folgt $|A|\geq 2$, d.h. es gibt $g\neq 1$ mit $g = g^{-1}$, also $g^2 = 1$. 
 Sei $r:G\to \Symm_G$ die Einbettung von $G$ wie im Beweis von \ref{1.6}. Dann ist $r(G)=:h$ ein Element der Ordnung 2. Wegen $\ord(h)=2$ kommen in der Zerlegung von $h$ in disjunkte Zykel nur Zykel der L\"ange 2 vor. Nach \ref{1.6} gilt $h(x) = x\cdot g$ f\"ur alle $x$ in $G$. 
 Da $g\neq 1$, ist $x\cdot g \neq x$ f\"ur alle $x$ in $G$, d.h. die Permutation $h$ hat keine Fixpunkte. Also ist $h$ das Produkt von n Zykeln der L\"ange 2. Damit ist $\sign(h) = (-1)^n=-1$. Also ist $\sign\circ r: G\to ({-1, 1})$ ein surjektiver Gruppenhomomorphismus und $\Ker(\sign\circ r)\nt G$. Nach dem Homomorphiesatz gilt $|\Ker(\sign\circ r)|=\frac{|G|}{2}= n$.\qed
\end{beweis}
\begin{bemerkung*}
 Ist $G$ eine einfache Gruppe mit $2\mid |G|$, so gilt $4\mid |G|$ (f\"ur $|G|>2$).
\end{bemerkung*}

\begin{satz}[Konstruktion] \label{konstruktion}\label{1.8}
 \index{Konstruktion}
 Sei $U\leq G$ eine Untergruppe von Index $\l[G:U\r]=k$ und $\lbrace U_{g_1},\ldots, U_{g_k}\rbrace=\lbrace N_1,\ldots,N_k\rbrace$ die Menge der rechten Nebenklassen von $U$. Dann wird durch 
  $$\varphi: G\to \Symm_n, g\longmapsto \l(\begin{array}{cccc} 1&2&\cdots&n\\i_1&i_2&\cdots&i_n\end{array}\r)$$ mit $N_1g=N_{i_1},\ldots, N_kg=N_{i_k}$ ein Gruppenhomomorphismus definiert.\\
  Es gilt $\Ker(\varphi)=\bigcap_{g\in G}g^{-1}Ug \leq U$. F\"ur $k\geq 2$ ist $\varphi$ nicht trivial.

\end{satz}

\begin{beweis}
 $\varphi$ ist wohldefiniert: 
\begin{eqnarray*}
 N_ig=N_jg &\Longleftrightarrow& Ug_ig=Ug_jg \Longleftrightarrow g_ig\in Ug_jg \\&\Longleftrightarrow& \exists u\in U:g_ig=ug_jg \Longleftrightarrow \exists u\in U:g_i=ug_j \\&\Longleftrightarrow& g_i \in Ug_j \Longleftrightarrow Ug_i=Ug_j \\&\Longleftrightarrow& N_i=N_j.
\end{eqnarray*}
Das bedeutet, $\varphi(g)$ ist tats\"achlich Permutation.
$\varphi$ ist ein Homomorphismus: Sei $\varphi(gh) = \pi$. Sei $N_ig=N_j und N_jh=N_m$. Dann ist $N_igh =N_m$. Also ist $\pi$ eine Permutation mit $i^\pi=m$. Gleichzeitig ist $\varphi(g) = \sigma$ mit $i^\sigma=j$ und $\varphi(h)=\tau$ mit $j^\tau=m$. Daher ist $i^{\sigma\tau}=j^\tau=m=i^\pi$, also $\pi=\sigma\tau$ und damit $\varphi(gh)=\varphi(g)\varphi(h)$.
Es gilt: $\Ker(\varphi)=\lbrace g\in G|Ug_ig=Ug_i,i=1,\ldots, n\rbrace = \lbrace g\in G|g_igg_i^{-1}\in U, i=1,\ldots, n\rbrace = \lbrace g\in G|g\in g_i^{-1}Ug_i, i=1,\ldots , n\rbrace = \bigcap_{g\in G}g^{-1}Ug$.
Sei $k\geq 2$. Sei $Ug_1\neq Ug_2$. Setze $g:=g_1^{-1}g_2$, dann ist $Ug_1g=Ug_1g_1^{-1}=Ug_2\neq Ug_1$. Also ist $\varphi(g)\neq 1\in \Symm_N$.
\end{beweis}

\begin{bemerkung*}\spspace
 \begin{enumerate}
  \item Setzen wir $U=\l<1\r>$, so erhalten wir wieder den Satz von Cayley mit genau dem dortigen Beweis.
  \item Die Rechtsmultiplikation auf den Rechtsnebenklassen ist eine sogenannte \emph{Gruppenoperation}. Zu jeder Gruppenoperation geh\"ort ein Homomorphismus in eine geeignete symmetrische Gruppe - in diesem Fall genau der oben konstruierte.
 \end{enumerate}

\end{bemerkung*}

\begin{definition}[inneres und \"au\ss{}eres Produkt]\spspace
\index{Produkt!direktes!inneres}
\index{Produkt!direktes!\"au\ss{}eres}
\begin{enumerate}
 \item Seien $G_1, G_2$ Gruppen. Das kartesische Produkt $G_1 \times G_2$ wird mit der komponentenweisen Verkn\"upfung zu einer Gruppe, dem sogenannten \emph{\"au\ss{}eren direkten Produkt} von $G_1$ und $G_2$.
 \item Seien $N_1,\ldots, N_k$ Normalteiler von $G$ mit:
 \begin{enumerate}
  \item $G=N_1N_2\cdots N_k$
  \item $N_i \cap N_1\cdots N_{i-1}N_{i+1}\cdots N_k = \l<1\r>$ f\"ur $i=1,\ldots, k$.

 \end{enumerate}
 Dann hei\ss{}t $G$ \emph{inneres direktes Produkt} von $N_1,\ldots, N_k$.
\end{enumerate}

 
\end{definition}

\begin{satz}\label{gnlmph}\label{1.10}
 Sei $G=N_1\cdot\ldots\cdot N_k$ ein inneres direktes Produkt der Normalteiler $N_1, \ldots, N_k$. Dann gilt:
 \begin{enumerate}
 \item \label{rt}\label{1.10.1}F\"ur $i\neq j$ kommutieren Elemente von $N_i$ und $N_j$, d.h. f\"ur $a\in N_i, b\in N_j$ gilt $ab=ba$.
 \item Jedes $a\in G$ besitzt eine (bis auf Reihenfolge) eindeutige Darstellung $a=a_1\cdot a_2 \cdot \ldots \cdot a_k$ mit $ a_i\in N_i$.
\end{enumerate}

\end{satz}

\begin{beweis}\spspace
 \begin{enumerate}
  \item \label{schnaps}\label{1.10.1b} $N_i, N_j \nt G$, daher gilt $\underbrace{a^{-1}b^{-1}a}_{\in N_i}\underbrace{b}_{\in N_j}=\underbrace{a^{-1}}_{\in N_i}\underbrace{b^{-1}ab}_{\in N_i} \in N_i\cap N_j \leq N_i\cap (N_1\cdot \ldots \cdot N_{i-1}N_{i+1}\cdot\ldots\cdot N_k) = \langle 1 \rangle$. Also folgt $a^{-1}b^{-1}ab=1 \Longleftrightarrow ab=ba$.

  \item Wegen $G=N_1\cdot\ldots\cdot N_k$ gibt es eine Darstellung $a=a_1\cdot\ldots\cdot a_k$ mit $a_i\in N_i$.\\Zur Eindeutigkeit: Sei zun\"achst $a=1=a_1\cdot\ldots\cdot a_k$. Nach \ref{1.10.1b} gilt: $ N_i \ni a_i^{-1}=a_1a_2\ldots a_{i-1}a_{i+1}\ldots a_k \in N_1\ldots N_{i-1}N_{i+1}\ldots N_k$. Damit ist $a_i^{-1}\in N_i \cap (N_1\ldots N_{i-1}N_{i+1}\ldots N_k) = \langle1\rangle$, also ist $a_i = 1$.\\
  Sei nun $a=a_1\ldots a_k=b_1\ldots b_k$ mit $a_i, b_i \in N_i$. Dann gilt nach \ref{1.10.1b}: $1=(a_1^{-1}b_1)(a_2^{-1}b_2)\cdot\ldots\cdot (a_k^{-1}b_k)$. Wie oben gezeigt gilt dann $a_i^{-1}b_i=1 \Longleftrightarrow a_i=b_i$ f\"ur $i=1,\ldots,k$.
 \end{enumerate}

\end{beweis}

\begin{satz} \label{hase}\label{1.11}
\index{Produkt!direktes!inneres}
\index{Produkt!direktes!\"au\ss{}eres}
 Sei $G=N_1\cdot\ldots\cdot N_k$ das innere direkte Produkt der Normalteiler $N_1,\ldots, N_k$. Dann ist $G$ isomorph zum \"au\ss{}eren direkten Produkt $G \cong N_1 \times \ldots \times N_k$. 
\end{satz}

\begin{beweis}
 Nach \ref{1.10} besitzt jedes $g\in G$ eine eindeutige Darstellung $g=a_1\cdot\ldots\cdot a_k$ mit $a_i \in N_i$. Wir definieren $$\varphi:G\to N_1\times \ldots \times N_k, g \longmapsto (a_1,\ldots, a_k).$$ Wegen der Eindeutigkeit der Darstellung ist $\varphi$ wohldefiniert und bijektiv. Sei $h=b_1\cdot\ldots\cdot b_k$ mit $b_i\in N_i$. Dann ist $gh=(a_1\cdot\ldots\cdot a_k)(b_1\cdot\ldots\cdot b_k)\stackrel{\ref{gnlmph}.\ref{rt}}{=}a_1b_1a_2b_2\ldots a_kb_k$ und $\varphi(gh)=(a_1b_1, a_2b_2,\ldots, a_kb_k)=(a_1, \ldots, a_k)\cdot (b_1, \ldots b_k)= \varphi(g)\varphi(h)$. Damit ist $\varphi$ ein Homomorphismus.
\end{beweis}

\begin{bemerkung} \label{bratwurst}\label{1.12}
\index{Produkt!direktes!inneres}
\index{Produkt!direktes!\"au\ss{}eres}
 Das \"au\ss{}ere Produkt $G_1\times \ldots \times G_k$ ist das innere Produkt der Normalteiler $\langle 1 \rangle\times \ldots \times \langle 1 \rangle\times G_i \times \langle 1 \rangle\times \ldots\times \langle 1 \rangle$. Wegen \ref{hase} unterscheiden wir im Folgenden nicht mehr zwischen \"au\ss{}erem und innerem direkten Produkt, denn isomorphe Gruppen sind vom Standpunkt der Gruppentheorie nicht unterscheidbar. Der Unterschied zwischen beiden besteht in der Sichtweise. Beim \"au\ss{}eren direkten Produkt haben wir mehrere ``kleine'' Gruppen gegeben und setzen daraus eine ``gro\ss{}e'' Gruppe zusammen. Beim inneren direkten Produkt stehen wir vor dem umgekehrten Problem. Eine gro\ss{}e Gruppe ist gegeben und wir suchen ihre ``Bestandteile'', d.h. ihre direkten Faktoren. Das gleiche Problem werden wir beim semidirekten Produkt haben.
\end{bemerkung}

\begin{definition}
\index{Untergruppe!diagonale}
 Sei $U\leq G_1\times G_2$ und seien $\pi_i:G_1\times G_2\to G_i$ die Projektionen. $U$ hei\ss{}t \emph{diagonal}, wenn $U\subsetneqq \pi_1(U)\times \pi_2(U)$.
\end{definition}

\begin{beispiel*}
 In $\ZZ_2\times\ZZ_2$ ist $U:=\lbrace (0,0), (1,1) \rbrace$ diagonal, denn $U < \pi_1(U)\times \pi_2(U)=\ZZ_2\times\ZZ_2$.
\end{beispiel*}

\begin{satz}\label{1.14}
\index{Untergruppe!diagonale}
 Seien $G_1,\ldots, G_r \nt G$ und $|G_1|,\ldots, |G_r|$ paarweise teilerfremd. Gilt dann $|G|=|G_1|\cdot \ldots \cdot |G_r|$, so ist $G=G_1\times \ldots \times G_r$ und jede Untergruppe $U$ von $G$ hat die Form $U=U_1\times \ldots\times U_r$ mit $U_i\leq G_i$, d.h. es gibt keine diagonalen Untergruppen in $G$.
\end{satz}

\begin{beweis}
 Wir zeigen durch Induktion \"uber $i$: $G_1\cdot\ldots\cdot G_i$ ist Untergruppe von $G$ und es gilt $|G_1\cdot \ldots\cdot G_i|=|G_1|\cdot\ldots\cdot |G_i|$ und $G_1\cdot\ldots\cdot G_i \cong G_1\times\ldots\times G_i$.\\
 Nach Induktionsvoraussetzung (IV) ist $G_1\cdot\ldots\cdot G_{i-1}\leq G$. Wegen $G_i\nt G$ folgt aus dem 1. Isomorphiesatz, dass auch $G_1\cdot\ldots\cdot G_{i-1}\cdot G_i \leq G$. Weiter gilt $|G_1\cdot\ldots\cdot G_{i-1}= |G_1|\cdot\ldots\cdot |G_{i-1}|$ nach Induktionsvoraussetzung, und diese Zahl ist teilerfremd zu $|G_i|$. Mit \ref{1.1} folgt $$|G_1\cdot\ldots\cdot G_i| = \frac{|G_1\cdot\ldots\cdot G_{i-1}|\cdot |G_i|}{|(G_1\cdot \ldots\cdot G_{i-1})\cap G_i|}=|G_1\cdot\ldots\cdot G_{i-1}|\cdot |G_i| = |G_1|\cdot\ldots\cdot |G_i|,$$ da nach dem Satz von Lagrange gilt $(G_1\cdot\ldots\cdot G_{i-1})\cap G_i=\langle 1 \rangle$.
Nach dem Satz \ref{1.11} folgt $G_1\cdot\ldots\cdot G_i \cong (G_1\cdot \ldots\cdot G_{i-1})\times G_i \stackrel{\textrm{IV}}{\cong} G_1\times \ldots\times G_{i-1}\times G_i$. 

Sei nun $U\leq G$ und $u\in U$. Nach \ref{1.10} gibt es eindeutig bestimmte $u_i\in G_i$ derart, dass $u=u_1\cdot\ldots\cdot u_r$. Sei $\sigma_i:=\ord(u_i)$ und sei $n_i := \frac{\sigma_1\cdot\ldots\cdot\sigma_r}{\sigma_i}$. Weil die $|G_i|$ paarweise teilerfremd sind, folgt $\ggT(\sigma_i, n_i)=1$. Deshalb gibt es $x, y\in \ZZ$ derart, dass $x\sigma_i + yn_i=1$. Nach \ref{1.10} kommutieren die $u_i$, also ist wegen $u_i^{\sigma_i}=1$
$$u^{yn_i}=(u_1\cdot\ldots\cdot u_r)^{yn_i}=u_1^{yn_i}\cdot\ldots\cdot u_r^{yn_i}=u_i^{yn_i}=u_i^{yn_i + x\sigma_i}=u_i^1=u_i.$$
Es gilt $$\pi_i(U)=\lbrace u_i\in U_i | \forall j\in \lbrace 1,\ldots,r \rbrace - \lbrace i \rbrace\exists u_j\in G_j:u=u_1\cdot\ldots\cdot u_r\in U\rbrace.$$
Ist aber $u=u_1\cdot\ldots\cdot u_r\in U$, so ist auch $u_i\in U$, d.h. $\pi_i(U)\subset U$ f\"ur alle $i$. Daher ist $\pi_1(U)\cdot\ldots\cdot\pi_r(U)\subset U$, also $U=\pi_1(U)\times \ldots\times \pi_r(U)$.
\end{beweis}

\begin{definition}[inneres semidirektes Produkt] 
\index{Produkt!direktes!inneres}
%\index{Produkt!direktes!\"au\ss{}eres}
\index{Komplement}
 Sei $N\nt G$. Gibt es eine Untergruppe $U\leq G$ derart, dass $UN=G$ und $U\cap N = \langle 1 \rangle$, so hei\ss{}t $U$ \emph{Komplement} zu $N$ in $G$ und $G$ hei\ss{}t \emph{inneres semidirektes Produkt von $N$ und $U$}.
\end{definition}

\begin{bemerkung}
 Sei $G=NU$ inneres semidirektes Produkt des Normalteilers $N$ mit $U$. Dann ist auch $G=NU$.\\
Die Abbildung $\psi:U\to \Aut(N), u\longmapsto (n\mapsto n^u)$ ist ein Gruppenhomomorphismus und es gilt $$(u_1n_1)(u_2n_2)=(u_1u_2)(n_1^{u^\psi}n_2)$$ f\"ur $u_1, u_2 \in U, n_1, n_2\in N$.
\end{bemerkung}

\begin{beweis}
 Wegen $N\nt G$ ist $UN=NU$. $\psi$ ist wohldefiniert, denn f\"ur $u\in U$ ist $$\varphi_u:G\to G, x\longmapsto u^{-1}xu$$ ein Automorphismus. Wegen $N\nt G$ ist $\varphi_u(n)\in N$ f\"ur $n\in N$, also ist $\varphi_u|_N\in \Aut(N)$. \\Die Homomorphieeigenschaft von $\psi$ ist klar.\\
F\"ur $u_1, u_2\in U, n_1, n_2\in N$ ist 
$$(u_1n_1)(u_2n_2)=u_1u_2u_2^{-1}n_1u_2n_2=u_1u_2n_1^{u_2}n_2=u_1u_2n_1^{u_2^\psi}n_2.$$
\end{beweis}

\begin{defundsatz} 
%\index{Produkt!direktes!inneres}
\index{Produkt!direktes!\"au\ss{}eres}
 Seien $U,N$ Gruppen und sei $\varphi:U\to \Aut(N)$ ein Gruppenhomomorphismus. Dann wird das kartesische Produkt $U\times N$ mit der Multiplikation 
$$(u_1,n_1)(u_2,n_2):=(u_1u_2,n_1^{(u_2^\varphi)}n_2)$$ zu einer Gruppe, dem sogenannten \emph{\"au\ss{}eren semidirekten Produkt von $U$ mit $N$ bez\"uglich $\varphi$}.\\
Schreibweise:$U \ltimes_\varphi N$ (Spitze zum Normalteiler).
\end{defundsatz}


\begin{beweis}
 Das Einselement ist $(1_U,1_N)$, Inverses zu $(u,n)$ ist $(u^{-1},(n^{-1})^{(u^{-1})^{\varphi}})$:
\begin{eqnarray*}(u,n)(u^{-1},(n^{-1})^{(u^{-1})^{\varphi}})=(uu^{-1},n^{(u^{-1})^\varphi}(n^{-1})^{(u^{-1})^{\varphi}})=\\=(1_U,n^{(u^{-1})^\varphi}(n^{(u^{-1})^\varphi})^{-1})=(1_U,1_N).\end{eqnarray*}
Zur Assoziativit\"at:
\begin{eqnarray*}\lbrack (u_1,n_1)(u_2,n_2)\rbrack (u_3,n_3) &=& (u_1u_2,n_1^{u_2^\varphi}n_2)(u_3,n_3) \\&=& (u_1u_2u_3,(n_1^{u_2^\varphi}n_2)^{u_3^\varphi})n_3) \\&=& (u_1u_2u_3,(n_1^{u_2^\varphi})^{u_3^{\varphi}}n_2^{u_3^\varphi}n_3) \\&=& (u_1u_2u_3,n_1^{(u_2u_3)^\varphi}n_2^{u_3^\varphi}n_3) \\&=& (u_1,n_1)(u_2u_3,n_2^{u_3^\varphi}n_3) \\&=& (u_1,n_1)\lbrack (u_2,n_2)(u_3,n_3) \rbrack.
\end{eqnarray*}
\end{beweis}

\begin{bemerkung}\spspace
\index{Produkt!direktes!inneres}
\index{Produkt!direktes!\"au\ss{}eres}
 \begin{itemize}
 \item Ist $\varphi$ der triviale Homomorphismus, der alles auf $\id_N$ abbildet, so ist $U \ltimes_\varphi N\cong U\times N$.
 \item $U \ltimes_\varphi N$ kann auch als inneres semidirektes Produkt von $U\times \langle 1 \rangle$ und $\langle 1 \rangle \times N$ gesehen werden. Die Unterscheidung zwischen inneren und \"au\ss{}erem semidirekten Produkt ist also wieder unn\"otig, Bemerkung \ref{1.12} gilt entsprechend.
 \item Ein wichtiges Problem der Gruppentheorie ist die Frage ``Wann gibt es zu einem Normalteiler $N \nt G$ ein Komplement?'' Eine relativ allgemeine Antwort liefert der 
\begin{satz*}[von Zassenhaus] \index{Zassenhaus!Satz von}
 Sind $|N|$ und $|G/N|$ teilerfremd, so besitzt $N$ ein Komplement in $G$.
\end{satz*}

\end{itemize}

\end{bemerkung}

\begin{beispiel}
\index{Diedergruppe}
 Seien $\langle d \rangle \cong \ZZ_n$ und $\langle s \rangle \cong \ZZ_2$ zyklisch. Durch 
\begin{eqnarray*}
\varphi:\langle s \rangle \to \Aut(\langle d \rangle ), s&\longmapsto&(x\mapsto x^{-1}\\ 1 &\longmapsto& \id_{\langle d \rangle}
\end{eqnarray*}
wird ein Homomorphismus gegeben. Das semidirekte Produkt $\Di_n:=\langle d \rangle \rtimes_\varphi \langle s \rangle$ der Ordnung $2n$ hei\ss{}t \emph{Diedergruppe}.

Bemerke: $\Di_n$ ist isomorph zur Symmetriegruppe des regelm\"a\ss{}igen $n$-Ecks. Das Element $d$ entspricht der Drehung um $\frac{360^\circ}{n}$, $s$ einer Spiegelung an einer Symmetrieachse.

\end{beispiel}

\begin{definition}[Kranzprodukt] \spspace
 \index{Kranzprodukt!regul\"ares}
 \index{Kranzprodukt!Permutations-}
\begin{enumerate}
 \item Seien $G,H$ Gruppen. Sei $|H|=n$, $H=\lbrace h_1,\ldots,h_n \rbrace$. Dann wird f\"ur $h\in H$ durch $h_i\cdot h= h_{\pi_h(i)}$ eine Permutation $\pi_h \in \Symm_n$ definiert. Die Abbildung
$$\vartheta_h:G^n\to G^n,(g_1,\ldots,g_n)\longmapsto (g_{\pi_h(1)},\ldots,g_{\pi_h(n)})$$
ist ein Automorphismus von $G^n$, dem $n$-fachen direkten Produkt von $G$ mit sich selbst. Durch $$\varphi: H\to \Aut(G^n), h\longmapsto \vartheta_n$$ wird ein Gruppenhomomorphismus definiert. Das semidirekte Produkt $$G \wr_r H := (G^n) \rtimes_\varphi N$$ hei\ss{}t \emph{regul\"ares Kranzprodukt von $G$ mit $H$}.
 \item Sei $G$ Gruppe, $H\leq \Symm_n$. Sei $h\in H$. Durch
$$\vartheta_h:G^n\to G^n, (g_1,\ldots,g_n)\longmapsto (g_{h(1)},\ldots,g_{h(n)})$$
wird ein Automorphismus von $G^n$ definiert. Die Abbildung
$$\varphi:H\to \Aut(G^n), h\longmapsto \vartheta_h$$
ist ein Gruppenhomomorphismus. Das semidirekte Produkt
$$G\wr_p H:=(G^n)\rtimes_\varphi H$$ hei\ss{}t \emph{Permutations-Kranzprodukt}.
Elemente von $G \wr_p H$ werden meistens in der Form $(f,h)$ geschrieben. Dabei ist $h\in H\leq \Symm_n$ und $f:\lbrace 1,\ldots,n\rbrace\to G$ eine Abbildung, die $(g_1,\ldots,g_n)\in G^n$ angibt
\end{enumerate}

\end{definition}

\begin{bemerkung*}
 \begin{enumerate}
  \item $|G\wr_r H| = |G|^{|H|}\cdot |H|$ und $|G\wr_p H|=|G|^n |H|$, wobei $H\leq \Symm_n$.
  \item $G\wr_r H$ ist ein Spezialfall von $G\wr_p H$ (Satz von Cayley).
  \item $G\wr_r H$ und $G\wr_p H$ k\"onnen durchaus verschieden sein. F\"r $G\cong \ZZ_2$ und $H\cong \Symm_3$ ist $|G\wr_r H|=2^6\cdot 6$ und $|G\wr_p H| = 2^3\cdot 6$, wenn wir $H$ als Permutationsgruppe auf drei Ziffern auffassen. In der Literatur wird meistens $G\wr H$ geschrieben.
 \end{enumerate}

\end{bemerkung*}


\section{Zentralisator, Normalisator und Kommutator}

\begin{definition}[Zentralisator, Normalisator und Zentrum]
 \label{zentralisator_normalisator_zentrum}
 \index{Zentralisator}
 \index{Normalisator}
 \index{Zentrum}
 Sei $H \leq G$ und $X \subset G$.
 \begin{enumerate}
  \item $\Cen_H \l( X \r) = \l\{ h \in H \mid h^{-1}xh=x \quad \forall x \in X \r\}$ (Vertauschung elementweise) hei\ss{}t \emph{Zentralisator von X in H}. Ist $\l| X \r| = 1$, so schreiben wir $\Zen_H \l( x \r)$.
  \item $\Norm_H\l(X\r)=\l\{h\in H\mid h^{-1}Xh=X\r\}$ (Vertauschung als Menge, d.h.  $h^{-1}xh=y\in X$) hei\ss{}t \emph{Normalisator von $X$ in $H$}.
  \item $\Zen \l( G \r) := \Cen_G\l( G \r) = \l\{ g \in G \mid g^{-1} x g = x \quad \forall x \in G \r\}$ (Die Elemente, die mit allen vertauschen) hei\ss{}t \emph{Zentrum von $G$}.
 \end{enumerate}
\end{definition}

\begin{satz}
 \label{aussagen_zu_ugr}
 Sei $G$ Gruppe, $X \subset G$ Teilmenge und $H, U \leq G$. Dann gilt:
 \begin{enumerate}
  \item $\Cen_H\l(X\r)$ und $\Norm_H\l(X\r)$ sind Untergruppen von $H$ und $G$.
  \item $U \nt G \Leftrightarrow \Norm_G\l( U \r) = G$
  \item $\Norm_G\l(U\r)$ ist die gr\"o\ss{}te Untergruppe von $G$, in der $U$ Normalteiler ist, d.h.:
   \begin{itemize}
    \item $U \nt \Norm_G\l(U\r)$
    \item Ist $U \nt V \Rightarrow V \leq \Norm_G\l(U\r)$
   \end{itemize}
  \item $\Cen\l(G\r)$ ist abelsch
  \item F\"ur alle $X \subset G$ gilt: $\Zen\l(G\r) \leq \Cen_G\l(X\r) \leq \Norm_G\l(X\r)$
  \item $\Zen\l(G\r) \Char G$
  \item \label{ugr_zen_nt} $U \leq \Zen\l(G\r) \Rightarrow U \nt G$
  \item $\Zen\l(G\r) = G \Leftrightarrow G$ abelsch
  \item $G$ abelsch $\Leftrightarrow G/\Zen\l(G\r) $ zyklisch
  \item $\Zen\l(G_1 \times G_2\r) = \Zen\l(G_1\r) \times \Zen\l(G_2\r)$
 \end{enumerate}
\end{satz}

\begin{beweis}
 Wir beweisen hier nur die Aussagen 1) und 9), 2) bis 8) und 10) sind trivial und werden dem geneigten Leser \"uberlassen.
 \begin{enumerate}
  \item Seien $u,v \in \Cen_H\l(X\r)$, d.h. $u^{-1}xu=x, v^{-1}xv=x$ f\"ur alle $x\in X$. Dann:
   \begin{equation*}
    \l(uv\r)^{-1}x\l(uv\r)=v^{-1}\l(u^{-1}xu\r)v=v^{-1}xv=x
   \end{equation*}
   Also ist $uv \in \Cen_H\l(X\r)$. Damit ist $\Cen_H\l(X\r)$ eine Untergruppe. Analog f\"ur $\Norm_H\l(X\r)$ ($X$ statt $x$).
  \setcounter{enumi}{8}
  \item Sei zun\"achst $G$ abelsch. Dann ist $G = \Zen\l(G\r). \Rightarrow \l|G/\Zen\l(G\r)\r|=1$, also $G/\Zen\l(G\r)$ zyklisch.

   Sei nun umgekehrt $x \in G$, so dass $G/\Zen\l(G\r) = \l< x\Zen\l(G\r)\r>$. Seien $a, b \in G$. Dann existiert $i, j \in \NN_0 : a\Zen\l(b\r) = x^i\Zen\l(G\r), b\Zen\l(G\r) = x^j\Zen\l(G\r)$. Also gibt es  $w, z \in \Zen\l(G\r)$, so dass $a=x^iw, b = x^jz$. Damit ist $ab=x^iwx^jz=x^{i+j}zw=x^jzx^iw=ba$, denn $w,z$ vertauschen mit allen Elementen. Also ist $G$ abelsch.
 \end{enumerate}
 \qed
\end{beweis}

\begin{satz}
 $G/\Zen\l(G\r) \cong \Inn G$
\end{satz}

\begin{beweis}
 Durch $\psi: G \to \Inn G, g \mapsto .^g$ wird ein Gruppenhomomorphismus definiert, denn es gilt:
 \begin{equation*}
  \psi\l(gh\r) = .^gh = \l(.^g\r)^h=\psi\l(g\r)\psi\l(h\r)
 \end{equation*}
 Weiter gilt:
 \begin{equation*}
  \ker\psi = \l\{g\in G \mid .^g = \id\r\}=\l\{g\in G \mid g^{-1}xg=x \quad \forall x\in G\r\}=\Zen\l(G\r)
 \end{equation*}
 Weil $\psi$ surjektiv ist, folgt mit dem Homomorphiesatz \ref{homomorphiesatz}:
 \begin{equation*}
  G/\Zen\l(G\r) \cong \Inn G
 \end{equation*}
 und mit dem vorangegangenen Satz folgt: $\Inn G$ zyklisch $\Rightarrow \Inn G=\l\{id\r\}$
 \qed
\end{beweis}

\begin{satz}
 Sei $U \leq G$. Dann gilt:
 \begin{equation*}
  \Cen_G\l(U\r) \nt \Norm_G\l(U\r)
 \end{equation*}
 und $\Norm_G\l(U\r)/Cen_g\l(U\r)$ ist isomorph zu einer Untergruppe von $\Aut U$. Ist insbesondere $U \nt G$, so gilt:
 \begin{equation*}
  \Cen_G\l(U\r) \nt G
 \end{equation*}
\end{satz}

\begin{beweis}
 Die Abbildung
 \begin{equation*}
  \psi: \Norm_G\l(U\r) \to \Aut\l(U\r), a \mapsto \l( u \mapsto u^a\r)
 \end{equation*}
 ist ein Gruppenhomomorphismus. Wegen $U \nt \Norm_G\l(U\r)$ ist die Abbildung $\l(u\mapsto u^a\r) \in \Aut U$. $\psi$ ist ein Homomorphismus, da
 \begin{equation*} 
  \psi\l(ab\r)=\l(u \mapsto u^{ab}\r)=\l(u \mapsto \l(u^a\r)^b\r)=\psi\l(a\r)\psi\l(b\r)
 \end{equation*}
 Es gilt:
 \begin{eqnarray*}
  \ker\psi&=&\l\{a \in \Norm_G\l(U\r) \mid u^a=a \quad \forall a \in U\r\}=\\&=& \l\{a\in \Norm_G\l(U\r) \mid a^{-1}ua=u \quad \forall u \in U\r\}=\\&=&\Cen_G\l(U\r)
 \end{eqnarray*}
 wobei die letzte Gleichung aus $\Cen_G\l(U\r)\leq\Norm_G\l(U\r)$ folgt.

 Kerne von Homomorphismen sind Normalteiler, also ist $\Cen_G\l(U\r) \nt \Norm_G\l(U\r)$. Mit dem Homomorphiesatz \ref{homomorphiesatz} folgt
 \begin{equation*}
  \Norm_G\l(U\r)/\Cen_G\l(U\r) \cong \Img\psi \leq \Aut U
 \end{equation*}
 \qed
\end{beweis}

\begin{satz}
 \label{satz_uber_cen_und_norm}
 Seien $K\leq H \leq G$ Untergruppen von $G, Y \subset X \subset G$ Teilmengen, $g,a \in G$ und $h \in H$. Dann gilt:
 \begin{enumerate}
  \item $\Cen_H\l(X\r)^g = \Cen_{H^g}\l(X^g\r)$, also $g^{-1}\Cen_H\l(X\r)g = \Cen_{g^{-1}Hg}\l(g^{-1}Xg\r)$
  \item $\Norm_H\l(X\r)^g = \Norm_{H^g}\l(X^g\r)$, also $g^{-1}\Norm_H\l(X\r)g = \Norm_{g^{-1}Hg}\l(g^{-1}Hg\r)$
  \item $\Cen_G\l(g^{-1}ag\r) = g^{-1}\Cen_G\l(a\r)g$
  \item $\Norm_G\l(g^{-1}Hg\r) = g^{-1}\Cen_G\l(H\r)g$
  \item $\Cen_H\l(h\r) = \Cen_G\l(h\r)\cap H$
  \item $\Norm_H\l(K\r) = \Norm_G\l(K\r)\cap H$
  \item $\Cen_H\l(X\r) \leq\Cen_G\l(X\r)$
  \item $\Norm_H\l(X\r) \leq\Norm_G\l(X\r)$
  \item \label{cenx_ugr_ceny} $\Cen_H\l(X\r) \leq \Cen_H\l(Y\r)$. Die Aussage $\Norm_H\l(X\r) \leq \Norm_H\l(Y\r)$ ist im Allgemeinen falsch.
 \end{enumerate}
\end{satz}

\begin{beweis}
 Hier beweisen wir nur den Punkt \emph{1.} und bringen ein Gegenbeispiel zur \emph{2.} Aussage von Punk \emph{9.} Denn \emph{2.} folgt analog \emph{1.}, \emph{3.} \& \emph{4.} sind Spezialf\"alle von \emph{1.} und \emph{2.} und \emph{5.} bis \emph{9.} sind sofort aus der Definition \ref{zentralisator_normalisator_zentrum} klar.
 \begin{enumerate}
  \item
   \begin{eqnarray*}
    h^g \in \Cen_{H^g}\l(X^g\r) &\Leftrightarrow& \l(h^g\r)^{-1}x^g h^g = x^g \quad \forall x \in X\\&\Leftrightarrow& g^{-1}h^{-1}gg^{-1}xgg^{-1}hg=g^{-1}xg \quad \forall x \in X\\&\Leftrightarrow&h^{-1}xh=x \quad \forall x \in X\\&\Leftrightarrow& h \in \Cen_H\l(X\r)
   \end{eqnarray*}
  \setcounter{enumi}{8}
  \item Gegenbeispiel zur zweiten Aussage:
   \begin{eqnarray*}
    H=G=\Symm_4, X=\Alt_4, Y=\l< \l(1 2 3\r) \r>\\\l(3 4\r)\l(1 2 3\r)\l(3 4\r) = \l(1 2 4\r) \in \l<\l(1 2 3\r)\r>=Y\\\Rightarrow \l(3 4\r) \in \Norm_{\Symm_4}\l(Y\r) \Rightarrow \Norm_{\Symm_4}\l(Y\r) < \Symm_4
   \end{eqnarray*}
  Aber $\Norm_{\Symm_4}\l(\Alt_4\r) = \Symm_4$.  Widerspruch!
 \end{enumerate}
 \qed
\end{beweis}

\begin{definition}[Konjugationsklasse]
\index{Konjugationsklasse}
\label{konjugationsklasse}
 Sei $g\in G,H\leq G$. Die Menge $K:=\l\{h^{-1}gh\mid h\in H\r\}$ hei\ss{}t \emph{$h$-Konjugationsklasse von $g$}. $G$-Konjugationsklassen hei\ss{}en einfach \emph{Konjugationsklassen}.
\end{definition}

\begin{bemerkung*}
 $H$-Konjugation ist eine \"Aquivalenzrelation, die \"Aquivalenzklassen sind die $H$-Konjugationsklassen. Sind $K_1, \ldots, K_r$ alle Konjugationsklassen von $G$, so gilt:
 \begin{equation*}
  G = \bigcupdot_{i=1}^r K_i
 \end{equation*}
 wobei $K_i \cap K_j = \emptyset$ falls $i\neq j$.
\end{bemerkung*}

\begin{lemma}
\label{anzahl_elemente_konjugationsklasse}
 Sei $g\in G, H \leq G$. Die $H$-Konjugationsklasse $K$ von $g \in G$ enth\"alt genau
 \begin{equation*}
  \l|K\r|=\frac{\l| H \r|}{\l| \Cen_H\l( g \r) \r|}=\frac{\l| H \r|}{\l| \Cen_G\l( g \r) \cap H \r|}
 \end{equation*}
 Elemente.
\end{lemma}

\begin{beweis}
 F\"ur $a,b \in H$ gilt:
 \begin{eqnarray*}
  a^{-1}ga=b^{-1}gb &\Leftrightarrow& ba^{-1}g=gba^{-1}\\&\Leftrightarrow&ba^{-1} \in \Cen_H\l(g\r)\Leftrightarrow b\in \Cen_H\l(g\r)a
 \end{eqnarray*}
 Damit ist
 \begin{equation*}
  \l| K \r|=\frac{\l|H\r|}{\l| \Cen_H\l(g\r) \r|}
 \end{equation*}
 \qed
\end{beweis}

\begin{satz}[Klassengleichung]
\index{Klassengleichung}
\label{klassengleichung}
 Sei $x_1, \ldots, x_r$ eine Transversale der nicht-zentralen Konjugationsklassen von $G$ (das hei\ss{}t der Klassen, die nicht in $\Zen\l(G\r)$ liegen). Dann gilt:
 \begin{equation*}
  \l|G\r|=\sum_{i=1}^r\frac{\l|G\r|}{\l|\Cen_G\l(x_i\r)\r|} + \l|\Zen\l(G\r)\r|
 \end{equation*}
\end{satz}

\begin{beweis}
 Es gilt:
 \begin{equation*}
  \frac{\l|G\r|}{\l|\Cen_G\l(x\r)\r|}=1 \Leftrightarrow G=\Cen_G\l(x\r) \Leftrightarrow x \in \Zen\l(G\r)
 \end{equation*}
 Also bildet jedes Element des Zentrums eine eigene Konjugationsklasse. Damit ist
 \begin{equation*}
  G=\bigcupdot_{i=1}^rK\l(x_i\r) \cupdot \Zen\l(G\r)
 \end{equation*}
 die disjunkte Vereinigung der $K\l(x_i\r)$ mit $\Zen\l(G\r)$ und die Behauptung folgt aus Lemma \ref{anzahl_elemente_konjugationsklasse}
 \qed
\end{beweis}

\begin{definition}[$p$-Gruppe, $p'$-Gruppe]
\index{$p$-Gruppe}
\index{$p'$-Gruppe}
\label{pgruppe}
 Sei $p$ prim. Eine Gruppe $G$ hei\ss{}t \emph{$p$-Gruppe}, wenn $\l|G\r| = p^n$ f\"ur ein $n\in \NN$. $G$ hei\ss{}t $p'$-Gruppe, wenn $p \nmid \l|G\r|$.
\end{definition}

\begin{satz}
\label{pgruppe_hat_nichttriviales_zentrum}
 Jede $p$-Gruppe hat ein nichttriviales Zentrum, das hei\ss{}t $Z\l(G\r) > \quad \l<1\r>$.
\end{satz}

\begin{beweis}
 Sei $x_1,\ldots,x_r$ eine Transversale der nicht-zentralen Konjugationsklassen. Nach Satz \ref{klassengleichung} Klassengleichung gilt:
 \begin{equation*}
  \l|G\r|=\sum_{i=1}^r\frac{\l|G\r|}{\l|\Cen_G\l(x_i\r)\r|} + \l|\Zen\l(G\r)\r|
 \end{equation*}
 Da die Konjugationsklasse $K\l(x_i\r)$ nicht zentral ist, gilt $\l|K\l(x_i\r)\r| > 1$ (siehe Beweis von \ref{konjugationsklasse}, also ist
 \begin{equation*}
  1 < \l|K\l(x_i\r)\r| = \frac{\l|G\r|}{\l| \Cen_G\l(x_i\r) \r|} \mid \l|G\r| = p^n
 \end{equation*}
 und damit ist $p \mid \frac{\l|G\r|}{\l|\Cen_G\l(x_i\r)\r|}$. In der Klassengleichung sind sowohl die linke Seite als auch die Summanden $\frac{\l|G\r|}{\l|\Cen_G\l(x_i\r)\r|}$ durch $p$ teilbar $\Rightarrow p \mid \Zen\l(G\r)$.
 \qed
\end{beweis}

\begin{folgerung}
 Sei $p$ prim. Ist $G$ Gruppe mit $\l|G\r|=p^2$, so gilt:
 \begin{equation*}
  G \cong \ZZ_{p^2} \qquad \mbox{ oder } \qquad G \cong \ZZ_{p} \times \ZZ_{p}
 \end{equation*}
\end{folgerung}

\begin{beweis}
 Nach Satz \ref{pgruppe_hat_nichttriviales_zentrum} gilt $\l|\Zen\l(G\r)\r| = p$ oder $p^2$. Ist $\l|\Zen\l(G\r)\r| = p^2$ so ist $g = \Zen\l(G\r) \Rightarrow G$ abelsch. Ist $\l|\Zen\l(G\r)\r|=p$, so ist $\l| G/\Zen\l(G\r) \r| = p$, also ist $G/\Zen\l(G\r) \cong \ZZ_p$ zyklisch.

 Daraus folgt nach \ref{cenx_ugr_ceny} dass $G$ abelsch (und damit $\l|Z\l(G\r)\r| = p^2$). Mit dem Hauptsatz \"uber endliche abelsche Gruppen \ref{hauptsatz_ueber_endliche_abelsche_gruppen} folgt: $G$ ist direktes Produkt zyklischer Gruppen, daf\"ur gibt es genau die zwei angegebenen M\"oglichkeiten.
 \qed
\end{beweis}

\begin{definition}[Kommutator]
\index{Kommutator}
\index{Kommutatorgruppe}
\label{kommutator}
 Seien $g, h \in G, X, Y, Z \subset G$. Man definiert:
 \begin{enumerate}
  \item $\l[g, h\r] := g^{-1}h^{-1}gh$ hei\ss{}t \emph{Kommutator von $g$ und $h$}
  \item $\l[X, Y\r] := \l< \l[ x,y \r] \mid x\in X, y\in Y\r>$
  \item $\l[X,Y,Z\r] := \l[\l[X,Y\r],Z\r]$
  \item $G' := \l<\l[g,h\r] \mid g,h\in G\r> = \l[G,G\r]$ hei\ss{}t \emph{Kommutatorgruppe von $G$}
  \item Iterierte Kommutatorbildung liefert ``h\"ohere'' Kommutatorgruppen:
   \begin{equation*}
    G'' := \l[G', G'\r], \ldots, G^{\l(n\r)} := \l[G^{\l(n-1\r)},G^{\l(n-1\r)}\r]
   \end{equation*}
 \end{enumerate}
 Beachte: $G'$ ist das Erzeugnis aller Kommutatoren. Die Menge aller Kommutatoren bildet im Allgemeinen keine Untergruppe.
\end{definition}

\begin{lemma}[Rechenregeln]
\label{kommutator_rechenregeln}
 Wir werden nun einige Rechenregeln zum Kommutator beweisen:
 \begin{enumerate}
  \item \label{kommutator_inverses_element} $\l[g,h\r]^{-1}=\l[h,g\r]$
  \item Ist $ \varphi: G \to H$ Gruppenhomomorphismus, so gilt:
   \begin{equation*}
    \varphi \l(\l[ g,h \r]\r) = \l[\varphi\l(g\r), \varphi\l(h\r)\r]
   \end{equation*}
   Ist $H$ abelsch, so folgt: $G' \leq \Ker\varphi$.
  \item $\l[g,g\r]^u = \l[g^u, h^u\r]$
  \item $\l[gh,u\r] = \l[g,u\r]^h \l[h,u\r]$
  \item $\l[u,gh\r] = \l[u,h\r]\l[u.g\r]^h$
  \item Elemente von $G'$ sind Produkte von Kommutatoren
  \item Kommutiert $\l[g,h\r]$ mit $g$, so gilt f\"ur alle $n \in \ZZ: \l[g^n,h\r] = \l[g,h\r]^n$
  \item Kommutiert $\l[g,h\r]$ mit $h$, so gilt f\"ur alle $n \in \ZZ: \l[g,h^n\r] = \l[g,h\r]^n$
  \item Kommutiert $\l[g,h\r]$ mit $g$ und $h$, so gilt f\"ur alle $n \in \ZZ:$
   \begin{equation*}
    \l(gh\r)^n = g^nh^n\l[g,h\r]^{\binom{n}{2}}
   \end{equation*}
\end{enumerate}
\end{lemma}

\begin{beweis}
 Auch hier beweisen wir nicht alle Punkte, denn \emph{2.} ist klar, wenn man die Homomorphieeigenschaft von $\varphi$ betrachtet, \emph{3.} ist ein Speziallfall von \emph{2. 5.} folgt analog \emph{4.} und \emph{8.} analog zu \emph{7. 6.} folgt aus \emph{1.}
 \begin{enumerate}
  \item $\l[g,h\r]^{-1} = h^{-1}g^{-1}hg=\l[h,g\r]$
  \setcounter{enumi}{3}
  \item
   \begin{eqnarray*}
    \l[g,u\r]^h\l[h,u\r]&=&h^{-1}g^{-1}u^{-1}guhh^{-1}u^{-1}u=h^{-1}g^{-1}u^{-1}guu^{-1}hu=\\&=&\l(h^{-1}g^{-1}\r)u^{-1}\l(gh\r)u=\l(gh\r)^{-1}u^{-1}\l(gh\r)u=\\&=&\l[gh,u\r]
   \end{eqnarray*}
  \setcounter{enumi}{6}
  \item F\"ur $n=0$ und $n=1$ klar. Induktion \"uber $n$: Weil $\l[g,h\r]$ mit $g^{n-1}$ vertauschbar ist, gilt:
   \begin{eqnarray*}
    \l[g^n,h\r]&=&\l[gg^{n-1},h\r]\stackrel{4.}{=}\l[g,h\r]^{g^{n-1}}\l[g^{n-1},h\r]=\\&=&\l[g,h\r]\l[g,h\r]^{n-1}=\l[g,h\r]^n
   \end{eqnarray*}
  F\"ur $n>0$ gilt nach \emph{4.} weiter:
   \begin{eqnarray*}
    1=\l[g^ng^{-n},h\r]&=&\l[g^n,h\r]^{g^{-n}}\l[g^{-n},h\r]=\l(\l[g,h\r]^n\r)^{g^{-n}}\l[g^{-n},h\r]=\\&=&\l[g,h\r]^n\l[g^{-n},h\r]\\&\Rightarrow& \l[g^{-n},h\r]=\l[g,h\r]^{-n}
   \end{eqnarray*}
  \setcounter{enumi}{8}
  \item Induktion \"uber $n$: $n=1$ klar. Wir benutzen, dass $\l[h,g\r]=\l[g,h\r]^{-1}$ mit $g$ und $h$ vertauscht:
   \begin{eqnarray*}
    \l(gh\r)^n&=&\l(gh\r)^{n-1}\l(gh\r)=g^{n-1}h^{n-1}\l[hg\r]^{\binom{n-1}{2}}gh=\\&=&g^{n-1}h^{n-1}gh\l[h,g\r]^{\binom{n-1}{2}}=\\&=&g^{n-1}gh^{n-1}h^{-\l(n-1\r)}g^{-1}h^{n-1}gh\l[h,g\r]^{\binom{n-1}{2}}=\\&=&g^nh^{n-1}\l[h^{n-1},g\r]h\l[h,g\r]^{\binom{n-1}{2}}=\\&\stackrel{7.}{=}&g^nh^{n-1}\l[h,g\r]^{n-1}h\l[h,g\r]^{\binom{n-1}{2}}=\\&=&g^nh^n\l[h,g\r]^{n-1+\frac{1}{2}\l(n-1\r)\l(n-2\r)}=g^nh^n\l[h,g\r]^{\l(n-1\r)\l(1+\frac{1}{2}n-1\r)}=\\&=&g^nh^n\l[h,g\r]^{\binom{n}{2}}
   \end{eqnarray*}
 \end{enumerate}
 \qed
\end{beweis}

\begin{lemma}
 Seien $U,V\leq G$. Dann gilt:
 \begin{enumerate}
  \item $\l[U,V\r]=\l[V,U\r]$
  \item $\l[U,V\r]\leq U \Leftrightarrow V \leq \Norm_G\l(U\r)$
  \item $\l[U,V\r] \nt \l<U,V\r>$
 \end{enumerate}
\end{lemma}

\begin{beweis}\spspace
 \begin{enumerate}
  \item klar mit \ref{kommutator_rechenregeln}.1
  \item Sei $v\in V$, dann gilt f\"ur alle $u \in U$:
   \begin{equation*}
    \l[u,v\r]=u^{-1}\l(v^{-1}uv\r) \in U \Leftrightarrow v^{-1}uv\in U \Leftrightarrow v \in \Norm_G\l(U\r)
   \end{equation*}
  \item Sei $u,u' \in U, v,v' \in V$. Dann gilt nach \ref{kommutator_rechenregeln} 4. und 5.:
   \begin{eqnarray*}
    \l[u,v\r]^{v'}&=&\l[u,v'\r]^{-1} \l[u,vv'\r] \in \l[U,V\r]\\
    \l[u,v\r]^{u'}&=&\l[uu',v\r]\l[u',v\r]^{-1} \in \l[U,V\r]
   \end{eqnarray*}
   Also ist $\l[U,V\r] \nt \l<U,V\r>$.
 \end{enumerate}
 \qed
\end{beweis}

\begin{satz} \label{2.15}
 F\"ur die Kommutatorgruppe $G'$ von $G$ gilt:
 \begin{enumerate}
  \item $G$ ist abelsch $\Leftrightarrow G' = \l<1\r>$.
  \item $G' \Char G$ (also ist $G'$ insbesondere Normalteiler von $G$).
  \item $G/G'$ ist abelsch.
  \item F\"ur $H\leq G$ gilt: $H \nt G$ und $G/H$ abelsch $\Leftrightarrow G'\leq H$. \label{2.15.4}
 \end{enumerate}
\end{satz}

\begin{beweis}
 Hier beweisen wir nur Punkt \emph{4.}, denn \emph{1.} ist klar, \emph{2.} auch wegen $\l[g,h\r]=\l[\alpha\l(g\r),\alpha\l(h\r)\r]$ f\"ur alle $\alpha \in \Aut G$ und \emph{3.} ist ein Spezialfall von \emph{4.}
 \begin{enumerate}
  \setcounter{enumi}{3}
  \item Sei $a,b \in G$. Dann folgt aus $\l(aH\r)\l(bH\r)=\l(bH\r)\l(aH\r)$:
   \begin{equation*}
    H=\l(aH\r)^{-1}\l(bH\r)^{-1}\l(aH\r)\l(bH\r)=\l(a^{-1}b^{-1}ab\r)H
   \end{equation*}
   $\Rightarrow a^{-1}b^{-1}ab \in H \Rightarrow G' \leq H$

   Sei nun umgekehrt $G' \leq H \leq G$. F\"ur $g \in G, h\in H$ gilt:
   \begin{equation*}
    g^{-1}Hg = hh^{-1}g^{-1}hg = h \l[h,g\r] \in HG' = H
   \end{equation*}
   $\Rightarrow H \nt G$. Seien $a,b\in G$, dann gilt:
   \begin{equation*}
    \l(aH\r)^{-1}\l(bH\r)^{-1}\l(aH\r)\l(bH\r)=\l(a^{-1}b^{-1}ab\r)H\stackrel{G'\leq H}{=}H
   \end{equation*}
   $\Rightarrow \l(aH\r)\l(bH\r)=\l(bH\r)\l(aH\r) \Rightarrow G/H$ ist abelsch.
 \end{enumerate}
 \qed
\end{beweis}

\begin{bemerkung}
 Sei $K$ eine Konjugationsklasse von $G$ und $h \in K$. Dann gilt: $K \subset hG'$ und insbesondere $\l|K\r|\leq \l|G'\r|$
\end{bemerkung}

\begin{beweis}
 Sei $g \in G$ beliebig. Dann ist $h^{-1}g^{-1}hg \in G'$, also $g^{-1}hg \in hG'$ und damit $K \subset hG'$. Damit ist $\l|K\r| \leq \l|hG'\r|=\l|G'\r|$.
 \qed
\end{beweis}

\begin{satz} \label{2.17}
 Seien $G,H$ Gruppen, $N \nt G$ und $U \leq G$. Dann gilt f\"ur alle $n \geq 0$:
 \begin{enumerate}
  \item $U^{\l(n\r)} \leq G^{\l(n\r)}$
  \item $\l(G/N\r)^{\l(n\r)} = G^{\l(n\r)}N/N \cong G^{\l(n\r)}/\l(G^{\l(n\r)}\cap N\r)$
  \item $\l(G \times H\r)^{\l(n\r)} = G^{\l(n\r)} \times H^{\l(n\r)}$
 \end{enumerate}
\end{satz}

\begin{beweis}
 \emph{1.} und \emph{3.} sind mit Induktion sofort klar.
 \begin{enumerate}
  \setcounter{enumi}{1}
  \item Induktion \"uber $n$. $n=0$ klar, $n \to n+1$:
   \begin{equation*}
    \l(G/N\r)^{\l(n+1\r)}=\l(\l(G/N\r)^{\l(n\r)}\r)'\stackrel{IV}{=}\l(G^{\l(n\r)}N/N\r)'
   \end{equation*}
   F\"ur $g \in G^{\l(n\r)}$ und $u\in N$ is $guN=gN$. Damit ist
   \begin{eqnarray*}
    \l(G^{\l(n\r)}N/N\r)'&=&\l<h_1^{-1}Nh_2^{-1}Nh_1Nh_2N \mid h_1,h_2 \in G^{\l(n\r)}\r>= \\&=&\l<h_1^{-1}h_2^{-1}h_1h_2N \mid h_1,h_2 \in G^{\l(n\r)}\r>=\\&=&\l<\l[h_1,h_2\r]N \mid h_1,h_2 \in G^{\l(n\r)}\r>=G^{\l(n+1\r)}N/N
   \end{eqnarray*}
   Damit ist die erste Gleichung gezeigt, die zweite folgt aus dem ersten Isomorphiesatz \ref{isomorphiesatz1}.
 \end{enumerate}
 \qed
\end{beweis}

\begin{beispiel*}
 Beispiele f\"ur charakteristische Untergruppen:
 \begin{itemize}
  \item $\l<1\r> \Char G, G \Char G$
  \item $\Zen\l(G\r) \Char G$
  \item $G' \Char G$
 \end{itemize}
\end{beispiel*}

\begin{satz}
 Sei $A \Char G$. Dann gilt: $\Cen_G\l(A\r) \Char G$.
\end{satz}

\begin{beweis}
 Sei $\alpha \in \Aut G$. Ist $x \in \Cen_G\l(A\r)$ und $a \in A$, so gilt:
 \begin{equation*}
  \alpha\l(a\r) = \alpha\l(x^{-1}ax\r)=\alpha\l(x\r)^{-1}\alpha\l(a\r)\alpha\l(x\r)
 \end{equation*}
 Durchl\"auft $a$ die Gruppe $A$, so durchl\"auft auch $\alpha\l(a\r)$ ganz $A$. Daher ist $\alpha\l(a\r) \in \Cen_G\l(A\r)$, also $\alpha\l(\Cen_G\l(A\r)\r) \leq \Cen_G\l(A\r)$ und deshalb $\Cen_G\l(A\r) \Char G$.
 \qed
\end{beweis}

\begin{satz}
 Sei $H \nt G$ mit $\ggT\l(\l|H\r|, \l|G/H\r|\r)=1$. Dann gilt: $H \Char G$
\end{satz}

\begin{beweis}
 Sei $\alpha \in \Aut G, \l| H \r| =: m$ und $\l| G/H \r| =: n$. Dann ist $\l| G \r| = mn$ und $\ggT \l( m, n \r) = 1$. Sei $\hat{H} := \alpha\l(H\r)$. Wir betrachten $H\hat{H}$: Nach dem 1. Isomorphiesatz \ref{isomorphiesatz1} gilt, dass $H\hat{H} \leq G$. Sei $\l|H\cap \hat{H}\r| =: d$. Dann gilt: $d \mid m$ und nach \ref{produkt_von_teilmengen} ist
 \begin{equation*}
  \l|H\hat{H}\r| = \frac{\l|H\r| \l|\hat{H}\r|}{\l|H\cap \hat{H}\r|} = \frac{m^2}{d} \mid mn
 \end{equation*}
 Wegen $\ggT\l(m,n\r)=1$ folgt $d=m$ und somit $H\cap \hat{H}=H$, also $\hat{H}=H$ und daher $\alpha\l(H\r)=H$.
 \qed
\end{beweis}

\begin{satz}
 \label{aussagen_zu_charakteristischen_ugr}
 Seien $A, B \leq G$. Dann gilt:
 \begin{enumerate}
  \item \label{aussagen_zu_charakteristischen_ugr_1} $A \Char G \Rightarrow A \nt G$.
  \item $A, B \nt G \Rightarrow A \cap B \nt G, AB \nt G$\\$A, B \Char G \Rightarrow A\cap B \Char G, AB \Char G$
  \item \label{aussagen_zu_charakteristischen_ugr_3} $A \Char B \nt G \Rightarrow A \nt G$\\$A \Char B \Char G \Rightarrow A \Char G$
  \item Vorsicht:
   \begin{eqnarray*}
    A \nt B \Char G &\stackrel{i.A.}{\nRightarrow}& A \nt G\\
    A \nt B \nt G &\stackrel{i.A.}{\nRightarrow}& A \nt G\\
    A \leq B \leq G, A \Char G &\stackrel{i.A.}{\nRightarrow}& A \Char B
   \end{eqnarray*}
  \item Sei $A \leq B \leq G$. Im Fall $A \nt G$ gilt: $B/A \nt G/A \Leftrightarrow B \nt G$. Im Fall $A \Char G$ gilt: $B/A \Char G/A \Rightarrow B \Char G$. Die Umkehrung, aus $B \Char G$ folgt $B/A \Char G/A$, gilt im Allgemeinen nicht.
 \end{enumerate}
\end{satz}

\begin{beweis}
 Punkt \emph{4.} bleibt dem Leser als \"Ubungsaufgabe \"uberlassen.
 \begin{enumerate}
  \item $A \Char G \Rightarrow \alpha\l(A\r) = A$ f\"ur alle $\alpha \in \Aut G$. Daher gilt auch $\alpha\l(A\r) = A$ f\"ur alle $\alpha \in \Inn G \Leftrightarrow A \nt G$.
  \item Sei $\alpha \in \Aut G$. Dann folgt: $\alpha\l(A \cap B\r)=\alpha\l(A\r) \cap \alpha\l(B\r)$ wegen der Bijektivit\"at und $\alpha\l(AB\r) = \alpha \l(A\r)\alpha\l(B\r)$, weil $\alpha$ ein Homomorphismus ist. Die Behauptung folgt, wenn wir einmal alle $\alpha \in \Aut G$ und einmal alle $\alpha \in \Inn G$ betrachten.
  \item Sei $\alpha \in \Aut G$ mit $\alpha\l(B\r)=B$. Dann ist $\alpha_{\mid B} \in \Aut B$ und folglich auch $\alpha_{\mid B}\l(A\r)=A$. Also ist $\alpha\l(A\r)=A$.
  \setcounter{enumi}{4}
  \item Die erste Aussage ist der Korrespondenzsatz \ref{korres}. Sei also nun $A \Char G$ und $B/A \Char G/A$. Sei $\Lambda$ ein Vertretersystem der Nebenklassen von $A$ in $B$. Dann ist $B = \bigcup_{\lambda \in \Lambda} \lambda A$. Sei $\alpha \in \Aut G$, dann:
  \begin{eqnarray*}
   \alpha\l(B\r) &=& \alpha\l(\bigcup_{\lambda\in\Lambda}\lambda A\r) = \bigcup_{\lambda\in\Lambda}\alpha\l(\lambda A\r)=\\&=&\bigcup_{\lambda\in\Lambda}\alpha\l(\lambda\r)\alpha\l(A\r)\stackrel{\ref{korres}}{=}\bigcup_{\lambda\in\Lambda}\alpha\l(b\r)A
  \end{eqnarray*}
  Weil $A \Char G$, wird durch $\bar{\alpha}: G/A \to G/A, gA \mapsto \alpha\l(g\r)A$ ein Automorphismus von $G/A$  gegeben. Ist $g_1A=g_2A$, so ist $g_1^{-1}g_2 \in A$ und daher $\alpha\l(g_1^{-1}g_2\r)=\alpha\l(g_1\r)^{-1}\alpha\l(g_2\r) \in A$ und folglich $\alpha\l(g_1\r)A=\alpha\l(g_2\r)A$, das hei\ss{}t $\bar{\alpha}$ ist wohldefiniert. Die Bijektivit\"at von $\bar{\alpha}$ folgt aus der Bijektivit\"at von $\bar{\alpha}$. Nach Voraussetzung gilt $\bar{\alpha}\l(B/A\r)=B/A$. Daher ist $\alpha\l(\lambda\r)A=\bar{\alpha}\l(\lambda A\r) \in B/A$ und demnach $\alpha\l(\lambda\r) \in B$. Also ist $\alpha\l(B\r) \leq B$ und daher $B \Char G$.
 \end{enumerate}
 \qed
\end{beweis}

\begin{folgerung}
 F\"ur alle $n \in \NN$ gilt: $G^{\l(n\r)} \Char G$.
\end{folgerung}

\begin{beweis}
 $G^{\l(n\r)} \Char G^{\l(n-1\r)}$, Induktion und \ref{aussagen_zu_charakteristischen_ugr} \ref{aussagen_zu_charakteristischen_ugr_3}.
\end{beweis}

\begin{satz}
 Sei $G$ eine Gruppe, die au\ss{}er $G$ und $\l<1\r>$ keine charakteristischen Untergruppen enth\"alt. Dann ist $G$ einfach oder ein direktes Produkt von isomorphen einfachen Gruppen.
\end{satz}

\begin{beweis}
 Sei $H > \l<1\r>$ ein minimaler Normalteiler von $G$. Wir schreiben $H_1 = H$ und betrachten alle Untergruppen von $G$ von der Form $H_1 \times \ldots \times H_n$ mit $H_i \cong H_i'$ und $H_i \nt G$. Sei $M$ eine solche Untergruppe mit maximaler Ordnung. Wir zeigen $H=G$, indem wir zeigen, dass $H \Char $ in $G$:

 Sei $\alpha \in \Aut G$. Es gen\"ugt zu zeigen, dass $\alpha\l(H_i\r) \subseteq M$ f\"ur alle $i$. Offensichtlich ist $\alpha\l(H_i\r) \cong H \cong H_1$. Weiter gilt $\alpha\l(H_i\r) \nt G$. Ist $a \in G$, so existiert ein $b \in G$ mit $\alpha\l(b\r)=a$. Damit ist $a^{-1}\alpha\l(H_i\r)a=\alpha\l(b\r)^{-1}\alpha\l(H_i\r)\alpha\l(b\r)=\alpha\l(b^{-1}H_ib\r)=\alpha\l(H_i\r)$. Ist $\alpha\l(H_i\r) \nleq M$, so ist $\l|\alpha\l(H_i\r) \cap M\r| < \l|H_i\r|$ und wegen $\alpha\l(H_i\r) \nt G, M \nt G$ ist $\alpha\l(H_i\r) \cap M \nt G$. Da $H$ minimaler Normalteiler war, folgt $\alpha\l(H_i\r) \cap M = \l<1\r>$. Damit ist $\l<\alpha\l(H_i\r), M\r> \cong M \times \alpha\l(H_i\r)$ ein direktes Produkt von derselben Form wie $M$, aber mit gr\"o\ss{}erer Ordnung. Widerspruch!

 Damit folgt $\alpha\l(H_i\r) \leq M$ und somit $M \Char G$, nach Voraussetzung also $M=G$. $H$ ist einfach: W\"are $N \nt H$ ein nichttrivialer Normalteiler, so w\"are auch $N \nt H_1 \times \ldots \times H_n = G$. Widerspruch zur Minimalit\"at von $H$.
 \qed
\end{beweis}

\section{Der Satz von Sylow}

\begin{satz}
 \label{ugr_ord_p}
 Sei $p$ prim und $G$ abelsche Gruppe mit $p \mid \l|G\r|$. Dann enth\"alt $G$ eine Untergruppe der Ordnung $p$.
\end{satz}

\begin{beweis}
 Nach Lemma \ref{untergruppenkriterium} gilt: $\l|AB\r| \mid \l|A\r| \l|B\r|$ f�r Untergruppen $A, B$ von $G$. Weil $G$ abelsch ist, ist $AB$ eine Untergruppe von $G$. Daher folgt durch Unduktion \"uber die Anzahl der Faktoren: Sind $A_1, \ldots, A_k$ Untergruppen von $G$, so gilt:
 $$
  \l|A_1\cdot \ldots \cdot A_k \r| \mid \l|A_1\r| \cdot \ldots \cdot \l|A_k\r|
 $$
 F\"ur jedes $g \in G$ ist $\l<g\r>$ eine zyklische Untergruppe von $G$ und das Produkt all dieser Untergruppen ist $G$. Also folgt:
 $$
  \l|G\r| \mid \prod_{g\in G} \ord g
 $$
 Daher gibt es ein $g \in G$ mit $p \mid \ord g$; setze $a := \frac{\ord g}{p}$. Dann it $\ord\l(g^a\r)=p$ und $\l<g^a\r>\leq G$ mit $\l<g^a\r>=p$.
 \qed
\end{beweis}

\begin{definition}[Sylowgruppe]
 \index{Sylowgruppe}
 \label{sylowgruppe}
 Sei $p$ prim und $\l|G\r|=p^am$ mit $p \nmid m$. Die Untergruppen $P \leq G$ mit $\l|P\r|=p^a$ hei\ss{}en \emph{$p$-Sylowgruppen von $G$}. Die Menge der $p$-Sylowgruppen wird mit $\Syl_pG$ bezeichnet.
\end{definition}

\begin{satz}[Satz von Cauchy]
 \index{Cauchy!Satz von}
 \label{satz_von_cauchy}
 Sei $p$ prim, $i \in \NN_0$, so dass $p^i \mid \l|G\r|$. Dann enth\"alt $G$ eine Untergruppe der Ordnung $p^i$.
\end{satz}

\begin{beweis}
 Beweis durch Induktion \"uber die Gruppenordnung $\l|G\r|$. Sei die Behauptung schon f\"ur alle Gruppen kleinerer Ordnung bewiesen. Besitzt $G$ eine Untergruppe $V < G$ mit $p^i \mid \l|V\r|$, so besitzt $V$ eine Untergruppe der Ordnung $p^i$ und damit auch $G$. Wir k\"onnen dann annehmen, dass $G$ keine solche Untergruppe $V$ besitzt und betrachten die Klassengleichung \ref{klassengleichung}
 $$
  \l|G\r| = \sum_{j=1}^r \frac{\l|G\r|}{\l|\Cen_G\l(x_j\r)\r|} + \l|\Zen\l(G\r)\r|
 $$
 Die linke Seite $\l|G\r|$ ist durch $p^i$ teilbar. Wegen $p^i \mid \l|\Cen_G\l(x_j\r)\r|$ sind alle Summanden $\frac{\l|G\r|}{\l|\Cen_G\l(x_j\r)\r|}$ durch $p$ teilbar. Damit folgt $p \mid \l|\Zen\l(G\r)\r|$. Mit \ref{ugr_ord_p} folgt: Weil $\Zen\l(G\r)$ abelsch, enth\"alt $\Zen\l(G\r)$ eine Untergruppe $N$ der Ordnung $p$. Nach \ref{aussagen_zu_ugr}.\ref{ugr_zen_nt} gilt: $N \nt G$. Die Ordnung der Faktorgruppe $G/N$ ist kleiner als $\l|G\r|$, also gibt es nach Induktionsvoraussetzung eine Untergruppe $B/N \leq G/N$ mit $\l|B/N\r| = p^{i-1}$. Nach dem Korrespondenzsatz \ref{korres} gibt es eine Untergruppe $B$: $N \leq B \leq G$, so dass $\nu\l(B\r) = B/N$ ($\nu: G \to G/N$ kanonischer Epimorphismus). Es gilt:
 $$
  \l|B\r| = \l|B/N\r| \l|N\r| = p^i
 $$
 \qed
\end{beweis}

\begin{folgerung}
 Sei $M \leq G$ maximale Untergruppe. Gilt $M \nt G$, so ist $\l[G:M\r]$ eine Primzahl.
\end{folgerung}

\begin{beweis}
 Weil $M$ maximal ist, gilt $M < G$ und es gibt kein $U \leq G: M \leq U \leq G$. Nach dem Korrespondenzsatz \ref{korres} besitzt daher die Faktorgruppe $G/M$ keine nichttriviale Untergruppe. Da $\l|G/M\r| > 1$ existiert eine Primzahl $p$ mit $p \nmid \l|G/M\r|$. Nach dem Satz von Cauchy \ref{satz_von_cauchy} folgt: $G/M$ besitzt eine Untergruppe der Ordnung $p$. Daher muss $p=\l|G/M\r|$ gelten.
 \qed
\end{beweis}

\begin{bemerkung}\spspace
 \label{bem3_5}
 \begin{enumerate}
  \item \label{bem3_5_1} Ist $P \in \Syl_pG$ und $P \leq G$ eine $p$-Untergruppe mit $BP = PB$, dann gilt: $B \leq P$. Insbesondere ist $P$ die einzige $p$-Sylowgruppe zu $\Norm_G\l(P\r)$.
  \item Ist $\alpha \in \Aut\l(G\r)$ und $P \in \Syl_pG$, so ist auch $\alpha\l(P\r) \in \Syl_pG$.
  \item F\"ur $P \in \Syl_pG$ gilt: $P \nt G \Leftrightarrow \Syl_pG=\l\{P\r\}P \Leftrightarrow \Char G$.
  \item $\OOC_p\l(G\r) := \bigcap_{P\in\Syl_pG}P \Char G$ und $\OOC_p\l(G\r)$ enth\"alt alle $p$-Normalteiler von $G$, das hei\ss{}t alle Normalteiler $N$ haben die Ordnung $p^k$.
 \end{enumerate}
\end{bemerkung}

\begin{beweis}\spspace
 \begin{enumerate}
  \item Nach Lemma \ref{untergruppenkriterium} gilt: $BP$ ist eine Untergruppe von $G$, und zwar eine $p$-Gruppe. Da offensichtlich $P \leq BP$ und $P \in \Syl_pG$ folgt: $P = BP$ und damit $B \leq P$. In $\Norm_G\l(P\r)$ gilt:
  $$
   xP = Px \qquad \forall x \in \Norm_G\l(P\r)
  $$
  Ist also $B \leq \Norm_G\l(P\r)$ eine $p$-Untergruppe, so folgt $BP=PB$.
  \item Ist $\alpha \in \Aut\l(G\r)$, so gilt $\l|P\r| = \l|\alpha\l(P\r)\r|$, also ist auch $\alpha\l(P\r)$ eine $p$-Sylowgruppe.
  \item Sei $Q \leq G$ eine weitere $p$-Sylow. Wegen $P\nt G$ ist dann $PQ=QP$ und mit 1. folgt: $Q=P$.

  F\"ur $\alpha \in \Aut\l(G\r)$ ist $\alpha\l(P\r)$ eine $p$-Sylow (nach 2.) und daher $\alpha\l(P\r)=P$.

  Der Rest folgt aus \ref{aussagen_zu_charakteristischen_ugr}.\ref{aussagen_zu_charakteristischen_ugr_3}
  \item F\"ur $\alpha \in \Aut\l(G\r)$ gilt: Jede $p$-Sylow wird wieder auf eine $p$-Sylow abgebildet. Also ist $\alpha\l(\OOC_p\l(G\r)\r)=\OOC_p\l(G\r)$. Ist $B \nt G p$-Normalteiler und $P\in \Syl_pG$, so gilt: $BP=PB$ nach dem 1. Isomorphiesatz \ref{isomorphiesatz1}. NAch 1. folgt $B \leq P$. Folglich ist $B$ in allen $p$-Sylowgruppen enthalten und damit $B\leq \OOC_p\l(G\r)$.
 \end{enumerate}
 \qed
\end{beweis}

\begin{definition}[$\OOC_{p'}\l(G\r), \OOC\l(G\r)$]
 \index{$\OOC_{p'}\l(G\r)$}
 \index{$\OOC\l(G\r)$}
 Der gr\"o\ss{}te Normalteiler $N \nt G$ mit $p \nmid \l|N\r|$ wird mit $\OOC_{p'}\l(G\r)$ bezeichnet. Statt $\OOC_{2'}\l(G\r)$ schreibt man oft kurz $\OOC\l(G\r)$.
\end{definition}

\begin{bemerkung}\spspace
 \begin{enumerate}
  \item $\OOC_{p'}\l(G\r)$ enth\"alt alle $p'$-Normalteiler von $G$, das hei\ss{}t alle $N \nt G$ mit $p \nmid \l|N\r|$.
  \item $\OOC_{p'}\l(G\r) \Char G$.
  \item $\OOC_p\l(G/\OOC_p\l(G\r)\r)=\l<1\r>, \OOC_{p'}\l(G/\OOC_{p'}\l(G\r)\r)=\l<1\r>$
 \end{enumerate}
\end{bemerkung}

\begin{beweis}\spspace
 \begin{enumerate}
  \item  Seien $N, U \nt G$ mit $p \nmid \l|N\r|, \l|U\r|$. F\"ur $g\in G$ ist
  $$
   g^{-1}NUg = g^{-1}Ngg^{-1}Ug = NU
  $$
  Also ist $NU \nt G$ und nach \ref{untergruppenkriterium} gilt: $\l|NU\r| \mid \l|N\r|\l|U\r|$, das hei\ss{}t es gilt $p \nmid \l|NU\r|$.

  Folglich ist f\"ur jeden $p'$-Normalteiler $N$ die Gruppe $N\OOC_{p'}\l(G\r) \nt G$ mit $p \nmid \l|N\OOC_{p'}\l(G\r)\r|$ und $\OOC_{p'}\l(G\r) \leq N\OOC_{p'}\l(G\r)$. Nach Definition folgt $N\OOC_{p'}\l(G\r) = \OOC_{p'}\l(G\r)$ und damit $N\leq\OOC_{p'}\l(G\r)$.
  \item Ist $\alpha \in \Aut\l(G\r)$, so gilt $\alpha\l(\OOC_{p'}\l(G\r)\r) \nt G$ und
  $$
   \l|\alpha\l(\OOC_{p'}\l(G\r)\r)\r| = \l|\OOC_{p'}\l(G\r)\r|
  $$
  Nach 1. folgt
  $$
   \alpha\l(\OOC_{p'}\l(G\r)\r) = \OOC_{p'}\l(G\r)
  $$
  \item Enth\"alt $G/\OOC_p\l(G\r)$ einen nichttrivialien $p$-Normalteiler $N/\OOC_p\l(G\r)$, so gibt es nach dem Korrespondenzsatz \ref{korres} ein $N \nt G$ mit $\OOC_p\l(G\r) \leq N \leq G$ und $\nu\l(N\r) = N/\OOC_p\l(G\r)$ ($\nu$ kanonischer Epimorphismus). Es gilt:
  $$
   \l|N\r| = \l|N/\OOC_p\l(G\r)\r| \l|\OOC_p\l(G\r)\r|
  $$
  ist $p$-Potenz. Widerspruch!

  F\"ur $\OOC_{p'}\l(G\r)$ analog.
 \end{enumerate}
 \qed
\end{beweis}

\begin{lemma}
 \label{anzahl_h_konjugierte}
 Seien $P, H \leq G$. Dann gibt es $\frac{\l|H\r|}{\l|\Norm_H\l(P\r)\r|}$ verschiedene $H$-Konjugierte von $P$ in $G$.
\end{lemma}

\begin{beweis}
 Analog zu \ref{anzahl_elemente_konjugationsklasse}.
 F\"ur $a,b\in H$ gilt:
 $$
  a^{-1}Pa=b^{-1}Pb \Leftrightarrow ba^{-1}Pab^{-1}=P \Leftrightarrow ab^{-1} \in \Norm_H\l(P\r) \Leftrightarrow a \in \Norm_H\l(P\r)b
 $$
 Das hei\ss{}t $a,b$ liegen in derselben Rechtsnebenklasse von $\Norm_H\l(P\r)$.
 \qed
\end{beweis}

\begin{lemma}
 \label{lemma_3_9}
 Sei $H$ eine $p$-Untergruppe von $G$ und $P \in \Syl_pG$. Dann gilt:
 $$
  H \cap \Norm_G\l(P\r) = H \cap P
 $$
\end{lemma}

\begin{beweis}
 $H \cap \Norm_G\l(P\r) \supset H \cap P$ ist klar wegen $P \leq \Norm_G\l(P\r)$. Sei daher nun $H \cap \Norm_G\l(P\r) \subset H \cap P$. $H \cap \Norm_G\l(P\r)$ ist eine $p$-Untergruppe von $\Norm_G\l(P\r)$. Wegen $P \nt \Norm_G\l(P\r)$ gilt:
 $$
  \l(H\cap\Norm_G\l(P\r)\r)P = P\l(H\cap\Norm_G\l(P\r)\r)
 $$
 Nach \ref{bem3_5}.\ref{bem3_5_1} folgt: $$H \cap \Norm_G\l(P\r) \leq P$$ und damit $$H\cap \Norm_G\l(P\r) \leq H \cap P$$
 \qed
\end{beweis}

Damit k\"onnen wir den wichtigsten Satz der Gruppentheorie beweisen:

\begin{satz}[Satz von Sylow]
 \index{Sylow!Satz von}
 \label{satz_von_sylow}
 Sei $p$ prim und $\l|G\r|=p^am$ mit $p \nmid m$. Es gilt:
 \begin{enumerate}
  \item Jede $p$-Untergruppe von $G$ ist in einer $p$-Sylowgruppe von $G$ enthalten.
  \item Alle $p$-Sylowgruppen von $G$ sind konjugiert und f\"ur $P\in \Syl_pG$ gilt:
   $$
    \l|\Syl_pG\r| = \frac{\l|G\r|}{\l|\Norm_G\l(P\r)\r|}
   $$
  \item Es gilt: 
   $$
    \l|\Syl_pG\r| \mid \l|G\r| \quad \mbox{und} \quad \l|\Syl_pG\r| \equiv 1 \mod p
   $$
   Genauer: $\l|\Syl_pG\r| \mid m$. Insbesondere gibt es eine $p$-Sylowgruppe in $G$.
 \end{enumerate}
\end{satz}

\begin{beweis}\spspace
 \begin{enumerate}
  \item Beweisen wir zusammen mit
  \item Sei $P \in \Syl_pG$. Wir betrachten die Menge $M := \l\{x^{-1}Px \mid x \in G\r\}$ der Konjugationen von $P$. Nach \ref{anzahl_h_konjugierte} gilt: $\l|M\r| = \frac{\l|G\r|}{\l|\Norm_G\l(P\r)\r|}$. Weil $P \leq \Norm_G\l(P\r)$ teilt $p$ nicht $\frac{\l|G\r|}{\l|\Norm_G\l(P\r)\r|}$. Sei $H\leq G$ eine $p$-Untergruppe. Wir betrachten die $H$-Konjugationen auf $M$. Sei $\l\{x_1^{-1}Px_1, \ldots, x_r^{-1}Px_r\r\}$ eine Transversale der $H$-Konjugationsklassen. Die $H$-Konjugationsklasse, die $x^{-1}Px$ enth\"alt, hat nach \ref{anzahl_h_konjugierte} genau $\frac{\l|H\r|}{\l|\Norm_H\l(x^{-1}Px\r)\r|}$ Elemente. Insbesondere ist die Zahl dieser Elemente ein Teiler von $\l|H\r|$, also eine $p$-Potenz. $M$ ist die disjunkte Vereinigung der $H$-Konjugationsklassen, also folt:
  $$
   \frac{\l|G\r|}{\l|\Norm_G\l(P\r)\r|} = \l|M\r| = \sum_{i=1}^r\frac{\l|H\r|}{\l|\Norm_H\l(x_i^{-1}Px_i\r)\r|}
  $$
  $p$ teilt nicht $\frac{\l|G\r|}{\l|\Norm_G\l(P\r)\r|}$ und
  $$
   p \nmid \frac{\l|H\r|}{\l|\Norm_H\l(x_i^{-1}Px_i\r)\r|} \Leftrightarrow H = \Norm_H\l(x_i^{-1}Px_i\r)
  $$
  Weil die linke Seite nicht durch $p$ teilbar ist, muss es ein $i \in \l\{1,\ldots,r\r\}$ geben, so dass $H=\Norm_H\l(x_i^{-1}Px_i\r)$. Damit ist $H\leq \Norm_G\l(x_i^{-1}Px_i\r)$ und folglich
  $$
   H\l(x_i^{-1}Px_i\r) = \l(x_i^{-1}Px_i\r)H
  $$
  Mit \ref{bem3_5}.\ref{bem3_5_1} folgt $H\leq x_i^{-1}Px_i$. Ist $H$ eine beliebige $p$-Untergruppe von $G$, so ist $H$ demnach in einer $p$-Sylowgruppe enthalten. Ist $H$ eine $p$-Sylowgruppe, so ist $H = x_i^{-1}Px_i$, das hei\ss{}t $H$ ist konjugiert zu $P$. Damit ist auch $\Syl_pG = M$ und daher
  $$
   \l|\Syl_pG\r| = \frac{\l|G\r|}{\l|\Norm_G\l(P\r)\r|}
  $$
  \item Setzen wir im ersten Teil $H=P$, so ergibt sich
  \begin{eqnarray*}
   \l|\Syl_pG\r| & = & \l|M\r| = \sum_{i=1}^r\frac{\l|P\r|}{\l|\Norm_P\l(x_i^{-1}Px_i\r)\r|}=\\&\stackrel{\ref{satz_uber_cen_und_norm}}{=}&\sum_{i=1}^r\frac{\l|P\r|}{\l|P\cap\Norm_G\l(x_i^{-1}Px_i\r)\r|}\stackrel{\ref{lemma_3_9}}{=}\sum_{i=1}^r\frac{\l|P\r|}{\l|P\cap\l(x_i^{-1}Px_i\r)\r|}
  \end{eqnarray*}
  Bei den Summanden $\l|P\cap\l(x_i^{-1}Px_i\r)\r|$ gibt es genau einen Summanden 1 (n\"amlich f\"ur $x_i^{-1}Px_i=P$), sonst ist
  $$
   P \cap x_i^{-1}Px_i < P
  $$
  und daher $p \mid \frac{\l|P\r|}{\l|P\cap\l(x_i^{-1}Px_i\r)\r|}$. Insgesamt folgt
  $$
   \l|\Syl_pG\r| \equiv 1 \mod p
  $$
  Dass $\l|\Syl_pG\r| \mid \l|G\r|$ (genauer: $\l|\Syl_pG\r| \mid m$) haben wir bereits im 1. Teil gezeigt.
 \end{enumerate}
 \qed
\end{beweis}

\begin{bemerkung*}
 Der Satz von Sylow ist der vielleicht wichtigste Satz der Gruppentheorie, weil er einen ersten Ansatzpunkt liefert, die Struktur der Gruppe n\"aher zu untersuchen. In einigen Spezialf\"allen gen\"ugt er (mit einigen kleinen zus\"atzlichen Resultaten) sogar schon, um alle Gruppen mit einer bestimmten Ordnung zu klassifizieren. Naturgem\"a\ss{} ist der Satz jedoch von Sylow keine Hilfe bei der Untersuchung von $p$-Gruppen.

 Die Bedeutung des Satzes legt die Frage nahe, ob er sich verallgemeinern l\"asst. Tats\"achlich gibt es zwei Resultate, die dies in bestimmte Richtungen tun:
 \begin{itemize}
  \item Sei $\l|G\r| = p^am$ mit $p \nmid m$ und sei $s \in \NN_0$ mit $0 \leq s \leq a$. Sei $r_s$ die Anzahl der Untergruppen von $G$ mit Ordnung $p^s$. Dann gilt:
  $$
   r_s \equiv 1 \mod p
  $$
  \item Satz von Hall: Sei $G$ \emph{aufl\"osbar} und $\l|G\r| = ab$ mit $\ggT\l(a,b\r) = 1$. Dann enth\"alt $G$ Untergruppen der Ordnung $a$ und alle diese Untergruppen sind konjugiert.
 \end{itemize}
\end{bemerkung*}

\begin{folgerung}
 Sei $P \in \Syl_pG, \Norm_G\l(P\r) \leq U \leq G$. Dann gilt
 \begin{enumerate}
  \item $\Norm_G\l(U\r)=U$
  \item $\l[G:U\r] = 1 \mod p$
 \end{enumerate}
\end{folgerung}

\begin{beweis}\spspace
 \begin{enumerate}
  \item Sei $x\in \Norm_G\l(U\r)$, das hei\ss{}t $x^{-1}Ux=U$. Dann gilt:
  $$
   x^{-1}Px \leq x^{-1}\Norm_G\l(Pr\r)x \leq x^{-1}Ux \leq U
  $$
  Also ist auch $x^{-1}Px$ eine $p$-Sylowgruppe und $U$, das hei\ss{}t $P$ und $x^{-1}Px$ sind in $U$ konjugiert (als $p$-Sylowgruppe von $U$). Daher gibt es ein $u\in U$, so dass
  $$
   u^{-1}\l(x^{-1}Px\r)u = P
  $$
  Folglich ist $xu \in \Norm_G\l(P\r) \leq U \Rightarrow x \in U$.
  \item $P$ ist eine $p$-Sylowgruppe von $U$. Wegen $\Norm_G\l(P\r) \leq U$ ist $\Norm_U\l(P\r) = \Norm_G\l(P\r)$. Mit dem Satz von Sylow \ref{satz_von_sylow} folgt
  $$
   \l[U:\Norm_G\l(P\r)\r] \equiv 1 \mod p
  $$
  Ebenfalls nach dem Satz von Sylow \ref{satz_von_sylow} ist aber auch
  $$
   \l[G:U\r] \l[U:\Norm_G\l(P\r)\r] = \l[G : \Norm_G\l(P\r)\r] \equiv 1 \mod p
  $$
  Deshalb ist $\l[G:U\r] \equiv 1 \mod p$
 \end{enumerate}
 \qed
\end{beweis}

\begin{folgerung}
 Sei $N \nt G$ und $P \in \Syl_pG$. Dann ist $P\cap N \in Syl_pN$ und $PN/N \in \Syl_p\l(G/N\r)$. Beachte: Ist nur $N \leq G$, so gilt die erste Aussage im Allgemeinen \emph{nicht}.
\end{folgerung}

\begin{beweis}
 Sei $R$ eine $p$-Sylowgruppe von $N$. Nach dem Satz von Sylow \ref{satz_von_sylow} ist $R$ in einer $p$-Sylowgruppe $P_1$ von G enthalten und es gibt ein $x \in G$ mit $x^{-1}P_1x=P$. Da $R$ die maximale $p$-Gruppe von $N$, so ist
 $
  P_1\cap N = R\cap N = R
 $.
 Nun ist
 $$
  P\cap N = x^{-1}P_1x \cap N \stackrel{N\nt G}{=} x^{-1}P_1x \cap x^{-1}Nx=x^{-1}P_1\cap Nx=x^{-1}Rx 
 $$
 Daraus folgt:
 $$
  \l|P_1 \cap N\r| = \l|x^{-1} R x\r| = \l|R\r| \Rightarrow P\cap N\in \Syl_pN
 $$

 Sei $ \overline{U} := PN/N \leq G/N =: \overline{G} $ ($N \nt PN$ nach dem 1. Isomorphiesatz \ref{isomorphiesatz1}). Sei $\l|G\r| = p^ag, \l|N\r|=p^bn, p \nmid g,n$. Nach der ersten Aussage des Satzes ist dann $P \cap N$ eine $p$-Sylowgruppe von $N$, also gilt $\l|P\cap N\r|=p^b$. Mit \ref{untergruppenkriterium} folgt:
 $$
  \l|PN\r| = \frac{\l|P\r|\l|N\r|}{\l|P\cap N\r|} = \frac{p^ap^bn}{p^b} = p^an
 $$
 und damit
 $$
  \l|\overline{U}\r| = \l|PN/N\r| = \frac{p^an}{p^bn} = p^{a-b}
 $$
 Wegen $\l|G/N\r| = \frac{p^ag}{p^bn}=p^{a-b}\frac{g}{n}$ ist $\overline{U}$ eine $p$-Sylowgruppe von $\overline{G}$.
 \qed
\end{beweis}

\begin{bemerkung}
 Sei $\l|G\r| = p_1^{a_1} \cdot \ldots \cdot p_k^{a_k}$ die Primfaktorzerlegung der Ordnung von $G, p_i \in \Syl_{p_i}G$. Dann ist $G = \l< p_1, \ldots, p_k \r>$.
\end{bemerkung}

\begin{beweis}
 Sei $G_1 := \l<p_1, \ldots, p_k\r> \Rightarrow P_i \leq G_1; \l|G_1\r|=\l|G\r|$. Daraus folgt Behauptung.
 \qed
\end{beweis}

\begin{satz}[Zerlegung von Gruppenelementen]
 \label{zerlegung_von_gruppenelementen}
 Sei $g \in G$ mit $\ord G = p_1^{a_1}\cdot \ldots \cdot p_k^{a_k}$. Dann gibt es eindeutig bestimmte Elemente $g_1, \ldots g_k$ mit $\ord g_i = p_i^{a_i}$, so dass $g=g_1\cdot\ldots\cdot g_k$ und $g_ig_j=g_jg_i$. Die $g_i$ sind Potenzen von $g$, das hei\ss{}t $g_i \in \l<g\r>$.

 \emph{Beachte}: Die Zahl $g=g_1\cdot\ldots\cdot g_k$ mit $\ord g_i = p_i^{a_i}$ ist nicht eindeutig.
\end{satz}

\begin{beweis}
 Sei $p := p_1^{a_1}$ und $q = \prod_{i=2}^kp_i^{a_i}$. Wegen $\ggT\l(p,q\r)=1$ gibt es $x,y\in\ZZ$ mit $xp+yq=1$. Wir setzen $g_1 := g^{yq}, g_2' := g^{xp}$. Dann ist
 \begin{eqnarray*}
  g_1^p&=&\l(g^{yq}\r)^p = \l(g^{pq}\r)^{y'}=1^y=1\\
  &\Rightarrow& \ord g_1 = p = p_1^{a_1}\\
  \l(g_2'\r)^q&=&\l(g^{xp}\r)^q=\l(g^{pq}\r)^x=1^x=1\\
  &\Rightarrow& \ord g_2'=q
 \end{eqnarray*}
 Induktiv folgt die Behauptung. Eindeutigkeit:
 $$
  g = \underbrace{g_1g_2}_{p_1} = \underbrace{g_3g_4}_{p2}
 $$
 \qed
\end{beweis}

\begin{definition}[Exponent]
 \index{Exponent}
 \label{exponent}
 Der \emph{Exponent} $\exp G$ einer Gruppe $G$ ist die kleinste Zahl $m \in \NN$ mit $g^m \equiv 1$ f\"ur alle $g \in G$. Mit anderen Worten:
 $$
  \exp G = \kgV_{g\in G} \l(\ord g\r)
 $$
\end{definition}

\begin{folgerung}
 Sei $\l|G\r| = p_1^{a_1}\cdot \ldots \cdot p_n^{a_n}$ und sei $P_i \in \Syl_pG$. Dann gilt:
 $$
  \exp G = \prod_{i=1}^n \exp P_i
 $$
\end{folgerung}

\begin{beweis}
 Dass $\prod_{i=1}^n \exp P_i \mid \exp G$ ist klar, denn die $\exp P_i$ sind paarweise teilerfremd und f\"ur $U \leq G$ gilt $\exp U \mid \exp G$. Sei $g \in G$ mit $\ord g = p_1^{b_1}\cdot\ldots\cdot p_n^{b_n}$. Nach \ref{zerlegung_von_gruppenelementen} gibt es ein $g \in G$ mit $\ord g_i = p_i^{b_i}$ so, dass $G = g_1 \cdot\ldots\cdot g_n$. Nach dem Satz von Sylow \ref{satz_von_sylow} ist $g_i$ in einer $p$-Sylowgruppe enthalten und daher $p_i^b=\ord g_i \mid \exp P_i$. Damit teilt $\ord g$ das Produkt $\prod_{i=1}^n \exp P_i$ und es gilt
 $$
  \exp G \mid \exp P_i
 $$
 \qed
\end{beweis}

\begin{bemerkung*}[Alperin und Kuo]
 F\"ur jede Gruppe $G$ gilt:
 $$
  \exp G \mid \l[G : G' \cap \Zen\l(G\r) \r]
 $$
\end{bemerkung*}

\section{Automorphismengruppen zyklischer Gruppen}

\begin{definition}[Eulersche $\varphi$-Funktion]\spspace
 \index{Eulersche $\varphi$-Funktion}
 \index{$\ZZ_n^*$}
 \label{eulersche_phi_fkt}
 \begin{enumerate}
  \item $\varphi \l(u\r) := \#\l\{ m \in \NN \mid \l(n,m\r)=1 \r\}$ hei\ss{}t die \emph{Eulersche $\varphi$-Funktion}
  \item $\ZZ_n^*, \l|\ZZ_n^*\r| = \varphi \l(n\r)$, da $m \in \ZZ_n^* \Leftrightarrow \l(m,n\r)=1$
 \end{enumerate}
\end{definition}

\begin{bemerkung*}
 Der Satz von Lagrange liefert sofort den Satz von Euler-Fermat: F\"ur $\ggT\l(x,n\r) = 1$ gilt:
 $$
  x^{\varphi\l(n\r)} \equiv 1 \mod n
 $$
\end{bemerkung*}

\begin{satz}
 \label{aut_isom_znstern}
 F\"ur $n \in \ZZ$ gilt:
 $$
  \Aut\l(\ZZ_n\r) \cong \ZZ_n^*
 $$
\end{satz}

\begin{beweis}
 Sei $\ZZ_n = \l<g\r>$. Jedes $\alpha \in \Aut G$ ist bereits eindeutig festgelegt durch $\alpha\l(g\r)$. Offensichtlich ist $\alpha\l(g\r) = g^k$ f\"ur ein $k \in \NN$ mit $1 \leq k \leq n$. Weil $\alpha$ ein Automorphismus ist, gilt:
 $$
  n = \ord g = \ord \alpha\l(g\r) = \ord g^k = \frac{n}{\ggT\l(n,k\r)}
 $$
 Daher ist $\ggT\l(n,k\r)=1$ und folglich $\overline{k} \in \ZZ_n^*$. F\"ur $\overline{k} \in \ZZ_n^*$ ist
 $$
  \alpha_k: \ZZ_n \to \ZZ_n, x \mapsto x^n
 $$
 ein Automorphismus. Wir setzen:
 $$
  \psi: \Aut\l(\ZZ_n\r) \to \ZZ_n^*, \alpha_k \mapsto \overline{k}
 $$
 F\"ur $k, i \in \ZZ_n^*$ ist
 $$
  \l(\alpha_k \circ \alpha_i\r)\l(g\r) = \alpha_k\l(g'\r) = g^{ik} = g^{ki} = \alpha_{ki}\l(g\r)
 $$
 Und damit ist
 $$
  \psi\l(\alpha_k \circ \alpha_i\r)\l(g\r) = \psi\l(\alpha_{ki}\r) = \overline{k_i} = \overline{ki} = \overline{k}\overline{i}=\psi\l(\alpha_k\r)\psi\l(\alpha_i\r)
 $$
 Das hei\ss{}t: $\psi$ ist ein Homomorphismus. Die Bijektivit\"at ist klar.
 \qed
\end{beweis}

\begin{satz}\spspace
 \label{aut_von_kreuzprod}
 \begin{enumerate}
  \item $\Aut G \times \Aut H$ ist isomorph zu einer Untergruppe von $\Aut\l(G\times H\r)$
  \item Falls $\ggT\l(\l|G\r|, \l|H\r|\r)=1$, so gilt:
   $$
    \Aut\l(G\times H\r) \cong \Aut\l(G\r) \times \Aut\l(H\r)
   $$
 \end{enumerate}
\end{satz}

\begin{beweis}\spspace
 \begin{enumerate}
  \item Sei $\varphi \in \Aut G, \psi \in \Aut H$. Wir setzen
  $$
   \varphi \times \psi: G \times H \to G\times H, \l(g,h\r) \mapsto \l(\varphi\l(g\r), \psi\l(h\r)\r)
  $$
  Dann ist $\varphi \times \psi$ Automorphismus von $G \times H$ (leicht nachzurechnen). Sei
  $$
   \Phi: \Aut G \times \Aut H \to \Aut \l( G\times H\r), \l(\varphi, \psi\r) \mapsto \varphi \times \psi
  $$
  ein injektiver Gruppenhomomorphismus:

  F\"ur $\alpha, \varphi \in \Aut G, \beta, \psi \in \Aut H$ gilt mit $g \in G, h \in H$:
  $$
   \l(g^{\alpha\varphi},h^{\beta\psi}\r)=\l(g^\alpha,h^\beta\r)^{\varphi\times\psi}=\l(g,h\r)^{\l(\alpha\times\beta\r)\l(\varphi\times\psi\r)}
  $$
  Also ist
  $$
   \Phi\l(\l(\alpha, \beta\r)\cdot\l(\varphi,\psi\r)\r)=\Phi\l(\alpha\varphi,\beta\psi\r)=\l(\alpha\times\beta\r)\l(\varphi\times\psi\r)=\Phi\l(\alpha,\beta\r)\Phi\l(\varphi,\psi\r)
  $$
  Das hei\ss{}t $\Phi$ ist ein Homomorphismus. $\Phi$ ist injektiv, denn f\"ur $\l(\alpha,\beta\r)\neq\l(\varphi,\psi\r)$ ist $\alpha\times\beta \neq \varphi \times \psi$. Nach dem Homomorphiesatz \ref{homomorphiesatz} ist dann
  $$
   \Aut G\times\Aut H \cong \Phi\l(\Aut G\times\Aut H\r) \leq \Aut\l(G\times H\r)
  $$
  \item F\"ur $\ggT\l(\l|G\r|,\l|H\r|\r)=1$ ist $\ord\l(g,h\r)=\ord\l(g\r)\ord\l(h\r)$. Bei jedem Automorphismus $\varphi \in \Aut\l(G\times H\r)$ gilt $\ord\varphi\l(g,h\r)=\ord\l(g,h\r)$. Daher ist
  $$
   \varphi\l(G\times\l<1\r>\r) = G\times \l<1\r> \qquad \mbox{und} \qquad
   \varphi\l(\l<1\r>\times H\r) = \l<1\r>\times H
  $$
  Seien
  $$
   \pi_1 : G\times H \to G \qquad \mbox{und} \qquad
   \pi_2 : G\times H \to H
  $$
  Projektionen. Wir definieren
  \begin{eqnarray*}
   \varphi_G: G \to G&,& \varphi_G\l(g\r)=\l(\pi_1\circ\varphi\r)\l(g,1\r)\\
   \varphi_H: H \to H&,& \varphi_H\l(h\r)=\l(\pi_2\circ\varphi\r)\l(h,1\r)
  \end{eqnarray*}
  Dann gilt: $\varphi_G\leq \Aut G, \varphi_H\in\Aut H$. Die Abbildung
  $$
   \Psi: \Aut\l(G\times H\r) \to \Aut G\times \Aut H, \varphi \mapsto \l(\varphi_G,\varphi_H\r)
  $$
  ist injektiv: Sind $\varphi, \tau \in \Aut\l(G\times H\r)$ mit $\varphi \neq \tau$, so gibt es $g \in G, h \in H$ mit $\varphi\l(g,h\r)\neq\tau\l(g,h\r)$, also ist $\varphi_G\l(g\r)\neq\tau_G\l(g\r)$ oder $\varphi_H\l(h\r)\neq\tau_H\l(h\r)$ und daher $\l(\varphi_G,\varphi_H\r)\neq\l(\tau_G,\tau_H\r)$. Damit ist
  $$
   \l|\Aut\l(G\times H\r)\r|\leq \l|\Aut G \times \Aut H\r|
  $$
  Mit 1. folgt die Behauptung.
 \end{enumerate}
 \qed
\end{beweis}

\begin{folgerung}
 Sei $n=p_1^{a_1}\cdot\ldots\cdot p_n^{a_n}$ die Primfaktorzerlegung. Dann ist auch
 $$
  \Aut\ZZ_n \cong \ZZ_n^* \cong \ZZ_{p_1^{a_1}}^*\times\ldots\times\ZZ_{p_n^{a_n}}^*
 $$
\end{folgerung}

\begin{beweis}
 Nach dem Hauptsatz \"uber endliche abelsche Gruppen \ref{hauptsatz_ueber_endliche_abelsche_gruppen} ist $\ZZ_n \cong \ZZ_{p_1^{a_1}}\times\ldots\times\ZZ_{p_n^{a_n}}$. Mit \ref{aut_isom_znstern} und \ref{aut_von_kreuzprod} folgt die Behauptung.
 \qed
\end{beweis}

\begin{folgerung}
 \label{hom_euler_fkt}
 F\"ur $n,m\in \NN, \ggT\l(n,m\r)=1$ gilt:
 $$
  \varphi\l(n,m\r) = \varphi\l(n\r) \varphi\l(m\r)
 $$
\end{folgerung}

\begin{folgerung}
 Sei $n=p_1^{a_1}\cdot\ldots\cdot p_n^{a_n}$ die Primfaktorzerlegung. Dann gilt:
 $$
  \varphi\l(n\r) = n\l(1-\frac{1}{p_1}\r)\l(1-\frac{1}{p_2}\r)\ldots\l(1-\frac{1}{p_k}\r)
 $$
\end{folgerung}

\begin{beweis}
 Ist $p$ prim, so gibt es genau $p^a - p^{a-1}$ Zahlen $m$ mit $1\leq m \leq p^a$, die zu $p^a$ teilerfremd sind. Daher ist
 $$
  \varphi\l(p^a\r)=p^a-p^{a-1}=p^a\l(1-\frac{1}{p}\r)
 $$
 Der Rest folgt mit \ref{hom_euler_fkt}.
\end{beweis}

Nun wollen wir die Gruppen $\ZZ_{p^k}^*$ f\"ur $p$ prim n\"aher untersuchen. F\"ur $\ZZ_p^*$ m\"ussen wir eine Aussage aus der K\"orpertheorie verwenden:

\begin{satz}
 Sei $p$ prim. Dann ist $\ZZ_p^* \cong \ZZ_{p-1}$ zyklisch.
\end{satz}

\begin{beweis}
 $\ZZ_p^*$ ist als multiplikative Gruppe des K\"orpers $\l(\ZZ_p, +, \cdot\r)$ abelsch und l\"asst sich daher nach dem Hauptsatz \"uber endliche abelsche Gruppen \ref{hauptsatz_ueber_endliche_abelsche_gruppen} als direktes Produkt zyklischer Gruppen schreiben:
 $$
  \ZZ_p^* \cong Z_1\times\ldots\times Z_k
 $$
 Sind die Ordnungen $z_i := \l|Z_i\r|$ paarweise teilerfremd, so ist $\ZZ_p^*$ zyklisch. Sonst gibt es eine Primzahl $q$, die $z_i$ und $z_j$ f\"ur ein Paar $i,j$ mit $i\neq j$ teilt. Daher gibt es in $Z_1\times\ldots\times Z_k \cong \ZZ_p^*$ mindestens $q^2$ Elemente $x$ mit $x^q = 1$, das hei\ss{}t das Polynom $X^q-1 \in \ZZ_p\l[X\r]$ besitzt mindestens $q^2$ Nullstellen. \"Uber einem K\"orper besitzt ein Polynom vom Grad $m$ aber h\"ochstens $m$ Nullstellen $\Rightarrow$ Widerspruch! Also ist $\ZZ_p^*$ zyklisch und wegen $\l|\ZZ_p^*\r|=p-1$ daher $\ZZ_p^* \cong \ZZ_{p-1}$.
 \qed
\end{beweis}

\begin{definition}[Primitivwurzel modulo $n$]
 \index{Primitivwurzel}
 \label{primitivwurzel}
 Ein Erzeuger von $\ZZ_p^*$ hei�t \emph{Primitivwurzel modulo $n$}. F\"ur $p$ prim l\"asst sich eine Primitivwurzel modulo $p$ nur durch Raten finden. Es gibt daf\"ur weder ein theoretisches Resultat, noch einen guten Algorithmus.
\end{definition}

\begin{lemma}
 \label{menge_E_n}
 Sei $p$ prim und $E_n := \l\{\overline{x} \in \ZZ_{p^n} \mid x \equiv 1 \mod p\r\}$. Dann ist
 $$
  E_n \leq \ZZ_{p^n}^* \qquad \mbox{mit } \l|E_n\r| = p^{n-1}
 $$
\end{lemma}

\begin{beweis}
 Seien $\overline{x}, \overline{y} \in E_n$. Aus $x,y \equiv 1 \mod p$ folgt $xy\equiv 1 \mod p$, also ist $\overline{x}\cdot\overline{y}\in E_n$ und damit $E_n \leq \ZZ_{p^n}^*$. $\l|E_n\r|=p^{n-1}$ ist klar.
 \qed
\end{beweis}

\begin{lemma}
 \label{E_n_zyklisch}
 Sei $p>2$ prim. Dann ist $E_n$ zyklisch und f\"ur $n>1$ gilt:
 $$
  E_n = \l< x \r> \quad \Leftrightarrow \quad x \equiv 1 \mod p,\quad x \not\equiv 1 \mod p^2
 $$
\end{lemma}

\begin{beweis}
 Sei $y \equiv 0 \mod p$, aber $y \not\equiv 0 \mod p^2$. Wir zeigen induktiv f\"ur $0 \leq k \leq n-2$: $p^{k+1} \mid \l(1+x\r)^{p^k}-1$, aber $p^{k+2} \nmid \l(1+x\r)^{p^k}-1$. Nach Wahl von $y$ ist dies f\"ur $k=0$ klar.

 $k \mapsto k+1$: Nach Induktionsvoraussetzung gibt es ein $a \in \NN$ mit $p \mid a$, so dass 
 $$
  \l(1+y\r)^{p^k}-1=ap^{k+1} \Leftrightarrow \l(1+y\r)^{p^k}=ap^{k+1}+1
 $$
 Also ist
 \begin{eqnarray*}
  \l(1+x\r)^{p^{k+1}} &=& \l(ap^{k+1}+1\r)^p=\\&=&1+\binom{p}{1}ap^{k+1}+\sum_{i=1}^p\binom{p}{i}a^i\l(p^{k+1}\r)^i=\\&=&1+ap^{k+2}+\sum_{i=2}^p\binom{p}{i}a^ip^{ki+i}
 \end{eqnarray*}
 F\"ur $i<p$ gilt $p \mid \binom{p}{i}$. Also gilt: $p^{ki+i+1} \mid \binom{p}{i}a^ip^{ki+i}$ und es gilt
 $$
  ik+i+1\geq k+3 \quad \Leftrightarrow \quad \l(i-1\r)k+i\geq 2
 $$
 was f\"ur $i\geq2$ und $k\geq0$ erf\"ullt ist. F\"ur $i=p$ ist der entsprechende Summand durch $p^{kp+p}$ teilbar und es gilt: $pk+p\geq k+3$ f\"ur $p\geq3$. Also folgt:
 $$
  \l(1+y\r)^{pk+1}-1 \equiv ap^{k+2} \mod p^{k+3}
 $$
 Das hei\ss{}t $p^{k+2}$ teilt $\l(1+y\r)p^{k+1}-1$, aber $p^{k+3}$ teilt $\l(1+y\r)p^{k+1}-1$ nicht. Setzen wir nun $\overline{x} := \overline{1}+\overline{y}$, so ist $\overline{x} \in E_n$ und es ist $\overline{x}^{p^k} \neq \overline{1}$ in $\ZZ_{p^k}^*$ f\"ur $k \leq n-2, \overline{x}^{p^{n-1}}=\overline{1}$.

 Da nach \ref{menge_E_n} $\l|E_n\r|=p^{n-1}$, folgt $E_n=\l<\overline{x}\r>$, das hei\ss{}t $E_n$ ist zyklisch. Jedes $\overline{x}$ mit $x-1\not\equiv0\mod p^2$ erzeugt $E_n$ und es gibt $p^{n-1}-p^{n-2}$ Erzeugende dieser Form. Wegen $\varphi\l(p^{n-1}\r)=p^{n-1}-p^{n-2}$ sind diese alle Erzeugenden.
 \qed
\end{beweis}

\begin{satz}
 Sei $p>2$ prim und $k \in \NN$. Dann ist $\ZZ_{p^k}^* \cong \ZZ_{p^{k-1}}\times \ZZ_{p-1}$ zyklisch.
\end{satz}

\begin{beweis}
 Sei $a$ eine Primitivwurzel modulo $p$, das hei\ss{}t $\ZZ_{p^k}^*\cong \l<a\r>$. Wir setzen $r := a^p-p$ und zeigen: $\ZZ_{p^k}^* = \l<r\r>$. Zun\"achst ist $r=a^{p-1}a\equiv a \mod p$ und wegen $\ZZ_{p^k}^*=\l<a\r>$ folgt
 $$r^m\equiv 1 \mod p \quad \Leftrightarrow \quad p-1 \mid m$$
 Ist $r^m\equiv 1 \mod p^k$, so ist insbesondere $r^m\equiv 1 \mod p$, also folgt $p-1$ teilt $\ord r$.

 Jetzt zeigen wir, dass es ein $b \in ZZ$ gibt mit $p \nmid b$, so dass $r^{p-1}=1+bp$. Dann folgt die Behauptung mit Lemma \ref{E_n_zyklisch}. Wegen $r \equiv a \mod p$ ist $r^{p-1} \equiv 1 \mod p$, also ist $r^{p-1}=1+bp$ f\"ur ein $b \in \ZZ$. Wir zeigen noch $p \nmid b$: Es ist
 \begin{eqnarray*}
  r^{p-1}-1&=&\l(a^p-p\r)^{p-1}-1=\\&=&a^{p\l(p-1\r)}-\binom{p-1}{1}\l(a^p\r)^{p-2}p + \\&+&\sum_{i=2}^{p-1}\binom{p-1}{i}\l(a^p\r)^{p-1-i}p^i\l(-1\r)^i-1=\\&\stackrel{\mod p^2}{=}&a^{p\l(p-1\r)}-p^2\l(a^p\r)^{p-2}+a^{p\l(p-2\r)}-1=\\&\stackrel{\mod p^2}{=}&a^{p\l(p-1\r)}-1+a^{p\l(p-2\r)}p
 \end{eqnarray*}
 Da $a^{p-1}=1 \mod p$ gibt es ein $k \in \ZZ$, so dass $a^{p-1}=1+kp$. Damit ist
 \begin{eqnarray*}
  \l(a^{p-1}\r)^p-1&=&\l(1+kp\r)^p-1=\sum_{i=1}^p\binom{p}{i}\l(kp\r)^i=\\&=&p\l(kp\r)+\sum_{i=2}^p\binom{p}{i}\l(kp\r)^i\equiv 0 \mod p^2
 \end{eqnarray*}
 Also erhalten wir:
 $$
  r^{p-1}-1\equiv a^{p\l(p-2\r)}p \mod p^2
 $$
 Da $a \not\equiv 0 \mod p$, ist auch $a^x \not\equiv 0 \mod p$ und daher $r^{p-1}-1\not\equiv 0 \mod p^2$, das hei\ss{}t $r^{p-1}=1+bp$ mit $p \nmid b$. Mit Lemma \ref{E_n_zyklisch} folgt nun $\l<r^{p-1}\r>=E_k$, das hei\ss{}t $\ord r^{p-1}?p^{k-1}$ und damit $p^{k-1} \mid \ord r$. Zusammen mit $p-1\mid \ord r$ erhalten wir $\ord r=\l(p-1\r)p^{k-1}$, also $\ZZ_{p^k}^*=\l<r\r>$.
 \qed
\end{beweis}

\begin{bemerkung}
 In Definition \ref{primitivwurzel} haben wir festgestellt, dass es schwierig ist, eine Primitivwurzel modulo $p$ zu finden. Haben wir aber eine solche gefunden, etwa $a$, so k\"onnen wir sofort eine Primitivwurzel modulo $p^k$ angeben, n\"amlich $r=a^r-p$.
\end{bemerkung}

\section{Gruppenoperationen}

Gruppenoperationen sind ein wichtiges Hilfsmittel mit Anwendungen sowohl innerhalb als auch au\ss{}erhalb der Gruppentheorie (z. B. in der Kombinatorik, Zahlentheorie,...). Um die St\"arke zu demonstrieren, werden wir die Aussage des Satzes von Sylow verallgemeinern.

F\"ur eine Menge $\Omega$ bezeichne $\Symm_\Omega$ die symmetrische Gruppe auf $\Omega$. Gruppenoperationen lassen sich auf zwei \"aquivalente Arten definieren.

\begin{definition}[Operation]\spspace
 \index{Operation}
 \index{Operation!treue}
 \index{Operation!triviale}
\begin{itemize}
 \item Eine \emph{Gruppenoperation von $G$ auf $\Omega$} ist ein Gruppenhomomorphismus $$\varphi:G\to \Symm_\Omega.$$
\end{itemize}
Dazu \"aquivalent:
\begin{itemize}
 \item Eine \emph{Gruppenoperation von $G$ auf $\Omega$} ist eine Abbildung $$\Omega \times G \to \Omega, (\alpha,g)\longmapsto \alpha^g$$ derart, dass f\"ur alle $\alpha\in \Omega$ und alle $g,h\in G$ gilt $(\alpha^g)^h=\alpha^{(gh)},\alpha^1=\alpha$.
\end{itemize}
Der Homomorphismus $\varphi:G\to \Symm_\Omega$ wird dann gegeben durch $$\varphi(g)=\l(\begin{array}{c}\alpha \\ \alpha^g 
\end{array}\r).$$
Die Operation hei\ss{}t \emph{treu} oder \emph{effektiv}, wenn $\varphi$ injektiv ist.\\
Sie hei\ss{}t \emph{trivial}, wenn $\Ker(\varphi) = G$.

\end{definition}

\begin{beispiel} \spspace
 \begin{enumerate}
  \item $\Symm_n$ operiert treu auf $\lbrace 1,\ldots,n \rbrace$.
  \item $\Di_n$ operiert treu auf den Ecken eines regelm\"a\ss{}igen $n$-Ecks. Versieht man die Ecken mit den Zahlen $1,\ldots,n$, so operiert $\Di_n$ auch treu auf $1,\ldots,n$. Daraus folgt $\Di_3 \cong \Symm_3$, die $2$-Sylowgruppen von $\Symm_4$ sind isomorph zu $\Di_4$.
  \item Sei $U\leq G$. $G$ operiert auf den Rechtsnebenklassen von $U$ durch Rechtsmultiplikation. Der zugeh\"orige Homomorphismus ist gerade der in \ref{1.8} konstruierte.
  \item \label{klaus}\label{5.2.4}$G$ operiert durch Konjugation auf 
   \begin{itemize}
    \item der Menge ihrer Elemente,
    \item der Menge der nichtleeren Teilmengen von $G$,
    \item der Menge der Untergruppen (der $p$-Untergruppen, der Untergruppen von Ordnung $n||G|,\ldots$) von $G$,
    \item der Menge der $p$-Sylowgruppen von $G$.
   \end{itemize}
  \item $\Gl(n,\FF_q)$ operiert auf der Menge der k-dimensionalen Untervektor\-r\"aume von $(\FF_q)^n$.

 \end{enumerate}

 
\end{beispiel}

Der letzte Punkt in \ref{5.2.4} verdient es, besonders hervorgehoben zu werden:
\begin{satz}
 Sei $G$ eine Gruppe mit genau $n$ $p$-Sylowgruppen $P_1,\ldots,P_n, n>1$. Durch $\psi:G\to \Symm_n, \psi(g)=\sigma$, wobei $g^{-1}P_1g=P_{\sigma(1)},\ldots,g^{-1}P_ng=P_{\sigma(n)}$ wird ein nichttrivialer Homomorphismus definiert und es gilt $\Ker(\psi)=\bigcap_{i=1}^n \Norm_G(P_i)$.
\end{satz}
\begin{beweis}
 Nach Sylow sind die $p$-Sylowgruppen konjugiert, d. h. $\psi$ ist nicht trivial und es gilt $\Ker(\psi)=\lbrace g\in G| g^{-1}P_ig=P_i, i=1,\ldots,n \rbrace = \bigcap_{i=1}^n \Norm_G(P_i)$.
\end{beweis}
Als kleine Anwendung zeigen wir:
\begin{satz} 
 Ist $|G|=112=2^4\cdot7$, so ist $G$ nicht einfach.
\end{satz}
\begin{beweis}
 Ist die $2$-Sylowgruppe nicht normal in $G$, so gilt $|\Syl_2G|=7$ nach Sylow. Die Operation von $G$ auf $\Syl_2G$ durch Konjugation liefert einen nichttrivialen Homomorphismus $\psi:G\to \Symm_7$. Dann ist auch $\sign\circ\psi:G\to \lbrace -1,1 \rbrace$ ein Homomorphismus. Ist $\sign\circ\psi$ nicht trivial, so ist $\Ker(\sign\circ\psi)\neq G$ und $\Ker(\sign\circ\psi)\nt G$. Sonst ist $\Img(\psi)\leq \Alt_7$. Da $|G| \nmid |\Alt_7|=2^3\cdot 3^2 \cdot 5\cdot 7$, kann $\Alt_7$ keine zu $G$ isomorphe Untergruppe enthalten, also ist $\psi$ weder trivial noch injektiv und $\Ker(\psi)$ daher ein nichttrivialer Normalteiler von $G$.
\end{beweis}
\begin{definition}[Stabilisator]
 \index{Stabilisator}
 \index{Fixuntergruppe|see{Stabilisator}}
\index{Isotropiegruppe|see{Stabilisator}}
\index{Standuntergruppe|see{Stabilisator}}
 $G$ operiere auf $\Omega$. Sei $\alpha\in\Omega$. Die Menge $G_\alpha:=\lbrace g\in G|\alpha^g=\alpha \rbrace$ hei\ss{}t \emph{Stabilisator von $\alpha$ in $G$}. Andere Bezeichnungen sind \emph{Fixuntergruppe, Isotropiegruppe} und \emph{Standuntergruppe}.
\end{definition}

\begin{satz} \label{5.6}
 $G$ operiere auf $\Omega$. Sei $\alpha \in \Omega$. $G_\alpha$ ist eine Untergruppe von $G$ und f\"ur $x\in G$ gilt $G_{\alpha^x}=(G_\alpha)^x$.
\end{satz}
\begin{beweis}
 Seien $x,y\in G_\alpha$, dann ist $\alpha^{xy}=(a^x)^y = \alpha^y = \alpha$. Also ist $xy\in G_\alpha$ und damit ist $G_\alpha$ Untergruppe von $G$.\\
 Es gilt 
\begin{eqnarray*}y\in G_{\alpha^x} &\Longleftrightarrow& (\alpha^x)^y = \alpha^x \\ &\Longleftrightarrow& \alpha^{xyx^{-1}}=\alpha \\&\Longleftrightarrow& xyx^{-1}\in G_\alpha \\ &\Longleftrightarrow& y\in x^{-1}G_\alpha x=(G_\alpha)^x
 \end{eqnarray*}

\end{beweis}

\begin{bemerkung*}
 $G$ operiert durch Konjugation 
\begin{itemize}
 \item auf der Menge ihrer Elemente. Der Stabilisator eines Elements $x\in G$ ist der Zentralisator $\Zen_G(x)$.
 \item auf der Menge der Untergruppen $U\leq G$. Der Stabilisator von $U$ ist der Normalisator $\Norm_G(U)$.
\end{itemize}
Einige Aussagen in (setze REF!) folgen daher aus \ref{5.6}
\end{bemerkung*}

\begin{definition}[Bahn]
 \index{Bahn}
\index{Bahn!L\"ange einer}
$G$ operiere auf $\Omega$. Sei $\alpha \in \Omega$. Die Menge $\alpha^G:=\lbrace \alpha^g|g\in G\rbrace$ hei\ss{}t \emph{Bahn} (engl. \emph{orbit}) \emph{von $\alpha$}. Die Anzahl der Elemente einer Bahn hei\ss{}t auch \emph{L\"ange der Bahn}.
\end{definition}

\begin{bemerkung*}
 Die Bahnen gehen von einer \"Aquivalenzrelation hervor, n\"amlich aus $$\alpha \sim \beta \Longleftrightarrow \exists g\in G : \alpha^g =\beta.$$
Sind daher $B_1,\ldots,B_n$ die Bahnen von $G$ auf $\Omega$, so gilt $$\Omega=\bigcup_{i=1}^n B_i \qquad\mbox{disjunkt}.$$
\end{bemerkung*}

\begin{satz}[Bahnengleichung] 
\addtotoc{Bahnengleichung}
\label{5.8}
 \index{Bahnengleichung}
 $G$ operiere auf $\Omega$. F\"ur $\alpha\in \Omega$ gilt $$|\alpha^g|=\frac{|G|}{|G_\alpha|}$$. Ist $\Lambda$ ein Vertretersystem der Bahnen von $G$ auf $\Omega$, so gilt \begin{equation}
 |\Omega|=\sum_{\alpha\in\Lambda}\frac{|G|}{|G_\alpha|}\tag{Bahnengleichung}
\end{equation}

\end{satz}

\begin{beweis}
 F\"ur $x,y\in G$ gilt
\begin{eqnarray*}
 \alpha^x=\alpha^y &\Longleftrightarrow& \alpha^{yx^{-1}}=\alpha \\&\Longleftrightarrow& yx\in G_\alpha \\&\Longleftrightarrow& y\in G_\alpha x.
\end{eqnarray*}
Es gibt also in der Bahn $\alpha^G$ genauso viele Elemente, wie es Nebenklassen von $G_\alpha$ gibt, n\"amlich $\frac{|G|}{|G_\alpha|}$ St\"uck. Die Bahnengleichung folgt dann aus $$\Omega=\bigcup_{\alpha\in\Lambda}\alpha^G$$.
\end{beweis}
\begin{bemerkung*}
 Sei $H\leq G$. Betrachten wir die Operation von $H$ auf $G$ durch Konjugation, so sind die Bahnen gerade die $H$-Konjugationsklassen und (setze REF!) folgt aus \ref{5.8}. Auch die Klassengleichung (setze REF!) folgt sofort.\\
Lassen wir $H$ auf der Menge der Untergruppen von $G$ durch Konjugation operieren, so folgt (setze REF!).
\end{bemerkung*}

Wir demonstrieren nun die St\"arke des Konzepts der Gruppenoperationen, indem wir im folgenden Satz auf einen Schlag den Satz von Cauchy und die Aussage (setze REF!) des Satzes von Sylow erneut beweisen. Die Aussage aus dem Satz von Sylow wird dabei sogar noch verallgemeinert.\\
Beachte: Der Beweis verwendet au\ss{}er Gruppenoperationen nur Aussagen \"uber Nebenklassen von Untergruppen und Kenntnisse \"uber Untergruppen zyklischer Gruppen.

\begin{satz}[Cauchy, Sylow (\ref{3.10}) und mehr] \label{5.9}
\index{Cauchy!Satz von}
\index{Sylow!Satz von}
 Sei $p$ prim und $|G|=p^a\cdot m$ mit $p\nmid m$. F\"ur $0\leq s\leq a$ bezeichne $r_s$ die Anzahl der Untergruppen der Ordnung $p^s$ von $G$. Dann gilt $r_s \equiv 1 \mod p$. Insbesondere ist $r_s\leq 1$ (Cauchy) und $r_a\equiv 1\mod p$ (Sylow).
\end{satz}

\begin{beweis}[nach Wielandt]
 \index{Wielandt}
Sei $n:=\frac{|G|}{p^s}=p^{a-s}\cdot m$ und sei $\Omega:=\lbrace M\subset G| |M|=p^s \rbrace$. Offensichtlich gilt $|\Omega|=\binom{np^s}{p^s}$. $G$ operiert auf $\Omega$ durch Rechtsmultiplikation, da f\"ur $g\in G$ gilt $|Mg|=|M|$.\\
Seien $T_i$ die Bahnen dieser Operation, $M_i$ ein Repr\"asentant von $T_i$ und sei $U_i:=G_{M_i}=\lbrace g\in G|M_ig=M_i \rbrace$ der Stabilisator von $M_i$. Dann gilt $|T_i|=\frac{|G|}{|U_i|}$. Aus $M_iU_i=M_i$ folgt $M_i=\bigcup_{j=1}^{k_i}g_{ij}U_i$ f\"ur bestimmte $g_{ij}\in G$, das hei\ss{}t $M_i$ ist Vereinigung von Nebenklassen von $U_i$. Also ist $p^s=|M_i|=k_i\cdot|U_i|\Longleftrightarrow |U_i|=\frac{p^s}{k_i}$ und folglich $|U_i|=p^{b_i}$ mit $b_i\leq s$.
Ist $|U_i|=p^{b_i}< p^s$, so gilt $|T_i|=\frac{|G|}{|U_i|}=\frac{p^sn}{p^{b_i}}\equiv 0 \mod pn$. Ist $|U_i|=p^s$, so gilt $|T_i|=\frac{p^sn}{p^s}=n$. Insgesamt erhalten wir \begin{equation}\binom{np^s}{p^s}=|\Omega|\equiv \sum_{|T_i|=n}|T_i|\mod pn \tag{*}\end{equation}
 Wir zeigen nun: $\#\lbrace T_i||T_i|=n\rbrace =r_s$. \\Ist n\"amlich $|T_i|=n$, so gibt es ein $m_i\in M_i$ derart, dass $M_i=m_iU_i$. Dann ist $m_iU_im_i^{-1}=:V_i\leq T_i$ eine Untergruppe der Ordnung $p^s$ in $T_i$.\\Ist andererseits $U\leq G$ mit $|U|=p^s$, so ist $T:=\lbrace Ug|g\in G\rbrace$ eine Bahn mit $|T|=\frac{|G|}{|U|}=n$. Sind $V_i, V_j$ Untergruppen von $G$ mit $|V_i|=|V_j|=p^s$, so gilt $V_i=V_jg$ f\"ur ein $g\in G$. Also gibt es ein $v_j\in V_j$ mit $1=v_jg$ und damit $g=v_j^{-1}\in V_j$, ergo $V_i=V_j$.\\
Kurz: verschiedene Untergruppen mit Ordnung $p^s$ liegen in verschiedenen Bahnen.
Setzen wir das Erhaltene in (*) ein, so erhalten wir $\binom{np^s}{p^s}\equiv nr_s\mod pn$. Diese Kongruenz gilt f\"ur alle Gruppen mit Ordnung $p^am$, insbesondere auch f\"ur die zyklische. Bei der zyklischen Gruppe ist bekanntlich $r_s=1$, also ist $\binom{np^s}{p^s}\equiv n\mod pn$. Damit gilt f\"ur beliebige Gruppen $G$ mit $|G|=p^am$: $n\equiv n\cdot r_s\mod pn$, also $pn|nr_s-n=n(r_s-1)$ und damit $p|r_s-1$ bzw. $r_s\equiv 1\mod p$.
\end{beweis}



\begin{folgerung} \label{5.10}\index{Normalteiler!in $p$-Gruppen}
\index{$p$-Gruppe}
 Sei $p$ prim und $P$ eine Gruppe der Ordnung $p^a$. Sei $1\leq s\leq a$. Dann enth\"alt $P$ einen Normalteiler der Ordnung $p^s$.
\end{folgerung}


\begin{beweis}
 $P$ operiert durch Konjugation auf der Menge $\Omega:=\lbrace U\leq P||U|=p^s\rbrace$ der Untergruppen von $P$ mit Ordnung $p^s$. Sei $r_s:=|\Omega|$. Nach der Bahnengleichung gilt dann $r_s=|\Omega|=\sum_i{\frac{|P|}{P_{U_i}}}$, wobei $U_i$ Repr\"asentanten der verschiedenen Bahnen sind. F\"ur die Stabilisatoren $P_{U_i}$ erhalten wir $P_{U_i}=\lbrace g\in G|g^{-1}U_ig=U_i\rbrace = \Norm_P(U_i)$. Also ist 
\begin{equation*}
 r_s=\sum_i{\frac{|P|}{|\Norm_P(U_i)|}} \tag{*}
\end{equation*}
Weil P eine $p$-Gruppe ist, gilt $ p|\frac{|P|}{\Norm_P(U_i)} \Longleftrightarrow \Norm_P(U_i) \neq P$. Nach \ref{5.9} ist $r_s\equiv 1\mod p$, also $p\nmid r_s$. Also gibt es in (*) einen Summanden, der nicht durch $p$ teilbar ist, d. h. es gibt $U_i\in \Omega$ mit $\Norm_P(U_i) = P \Longleftrightarrow U_i \nt P$.
\end{beweis}

F\"ur weitere Aussagen \"uber Untergruppen von $p$-Gruppen ben\"otigen wir zun\"achst eine Aussage \"uber deren Normalisatoren:
\begin{lemma}[Matsuyama] \label{5.11}
\index{Normalisator!in $p$-Gruppen}
\index{Matsuyama}
\index{$p$-Gruppe}
 Sei $p$ prim, $P$ eine $p$-Gruppe und $H < G$. Dann ist entweder $H\vartriangleleft P$ oder es gibt ein $x\in P$ mit $H^x \neq H$ und $H^x \leq \Norm_P(H)$. 
\end{lemma}
\begin{beweis}
 Sei $H\ntriangleleft P$. Wir setzen $X:=\lbrace H^x|x\in P\rbrace \backslash \lbrace H\rbrace$. Dann ist $X\neq \varnothing$ und $H$ operiert auf $X$ durch Konjugation. Nach Lemma (setze REF!) gibt es $\frac{|P|}{|\Norm_P(H)|}$ $H$-Konjugierte von $H$ in $P$, also ist $|X|=\frac{|P|}{|\Norm_P(H)|}-1\not\equiv 0\mod p$. Die L\"angen der Bahnen von $H$ auf $X$ sind nach Lemma \ref{5.8} Potenzen von $p$. Wegen $|X|\not\equiv 0\mod p$ folgt, dass es eine Bahn der L\"ange $1$ gibt, etwa $\lbrace H^y\rbrace$. Dann ist $h^{-1}H^yh=H^y$ f\"ur $h\in H$, das hei\ss{}t es gilt $H\leq \Norm_P(H^y)$. Setzen wir $x:=y^{-1}$, so ist nach (setze REF!) $H^x\leq \Norm_P(H^y)^x=\Norm_P(H^{yx})=\Norm_P(H)$.
\end{beweis}

\begin{folgerung} \label{5.12}
\index{$p$-Gruppe}
 Sei $p$ prim, $P$ eine $p$-Gruppe. Ist $H <P$, so ist $H < \Norm_P(H)$.
\end{folgerung}

\begin{beweis}
 Ist $H \vartriangleleft P$, so ist $\Norm_P(H)=P > H$. Ist $H\ntriangleleft P$, so gibt es nach \ref{5.11} ein $H^x \neq H$ mit $H^x\leq \Norm_P(H)$, also ist $N_P(H) > H$.
\end{beweis}

\begin{folgerung} \label{5.13}
\index{$p$-Gruppe}
\index{Normalteiler!in $p$-Gruppen}
 Sei $p$ prim, $P$ eine $p$-Gruppe mit $|P|=p^n$. Alle Untergruppen der Ordnung $p^{n-1}$ sind Normalteiler in $P$.
\end{folgerung}
\begin{beweis}
 Nach \ref{5.12} ist $\Norm_P(H) = P$ f\"ur $H < P$ mit $|H| = p^{n-1}$.
\end{beweis}

Mit \ref{5.12} und \ref{5.13} k\"onnen wir einen weiteren Satz \"uber Untergruppen von $p$-Gruppen beweisen:

\begin{satz}\label{5.14}
 \index{$p$-Gruppe}
 \index{Untergruppen!von $p$-Gruppen}
Sei $P$ eine $p$-Gruppe und $H < P$ mit $|H|=p^s$. Sei $$a:=|\lbrace K < P|H < K \text{ und } |K|=p^{s+1}\rbrace|$$ die Anzahl der Untergruppen von $P$ mit Ordnung $p^{s+1}$, die $H$ enthalten. Dann ist $a\equiv 1 \mod p$. Insbesondere ist $a\geq 1$.
\end{satz}

\begin{beweis}
 Nach \ref{5.12} ist $N < \Norm_P(H)$. Ist $K\leq P$ mit $H < K$ und $|K|=p^{s+1}$, so gilt $H\vartriangleleft K$ nach \ref{5.13}. Laut (setze REF!) ist daher $K\leq \Norm_P(H)$. Wir wenden nun den Korrespondenzsatz auf $\theta:\Norm_P(H)\to \Norm_P(H)/H$ an. Die Untergruppen $K$ mit $H < K$ und $|K|=p^{s+1}$ entsprechen dabei den Untergruppen der Ordnung $p$ von $\Norm_P(H)/H$. F\"ur deren Anzahl $a$ gilt $a\equiv 1\mod p$ nach \ref{5.9}.
\end{beweis}

\begin{bemerkung}
 Sei $p$ prim, $P$ eine $p$-Gruppe mit $|P|=p^n$. F\"ur $1\leq s < n$ sei $r_s$ die Anzahl der Untergruppen von $P$ mit Ordnung $p^s$. Nach \ref{5.9} ist $r_s\equiv 1\mod p$. Der einfachste Fall w\"are also $r_s=1$. In diesem Fall l\"asst sich zeigen:
\begin{itemize}
 \item Ist $r_1=1$, so ist $P$ zyklisch oder es ist $p=2$ und $P$ ist eine sogenannte \emph{verallgemeinerte Quaternionengruppe}.
 \item Ist $r_s=1$ f\"ur $s > 1$, so ist $P$ zyklisch.
\end{itemize}

\end{bemerkung}

F\"ur die Kombinatorik interessant ist das folgende Lemma
\begin{satz}[Lemma von Cauchy-Frobenius]
\addtotoc{Lemma von Cauchy-Frobenius}
\index{Cauchy-Frobenius!Lemma von}
\index{Burnside!Lemma von}
\index{Bahn!Anzahl der $\sim$en}
 (oft falsch als Lemma von Burnside bezeichnet)\\
$G$ operiere auf $\Omega$. F\"ur $g\in G$ bezeichne $F(g):=\lbrace \alpha \in \Omega|\alpha^g=\alpha \rbrace$. die Menge der Fixpunkte von $g$. Dann gilt f\"ur die Anzahl $N$ der Bahnen von $G$ auf $\Omega$ $$N=\frac{1}{|G|}\sum_{g\in G}{|F(g)|},$$ das hei\ss{}t, die Zahl der Bahnen ist gleich der durchschnittlichen Zahl der Fixpunkte.
\end{satz}
\begin{beweis}
 Es gilt $\alpha\in F(g) \Longleftrightarrow g\in G_\alpha$. Damit ist 
\begin{eqnarray*}
 \sum_{g\in G}{|F(g)|}&=&\sum_{g\in G}\sum_{\alpha\in F(g)}{1}=\sum_{\alpha\in\Omega}\sum_{g\in G_\alpha}{1}=\\&=&\sum_{\alpha\in\Omega}{|G_\alpha|}\stackrel{\ref{5.8}}{=}|G|\sum_{\alpha\in\Omega}\frac{1}{|\alpha^G|}=|G|\cdot N
\end{eqnarray*}
Im letzten Schritt wurde verwendet, dass jede Bahn den Beitrag $1$ zur Summe liefert.
\end{beweis}

Eine gruppentheoretische Folgerung aus dem Lemma von Cauchy-Frobenius ist 
\begin{folgerung}
 F\"ur die Anzahl $c$ der Konjugationsklassen von $g$ gilt $$c=\frac{1}{|G|}\sum_{g\in G}|\Cen_G(g)|.$$
\end{folgerung}
\begin{beweis}
 $G$ operiert durch Konjugation auf sich selbst, die Bahnen sind die Konjugationsklassen und f\"ur $g\in G$ gilt
$$F(g)=\lbrace x\in G|g^{-1}xg=x\rbrace = \lbrace x\in G|x^{-1}gx=g \rbrace = \Cen_G(g).$$
\end{beweis}

\begin{definition}[transitiv]
 \index{Operation!transitive}
 \index{Operation!$n$-transitive}
$G$ operiere auf $\Omega$. Die Operation hei\ss{}t \emph{transitiv}, wenn es nur eine einzige Bahn gibt, d. h. wenn zu $\alpha,\beta \in \Omega$ ein $g\in G$ existiert mit $\alpha^g=\beta$.\\
Die Operation hei\ss{}t \emph{$n$-transitiv}, wenn zu jedem Paar $(\alpha_1,\ldots,\alpha_n)$, $(\beta_1,\ldots,\beta_n)$ von $n$-Tupeln mit $\alpha_1,\ldots,\alpha_n$ paarweise verschieden (genauso f\"ur $(\beta_1,\ldots,\beta_n)$) ein $g\in G$ existiert derart, dass $\alpha_i^g=\beta_i$ f\"ur $i=1,\ldots,n$.
\end{definition}

\begin{beispiel}
 Offensichtlich ist eine transitive Operation $1$-transitiv und eine $n$-transitive Operation ist $(n-1)$-transitiv. 
\begin{enumerate}
 \item $\Di_n$ operiert transitiv auf den Ecken des regelm\"a\ss{}igen $n$-Ecks.
 \item $\Symm_n$ operiert $n$-transitiv auf $\lbrace 1,\ldots,n\rbrace$.
 \item $\Alt_n$ operiert $(n-2)$-transitiv auf $\lbrace 1,\ldots,n\rbrace$.
\end{enumerate}

\end{beispiel}
\begin{beweis} \spspace
 \begin{enumerate}
  \item $\Di_n$ operiert transitiv, weil schon $\langle d \rangle$ transitiv operiert.
  \item $\l(\begin{array}{ccc} \alpha_1&\ldots&\alpha_n\\\beta_1&\ldots&\beta_n\end{array}\r)$ tut es.
  \item Seien $\alpha_1,\ldots,\alpha_{n-2}\in \lbrace 1,\ldots,n \rbrace$ paarweise verschieden und $\beta_1,\ldots,\beta_{n-2} \in \lbrace 1,\ldots,n\rbrace$ paarweise verschieden. Dann gibt es $\alpha_{n-1},\alpha_n,\beta_{n-1},\beta_n$ derart, dass $\lbrace \alpha_1,\ldots,\alpha_n\rbrace=\lbrace 1,\ldots,n\rbrace=\lbrace\beta_1,\ldots,\beta_n\rbrace$. Wir betrachen $$\sigma=\l(\begin{array}{ccccc} \alpha_1&\ldots&\alpha_{n-2}&\alpha_{n-1}&\alpha_n\\\beta_1&\ldots&\beta_{n-2}&\beta_{n-1}&\beta_n\end{array}\r),$$ $$\pi=\l(\begin{array}{ccccc} \alpha_1&\ldots&\alpha_{n-2}&\alpha_{n-1}&\alpha_n\\\beta_1&\ldots&\beta_{n-2}&\beta_{n}&\beta_{n-1}\end{array}\r).$$
Dann ist $(\alpha_{n-1}\quad\alpha_n)\circ\sigma = \pi$ und daher $\sign(\pi)=-\sign(\sigma)$.
Folglich liegt eine der beiden Permutationen in der $\Alt_n$.
 \end{enumerate}

\end{beweis}

\begin{satz}[Frattini-Argument I]
\addtotoc{Die Frattini-Argumente}
\label{5.22}
 \index{Frattini-Argument I}
 $G$ operiere auf $\Omega$. Besitzt $G$ eine Untergruppe $U$, die transitiv auf $\Omega$ operiert, so gilt $G=G_\alpha U$ f\"ur alle $\alpha\in\Omega$.
\end{satz}
\begin{beweis}
 Sei $\alpha\in\Omega$ fest. Zu jedem $g\in G$ gibt es ein $x\in U$ derart, dass $\alpha^g=\alpha^x$. Also ist $\alpha^{gx^{-1}}=\alpha$ und damit $gx^{-1}\in G_\alpha$, d. h. es gilt $g\in G_\alpha x\subset G_\alpha U$.\\
Bemerke: es gilt auch $G=UG_\alpha$, ersetze $g$ durch $g^{-1}$ im Beweis.
\end{beweis}

\begin{satz}[Frattini-Argument II]
\index{Frattini-Argument II}
 Sei $M \lhd G$. Dann gilt f\"ur jede $p$-Sylowgruppe $P$ von $M$ $$G=M\Norm_G(P).$$
\end{satz}
\begin{beweis}
 Wegen $M \lhd G$ operiert $G$ auf $\Syl_PM$ durch Konjugation. Nach Sylow operiert $M$ dabei transitiv. F\"ur $P\in \Syl_PG$ ist der Stabilisator $G_P=\lbrace g\in G|g^{-1}Pg=P\rbrace =\Norm_G(P)$. Mit dem Frattini-Argument I folgt $G=M\Norm_G(P)$. 
\end{beweis}

\begin{satz}
\index{Operation!transitive}
 $G$ operiere transitiv auf $\Omega$. Dann sind \"aquivalent:
\begin{enumerate}
 \item $G$ operiert $2$-transitiv auf $\Omega$.
 \item F\"ur jedes $x\in\Omega$ operiert $G_x$ transitiv auf $\Omega\backslash\lbrace x\rbrace$.
\end{enumerate}

\end{satz}
\begin{beweis}
 Operiere zun\"achst $G$ $2$-transitiv auf $\Omega$. Seien $\beta,\gamma\in\Omega\backslash\lbrace\alpha\rbrace$. Nach Voraussetzung gibt es ein $g\in G$ mit $(\alpha_g,\beta_g)=(\alpha,\gamma)$. Offensichtlich ist $g\in G_\alpha$.\\
Operiere umgekehrt $G_x$ transitiv auf $\Omega\backslash\lbrace x\rbrace$ f\"ur alle $x\in\Omega$, und seien $\alpha\neq\beta\in\Omega$ und $\gamma\neq\delta\in G$. Dann gibt es ein $g\in G_\alpha$ derart, dass $\beta^g=\delta$ und ein $h\in G_\delta$ mit $\alpha^h=\gamma$. F\"ur das Element $gh$ gilt dann $(\alpha^{gh},\beta^{gh})=(\alpha^h,\delta^h)=(\gamma,\delta)$.
\end{beweis}




\section{Aufl\"osbare Gruppen}

Bevor wir zu den aufl\"osbaren Gruppen kommen, beweisen wir zun\"achst den Satz von Jordan-H\"older. Dieser kann als Versuch gesehen werden, f\"ur Gruppen eine Art 'Primfaktorzerlegung' zu definieren.

\begin{definition}
\index{Normalreihe}
\index{Kompositionsreihe}
 Eine \emph{Normalreihe} $\GGC=(G_i)_{0\leq i \leq n}$ ist eine Kette $$G=G_0 \ntr G_1\ntr \ldots\ntr G_n=\langle 1 \rangle$$ von Untergruppen von $G$.\\
Die Faktorgruppen $G_{i-1}/G_i$ hei\ss{}en \emph{Faktoren} der Normalreihe $\GGC$.\\
Eine Normalreihe $\GGC'$ von $G$ hei\ss{}t \emph{feiner} als $\GGC$, wenn $\GGC$ Teilfolge von $\GGC'$ ist.\\
Zwei Normalreihen $\GGC = (G_i)_{0\leq i\leq n}$ und $\GGC'=(G'_j)_{0\leq j\leq m}$ hei\ss{}en \emph{isomorph}, wenn $n=m$ und wenn es ein $\sigma\in \Symm_n$ gibt derart, dass $$G_i/G_{i+1}\cong G'_{\sigma(i)}/G'_{\sigma(i)+1}.$$
Eine Normalreihe mit lauter einfachen Faktoren $\neq \langle 1 \rangle$ hei\ss{}t \emph{Kompositionsreihe}.
\end{definition}

\begin{beispiel} \spspace
 \begin{enumerate}
\index{Normalreihe}
\index{Kompositionsreihe}
 \item $\Di_n \ntr \langle d \rangle \ntr \langle 1 \rangle$ ist eine Normalreihe.
 \item $\ZZ_36 \ntr \ZZ_12 \ntr \ZZ_6 \ntr \langle 1 \rangle$ und $\ZZ_36\ntr\ZZ_18\ntr \ZZ_3\ntr \langle 1 \rangle$ sind isomorphe Normalreihen, obwohl die Gruppen innerhalb der Reihen nicht isomorph sind.
 \item Ist $p$ prim, so ist $\ZZ_{p^n}\ntr\ZZ_{p^{n-1}}\ntr\ldots\ntr\ZZ_p\ntr\langle 1 \rangle$ eine Kompositionsreihe von $\ZZ_{p^n}$, die Faktorgruppen sind isomorph zu $\ZZ_p$ und damit einfach. 
\end{enumerate}

\end{beispiel}

\begin{bemerkung} \label{6.3}
 Sei $U\nt U^*\leq G$ und $V\leq G$. Dann gilt
$$U\cap V \nt U^*\cap V.$$
\end{bemerkung}

\begin{beweis}
 Sei $u\in U^*\cap V$. Weil $u\in U^*$ und $U\nt U^*$ ist dann $u^{-1}(U\cap V)u\leq U$. Da $u\in V$ und $U\cap V\leq V$, ist auch $u^{-1}(U\cap V)u\leq V$ und daher $u^{-1}(U\cap V)u\leq U\cap V$.\qed
\end{beweis}
\begin{lemma}[Zassenhaus] \label{6.4}
\index{Zassenhaus!Lemma von}
 Seien $U\nt U^*\leq G$ und $V\nt V^*\leq G$. Dann gilt
$$\frac{U(U^*\cap V^*)}{U(U^*\cap V)}\cong \frac{V(U^*\cap V^*)}{V(U\cap V^*)}.$$
\end{lemma}
\begin{beweis}
 Wegen $U\nt U^*$ gilt $U\cap V^* \nt U^*\cap V^*$ nach \ref{6.3} und analog folgt $U^*\cap V \nt U^*\cap V^*$. Damit ist $W:=(U\cap V^*)(U^*\cap V)$ ein Normalteiler in $U^*\cap V^*$. Aus $U\nt U^*$ folgt auch $U\nt U(U^*\cap V^*)\leq U^*$. In der Gruppe $U(U^*\cap V^*)$ wenden wir den 1. Isomorphiesatz an und erhalten $$\frac{U(U^*\cap V^*)}{U}\stackrel{\varphi}{\cong}\frac{U^*\cap V^*}{U\cap U^*\cap V^*}=\frac{U^*\cap V^*}{U\cap V^*}.$$
Dabei bezeichne $\varphi$ den Isomorphismus aus dem 1. Isomorphiesatz. Wenden wir $\varphi$ auf die Untergruppe $U(U^*\cap V)/U$ an, so ergibt sich $$\varphi\l(\frac{U(U^*\cap V)}{U}\r)=\frac{(U^*\cap V)(U\cap V^*)}{U\cap V^*}=\frac{W}{U\cap V^*}.$$
Oben hatten wir bereits $W\nt U^*\cap V^*$ bewiesen. Mit dem 2. Isomorphiesatz folgt
\begin{eqnarray*}
 \frac{U^*\cap V^*}{(U^*\cap V)(U\cap V^*)}=\frac{U^*\cap V^*}{W}&\stackrel{\text{2. Isom.}}{\cong}&\frac{(U^*\cap V^*)/(U\cap V^*)}{W/(U\cap V^*)}\stackrel{\varphi^{-1}}{\cong}\\\frac{U(U^*\cap V^*)/U}{U(U^*\cap V)/U}&\stackrel{\text{2. Isom.}}{\cong}&\frac{U(U^*\cap V^*)}{U(U^*\cap V)}.
\end{eqnarray*}
Aufgrund der Symmetrie des Ausdrucks auf der linken Seite sowie der Voraussetzungen an $U$ und $V$ sind wir fertig.
\end{beweis}


\begin{satz}[Verfeinerungssatz von Schreier und Zassenhaus] \label{6.5}
 \index{Verfeinerungssatz}
\index{Zassenhaus}
\index{Schreier}
\index{Normalreihe}
Zwei Normalreihen $$G=G_0\ntr G_1\ntr G_2\ntr \ldots \ntr G_n=\langle 1 \rangle$$ und 
                  $$G=G'_0\ntr G'_1\ntr G'_2\ntr \ldots \ntr G'_m=\langle 1 \rangle$$ einer Gruppe $G$ besitzen isomorphe Verfeinerungen.
\end{satz}
\begin{beweis}
 Wir bilden $G_{i,j}:=G_i(G_{i-1}\cap G'_j)$ und $G'_{j,i}:=G'_j(G'_{j-1}\cap G_i)$ f\"ur $0\leq i\leq n$ und $0\leq j\leq m$, wobei wir $G_{-1}:=G'_{-1}:=G$ setzen. Nach \ref{6.4} gilt dann $G_{i,j+1}\nt G_{i,j}$ und $G'_{j,i+1}\nt G'_{j,i}$. 
Also sind 
$$G=G_{0,0}\ntr G_{0,1}\ntr \ldots \ntr G_{0,m}=G_{1,0}\ntr G_{1,1}\ntr \ldots \ntr G_{n,m-1}\ntr G_{n,m}=\langle 1 \rangle$$ und 
$$G=G'_{0,0}\ntr G'_{0,1}\ntr\ldots \ntr G'_{0,n}=G'_{1,0}\ntr G'_{1,1}\ntr\ldots\ntr G'_{m,n-1}\ntr G'{m,n}=\langle 1 \rangle$$ Verfeinerungen der urspr\"unglichen Normalreihen. Mit \ref{6.4} folgt auch 
$$G_{i,j-1}/G_{i,j}=\frac{G_i(G_{i-1}\cap G'_{j-1})}{G_i(G_{i-1}\cap G'_j)}\cong\frac{G'_j(G_{i-1}\cap G'_{j-1})}{G'_j(G_i\cap G'_{j-1})}=G'_{j,i-1}/G'_{j,i}.$$ f\"ur $i=1,\ldots,n$ und $j=1,\ldots,m$. Die Verfeinerungen sind also isomorph.
\end{beweis}

\begin{satz}[Jordan-H\"older]
 \index{Jordan-H\"older!Satz von}
 \index{Kompositionsreihe}
Je zwei Kompositionsreihen einer endlichen Gruppe $G$ sind isomorph.
\end{satz}
\begin{beweis}
 Die Kompositionsreihen lassen sich nicht ohne Wiederholungen verfeinern. Nach \ref{6.5} sind sie daher isomorph.
\end{beweis}
\begin{bemerkung} 
\index{einfache Gr.}
\index{$\PSL$}
\index{$\PSU$}
 Obwohl die Gruppen, die in verschiedenen Kompositionsreihen von $G$ stehen, in der Regel ganz verschieden sind, sind die Faktorgruppen der Reihen isomorph (nach einer Permutation). Die einfachen Gruppen, die als Faktoren einer Kompositionsreihe von $G$ vorkommen, sind also kennzeichnend f\"ur $G$. Wenn man so will, bilden sie eine Art 'Primfaktorzerlegung' von $G$. Allerdings m\"ussen Gruppen mit derselben 'Primfaktorzerlegung' keineswegs isomorph sein, beispielsweise haben alle $p$-Gruppen der Ordnung $p^n$ isomorphe Kompositionsreihen. F\"ur die Klassifikation von Gruppen sind die Faktoren also nur begrenzt hilfreich.
Dennoch w\"are es im 1. Schritt interessant, die m\"oglichen Faktoren, d.h. die endlichen einfachen Gruppen zu kennen. Wir kennen bisher nur die $\ZZ_p$ f\"ur $p$ prim.
Im n\"achsten Kapitel werden wir zeigen, dass die alternierenden Gruppen $\Alt_n$ f\"ur $n\geq 5$ einfach sind.\\
Der wohl tiefliegendste Satz der Mathematik, endg\"ultig bewiesen in den 1980ern, ist die Klassifikation der endlichen einfachen Gruppen. Neben $\ZZ_p$ und $\Alt_n$ gibt es noch 16 weitere unendliche Serien einfacher Gruppen, z. B. $\PSL(n,a):=\frac{\SL(n,a)}{\Zen(\SL(n,a))}, \PSU(n,a),\ldots$ ($n,a$ nicht zu klein).\\
Zus\"atzlich gibt es noch 26 sog. \emph{sporadische} einfache Gruppen, die zu keiner der Serien geh\"oren, z. B. die Mathieu-Gruppe, das Monster und das Babymonster. Die sporadischen Gruppen haben oft Verbindung zu anderen Teilgebieten der Mathematik, z. B. zur Kodierungstheorie.\\
Auch wenn die Faktoren die Gruppe nicht eindeutig festlegen, bestimmen sie oft Eigenschaften der Gruppe. Das ist die gruppentheoretische Motivation f\"ur die folgende
\end{bemerkung}
\begin{definition}[Aufl\"osbarkeit, $p$-Aufl\"osbarkeit]
 \index{Aufl\"osbarkeit}
 \index{Aufl\"osbarkeit!$p$-}
Eine Gruppe $G$ hei\ss{}t \emph{aufl\"osbar}, wenn sie eine Normalreihe mit abelschen Faktoren besitzt.\\
$G$ hei\ss{}t \emph{$p$-aufl\"osbar}, wenn $G$ eine Normalreihe besitzt, deren Faktoren entweder $p$- oder $p'$-Gruppen sind.
\end{definition}

\begin{bemerkung}
 Abelsche Gruppen sind aufl\"osbar, denn $G\ntr \langle 1 \rangle$ ist eine Normalreihe mit abelschen Faktoren.\\
Die Eigenschaft der Aufl\"osbarkeit kann als Verallgemeinerung der Kommutativit\"at gesehen werden.\\
Aufl\"osbare Gruppen sind $p$-aufl\"osbar f\"ur jede Primzahl $p$.\\
Der Begriff 'aufl\"osbar' kommt tats\"achlich von der Aufl\"osbarkeit von Gleichungen. Bekanntlich gibt es L\"osungsformeln f\"ur quadratische, kubische und Gleichungen 4. Grades. Seit dem Mittelalter wurde intensiv nach L\"osungsformeln f\"ur Gleichungen 5. Grades gesucht, d. h. nach Formeln f\"ur die Nullstellen von Polynomen, die die L\"osung durch mehrfaches Wurzelziehen liefern. Solche Formeln nennt man \emph{Aufl\"osung durch Radikale}. 1824 bewies der Norweger Niels Henrik Abel \index{Abel!Niel Henrik}, dass Gleichungen 5. Grades im Allgemeinen nicht durch Radikale aufl\"osbar sind. Genauer gilt: die Nullstellen eines Polynoms $f$ sind genau dann durch Radikale zu beschreiben, wenn die zugeh\"orige \emph{Galoisgruppe} $\Gal(f)$ aufl\"osbar ist.\index{Galoisgruppe}
$\Gal(f)$ l\"asst sich in die symmetrische Gruppe $\Symm_n, (n=\deg(f))$ einbetten. $\Symm_2, \Symm_3, \Symm_4$ sind aufl\"osbar, $\Symm_5$ nicht mehr. Es gibt tats\"achlich Polynome $f$ vom Grad 5 mit $\Gal(f)\cong \Symm_5$, z. B. $f(X)=X^5-5X-1$.
\end{bemerkung}

\begin{satz} \label{6.10}
\index{Aufl\"osbarkeit}
 $G$ ist genau dann aufl\"osbar, wenn es ein $n\in \NN$ gibt derart, dass $G^{(n)}=\langle 1 \rangle$.
\end{satz}

\begin{beweis}
 Sei zun\"achst $G^(n)=\langle 1 \rangle$. Dann ist $$G\ntr G'\ntr G''\ntr\ldots\ntr G^{(n)}=\langle 1 \rangle$$ eine Normalreihe. Mit \ref{2.15}\ref{2.15.4} folgt, dass die Faktoren abelsch sind.\\
Sei umgekehrt $$G=G_0\ntr G_1\ntr\ldots\ntr G_n= \langle 1 \rangle$$ eine Normalreihe mit abelschen Faktoren. Wir zeigen induktiv, dass $G^{(i)}\leq G_i$.
Da $G/G_n$ abelsch ist, gilt $G'\leq G_1$ nach \ref{2.15}\ref{2.15.4}. \\
Sei nun $G_i/G_{i+1}$ abelsch. Also ist $G_{i+1}\leq G'_i= \langle \lbrack G_i, G_i \rbrack \rangle \stackrel{\text{I.V.}}{\geq}\langle \lbrack G^{(i)},G^{(i)}\rbrack\rangle = G^{(i+1)}$.
Wegen $G_n=\langle 1\rangle $ ist auch $G^{(n)}=\langle 1\rangle$.
\end{beweis}

Der folgende Satz gilt sinngem\"a\ss{} auch f\"ur $p$-aufl\"osbare Gruppen, allerdings muss man dann im Beweis die entsprechenden Normalreihen angeben.
\begin{satz}\spspace \label{6.11} \index{Aufl\"osbarkeit}
\begin{enumerate}
 \item Unter- und Faktorgruppen aufl\"osbarer Gruppen sind aufl\"osbar.
 \item Sei $N\nt G$. Sind $N$ und $G/N$ aufl\"osbar, so auch $G$. \label{6.11.2}
 \item (Semi-)direkte Produkte aufl\"osbarer Gruppen sind aufl\"osbar.
\end{enumerate}

\end{satz}

\begin{beweis}\spspace
 \item Es gilt $U^{(n)}\leq G^{(n)}$ und nach \ref{2.17} gilt $(G/N)^{(n)} = G^{(n)}N/N$. Mit \ref{6.10} folgt die Behauptung.
 \item Seien $N=N_0\ntr N_1\ntr\ldots\ntr N_n = \langle 1 \rangle$ und $G/N=M_0\ntr M_1\ntr\ldots\ntr M_m=\langle 1 \rangle$ Normalreihen mit abelschen Faktoren. Sei $v:G\to G/N$ die kanonische Einbettung. Wir setzen $G_i:=v^{-1}(M_i)$ und zeigen, dass $$G=G_0\ntr G_1\ntr\ldots\ntr G_m = N\ntr N_1\ntr \ldots \ntr N_n=\langle 1 \rangle$$ eine Normalreihe mit abelschen Faktoren ist. \\
Es gilt $G_i/N = M_i\nt M_{i-1}=G_{i-1}/N$. Sei $v_{i-1}:G_{i-1}\to G_{i-1}/N$ der kanonische Epimorphismus. Der Korrespondenzsatz, angewandt auf $v_{i-1}$ liefert $G_i\nt G_{i-1}$. Da sogar $N\nt G$, gilt auch $N\nt G_i$ f\"ur $i=0,\ldots,n$.
Mit dem 2. Isomorphiesatz folgt
$$G_{i-1}/G_i\cong (G_{i-1}/N)/(G_i/N)=M_{i-1}/M_i,$$ also sind die Faktoren $G_{i-1}/G_i$ abelsch.
\item Sei $N\nt G$ und $G\cong N\rtimes U$. Dann ist $G/N\cong U$, nach Voraussetzung sind $N$ und $U$ aufl\"osbar, nach \ref{6.11.2} also auch $G$. Direkte Produkte sind ein Spezialfall.
\end{beweis}

\input{section_7.tex}
\input{section_8.tex}
\section{Die Frattini-Gruppe}

\begin{definition} \index{Frattini-Gruppe}
 Die \emph{Frattini-Gruppe} $\Phi(G)$ einer Gruppe $G$ wird definiert als der Durchschnitt aller maximalen Untergruppen von $G$.
\end{definition}

\begin{bemerkung*}
 $\Phi(G)\Char G$, weil Automorphismen die Maximalit\"at von Untergruppen erhalten. Wir zeigen nun, da\ss{} $\Phi(G)$ die Menge der ``Nicht-Erzeuger'' ist, das hei\ss{}t die Menge der Elemente, die man in jedem Erzeugendensystem weglassen kann.
\end{bemerkung*}

\begin{lemma}\spspace \label{9.2}\index{Frattini-Gruppe}
 \begin{enumerate}
 \item Sei $A\subset \Phi(G)$ und $B\subset G$ mit $G=\langle A, B \rangle$. Dann ist $G=\langle B\rangle$.
 \item Sei $g\in G$ und $G=\langle g_1, g_2,\ldots , g_n\rangle$. Dann ist $G=\langle g_1g, g_2, \ldots , g_n\rangle$.
\end{enumerate}

\end{lemma}

\begin{beweis}\spspace
 \begin{enumerate}
 \item Ist $\langle B\rangle < G$, so gibt es eine maximale Untergruppe $M$ mit $\langle B\rangle\leq M$. Wegen $A\subset \Phi(G)\leq M$ ist dann $G=\langle A, B\rangle \leq M\leq G$, ein Widerspruch.
 \item Angenommen, es ist $U:=\langle g_1g, g_2, \ldots , g_n\rangle < G$. Dann gibt es eine maximale Untergruppe $M < G$ mit $U\leq M$. Da $g\in \Phi(G)\leq M$, ist auch $g^{-1}\in M$ und folglich $(g_1g)g^{-1}=g_1\in M$. Damit ist $G=\langle g_1, \ldots , g_n\rangle\leq M < G$, ein Widerspruch.
\end{enumerate}

 
\end{beweis}

\begin{lemma}[Dedekind-Identit\"at] \label{9.3}
 \index{Dedekind-Identit\"at}
 Ist $A\leq C\leq G$ und $B\leq G$, so gilt $(AB)\cap C=A(B\cap C)$.
\end{lemma}

\begin{beweis}
 Sei zun\"achst $c\in (AB)\cap C$. Dann gibt es $a\in A, b\in B$ derart, dass $c=ab$. Da $a\in A\subset C$, gilt $b=a^{-1}c\in C$. Also ist $c=ab \in A(B\cap C)$.\\
Ist nun $c\in A(B\cap C)$, so gibt es $a\in A, b\in B\cap C$ mit $c=ab$. Da $b\in B\cap C\subset B$, ist $c\in AB$. Wegen $A\leq B$ und $B\cap C\leq C$ ist auch $c\in C$ und insgesamt $c\in (AB)\cap C$.
\end{beweis}

\begin{satz} \spspace
\index{Frattini-Gruppe}
 \begin{enumerate}
 \item Gilt $N\nt G, U\leq G$ und $N\leq \Phi(U)$, so ist $N\leq \Phi(G)$. \label{9.4.1}
 \item Aus $M\nt G$ folgt $Phi(M)\leq \Phi(G)$.
 \item Warnung: $U < G \stackrel{\text{i.A.}}{\Longrightarrow} \Phi(U)\leq \Phi(G)$.
 \item $\Phi(G\times H)=\Phi(G)\times \Phi(H)$.
\end{enumerate}
 
\end{satz}

\begin{beweis}\spspace
 \begin{enumerate}
 \item Angenommen, es ist $N\nleq \Phi(G)$. Dann gibt es eine maximale Untergruppe $V$ von $G$ mit $N\nleq V$. Wegen $N\nt G$ ist $NV\leq G$ mit $NV > V$. Die Maximalit\"at von V liefert $NV=G$. Nun ist $U=U\cap NV \stackrel{\ref{9.3}}{=}N(U\cap V)\stackrel{\ref{9.2}}{=}U\cap V$, denn aus $U=N(U\cap V)$ folgt $U=\langle N, U\cap V\rangle$, die Elemente von $N\leq \Phi(U)$ k\"onnen nach \ref{9.2} aber weggelassen werden, das hei\ss{}t, wir erhalten $U=\langle U\cap V\rangle = U\cap V$. Damit ist aber $N\leq U=U\cap V\leq V$, ein Widerspruch zu $N\nleq V$.
 \item Da $\Phi(M)\Char M\nt G$, ist $\Phi(M)\nt G$ nach \ref{2.20}. Setzen wir in \ref{9.4.1} $N=\Phi(M), U=M$, so folgt die Behauptung.
 \item Beispiel: $\ZZ_5 \stackrel{\varphi}{\rtimes}\ZZ_4$, wobei $\varphi:\ZZ_4\to\ZZ_5^*$ Isomorphismus
 \item \"Ubung
\end{enumerate}

\end{beweis}

\begin{satz}\index{Frattini-Gruppe}\label{9.5}
 $\Phi(G)$ ist nilpotent.
\end{satz}
\begin{beweis}
 Sei $P$ eine $p$-Sylowgruppe von $\Phi(G)$. Da $\Phi(G)\nt G$, k\"onnen wir das Frattini-Argument II (\ref{5.22}) anwenden und erhalten $G=\Phi(G)N_G(P)$. In $G=\langle \Phi(G), N_G(P)\rangle$ k\"onnen wir nach \ref{9.2} aber $\Phi(G)$ weglassen und erhalten $G=\langle N_G(P)\rangle = N_G(P)$. Damit ist $P\nt G$, also insbesondere $P\nt \Phi(G)$. Nach \ref{1.14} ist nun $\Phi(G)$ das direkte Produkt ihrer Sylowgruppen und daher gem\"a\ss{} \ref{8.3} nilpotent.
\end{beweis}

\begin{satz}[Gasch\"utz]\index{Gasch\"utz, Satz von}
 In jeder Gruppe $G$ gilt $G'\cap \Zen(G)\leq \Phi(G)$.
\end{satz}
\begin{beweis}
 Sei $D:=G'\cap \Zen(G)$. Ist $D\nleq\Phi(G)$, so gibt es eine maximale Untergruppe $M<G$ mit $D\nleq M$. Wegen $D\nt G$ ist $MD\leq G$ und wegen der Maximalit\"at von $M$ ist dann $DM=G$. Also gibt es zu $g\in G$ Elemente $d\in D, m\in M$ derart, dass $g=dm$. Nun gilt $g^{-1}Mg=m^{-1}d^{-1}Mdm\stackrel{d\in D\leq \Zen(G)}{=}m^{-1}Mm=M$, das hei\ss{}t, $M\nt G$. Nach Folgerung \ref{3.4} ist somit $[G:M]$ eine Primzahl, insbesondere ist $G/M$ daher abelsch. Laut \ref{2.15} gilt nun $D\leq G'\stackrel{\ref{2.15}}{\leq}M$, ein Widerspruch zu $D\nleq M$.
\end{beweis}

\begin{satz}\label{9.7}
 \index{Nilpotenz}
 \index{Frattini-Gruppe}
 F\"ur eine endliche Gruppe $G$ sind \"aquivalent:
\begin{enumerate}
 \item $G$ ist nilpotent.\label{9.7.1}
 \item $G'\leq \Phi(G)$.\label{9.7.2}
 \item $G/\Phi(G)$ ist nilpotent.\label{9.7.3}
\end{enumerate}

\end{satz}

\begin{beweis}
 Sei zun\"achst $G$ nilpotent, wir zeigen $G'\leq\Phi(G)$. Sei hierzu $M < G$ eine maximale Untergruppe. Weil $G$ nilpotent ist, gilt $M\nt G$. Mit Folgerung \ref{3.4} folgt: $G/M$ hat Primzahlordnung, ist also abelsch. Nach \ref{2.15} gilt nun $G'\leq M$. Da dies f\"ur alle maximalen Untergruppen gilt, ist $G'\leq \Phi(G)$.\\
 Gilt nun \ref{9.7.2}., so ist $G/\Phi(G)$ nach \ref{2.15} abelsch und somit nilpotent. Also folgt \ref{9.7.3}..\\
 Sei jetzt $G/\Phi(G)$ nilpotent und damit direktes Produkt von Sylowgruppen. Seien $P_i\leq G$ mit $\Phi(G)\leq P_i$ derart, dass $P_i/\Phi(G)$ $p_i$-Sylowgruppen von $G/\Phi(G)$ sind, das hei\ss{}t, es gelte $G/\Phi(G)=P_1/\Phi(G)\times\ldots\times P_r/\Phi(G)$.
Da $P_i/\Phi(G)\nt G/\Phi(G)$ Normalteiler sind, sind nach dem Korrespondenzsatz auch $P_i\nt G$. Sei $Q_i$ die $p_i$-Sylowgruppe von $P_i$. Nun gilt $P_i=Q_i\Phi(G)$. Da $|P_i|=|P_i/\Phi(G)|\cdot|\Phi(G)|$ gilt, enth\"alt $P_i$ bereits eine $p_i$-Sylowgruppe von $G$, das hei\ss{}t, $Q_i$ ist $p_i$-Sylowgruppe von $G$. Mit dem Frattini-Argument II (\ref{5.22}) folgt aus $P_i\nt G$ nun $G=\Norm_G(Q_i)P_i=\Norm_G(Q_i)Q_i\Phi(G)=\Norm_G(Q_i)\Phi(G)$.
Aus $G=\langle\Norm_G(Q_i), \Phi(G)\rangle$ folgt mit \ref{9.2}, dass $G=\langle\Norm_G(Q_i)\rangle=\Norm_G(Q_i)$. Damit ist $Q_i \nt G$.\\
Gilt $p||\Phi(G)|$, aber $p\nmid |G/\Phi(G)|$ f\"ur eine Primzahl $p$, so ist die $p$-Sylowgruppe $P$ ebenfalls Normalteiler von $G$, da dann $P\Char \Phi(G)\nt G$ nach \ref{9.5}.
Also sind alle Sylowgruppen von $G$ Normalteiler in $G$ und nach \ref{1.14} ist $G$ das direkte Produkt ihrer Sylowgruppen, also nilpotent

\end{beweis}


\begin{bemerkung}\spspace
\index{Herzog}
\index{Kaplan}
\index{Lev}
\begin{enumerate}
 \item Mit Hilfe des Satzes von Zassenhaus werden wir in Kapitel \ref{11} noch beweisen: Ist $p$ prim mit $p| |\Phi(G)|$, so gilt $p||G/\Phi(G)|$.
\item Nach einem Satz von Herzog, Kaplan und Lev (2004) gilt: Sei $G\neq \langle 1\rangle$ aufl\"osbar mit $|G'|\leq \sqrt[3]{|G|}$. Dann ist entweder $\Zen(G)\neq \langle 1\rangle$ oder $\Phi(G)\neq \langle 1 \rangle$.
\end{enumerate}

\end{bemerkung}


%%%%%%%%%%%%%%%%%%%%%%%%%%%%%%%%%%%%%%%%%%%%%%%%%%%%%%%%%%%%%%%%%%%%%%%%%%%%%%%%%%%%%%%%%%%%%
%
% Baustellen im Text sind mit '[ToDo]' im Text gekennzeichnet und koennen mit der Suchfunktion
% leicht gefunden werden.
%
%%%%%%%%%%%%%%%%%%%%%%%%%%%%%%%%%%%%%%%%%%%%%%%%%%%%%%%%%%%%%%%%%%%%%%%%%%%%%%%%%%%%%%%%%%%%%

\section{$p$-Gruppen}
In diesem Kapitel wollen wir uns ausf\"uhrlich mit $p$-Gruppen besch\"aftigen - nach Sylow bilden sie ja sozusagen die Bausteine aller Gruppen. Allerdings war der Satz von Sylow bisher auch unser Hauptwerkzeug bei der Untersuchung von Gruppen, so dass wir nun kaum Mittel zur Verf\"ugung haben. Dar\"uber hinaus gibt es ein prinzipielles Problem: es gibt sehr viele $p$-Gruppen, f\"ur die Anzahl der $2$-Gruppen siehe zum Beispiel Tabelle \ref{tab:anzahl 2-gruppen}.
\begin{table}[hbp]
 \begin{tabular}{l|cccccccccc}
$|G|$         &2&4&8&16&32&64 &128 &256  &512     &1024\\
Anzahl&1&2&5&14&51&267&2328&56092&10494213& $>$ 49 Mia.\\
\end{tabular}
\caption{Anzahl der 2-Gruppen in Abh\"angigkeit der Kardinalit\"at}
\label{tab:anzahl 2-gruppen}
 
\end{table}

Klassifizieren ist also schwierig Und allgemeine Aussagen beweisen? Wir hatten ja bereits gezeigt:
\begin{itemize}
\addtotoc{bisherige Ergebnisse \"uber $p$-Gruppen}
\index{$p$-Gruppen!Eigenschaften von}
 \item $p$-Gruppen besitzen ein nichttriviales Zentrum (\ref{2.10}).
 \item $|P|=p^2 \Longrightarrow P\cong \ZZ_{p^2}$ oder $P\cong \ZZ_p \times \ZZ_p$ (\ref{2.11}).
 \item F\"ur $r_s:=\lbrace U\leq P\mid|U|=p^s\rbrace$ gilt $r_s\equiv 1 \mod p$ (\ref{5.9}).
 \item F\"ur jedes $1\leq s\leq n, |P|=p^n$ gibt es ein $N\nt P$ mit $|N|=p^s$ (\ref{5.10}).
 \item F\"ur $U<p$ gilt $U<\Norm_P(U)$ (\ref{5.12}). Insbesondere ist jede maximale Untergruppe Normalteiler.
 \item Absteigende bzw. aufsteigende Normalreihen erreichen $\langle 1\rangle$ bzw. $P$ (\ref{8.3}).
 \item $\langle 1\rangle\neq N\nt P \Longrightarrow n\cap \Zen(P)>1$ (\ref{8.5}).
\end{itemize}
Wie der Burnsidesche Basissatz zeigen wird, spielt die Frattini-Gruppe $\Phi(P)$ bei der Untersuchung von $p$-Gruppen eine wichtige Rolle. Zum Beweis ben\"otigen wir ein 
\begin{lemma}\label{10.1}
 Sei $K$ eine Menge von Gruppen, die abgeschlossen ist bez\"uglich der Bildung von Untergruppen sowie direkten Produkten und unter Isomorphie. Weiter seien $M_1,\ldots,M_k$ alle Normalteiler einer endlichen Gruppe $G$ derart, dass $G/M_i \in K$. Dann ist $D=\bigcap_{i=1}^kM_i$ der kleinste Normalteiler von $G$, dessen Faktorgruppe $G/D$ in $K$ liegt.
\end{lemma}

\begin{beweis}
 Wir betrachten den Homomorphismus $$\varphi:G\to \Pi_{i=1}^kG/M_i, g\mapsto (gM_1, \ldots, gM_k).$$ Dann ist $\Ker\varphi=\bigcap_{i=1}^kM_i=D$. Nach dem Homomorphiesatz ist $G/\Ker\varphi$ isomorph zu einer Untergruppe von $\Pi_{i=1}^kG/M_i\in K$. Weil $K$ abgeschlossen ist bez\"uglich Untergruppenbildung, folgt $G/D\in K$. Als Durchschnitt aller Normalteiler $N$ mit $G/N\in K$ ist $D$ der kleinste dieser Normalteiler.
\end{beweis}

\begin{satz}[Burnsidescher Basissatz]
\addtotoc{Burnsidescher Basissatz}
\label{10.2}
 \index{Burnside!Basissatz von}
 \index{Basissatz|see{Burnside}}
F\"ur jede $p$-Gruppe $P$ gilt
\begin{enumerate}
 \item \label{10.2.1}Die Frattini-Gruppe $\Phi(P)$ ist der kleinste Normalteiler von $P$ derart, dass $P/N$ elementarabelsch (das hei\ss{}, $P/N\cong (\ZZ_p)^k$) ist.
 \item \label{10.2.2}Sei $|P/\Phi(P)|=p^n$. Dann ist $n$ die kleinste Zahl, f\"ur die $x_1,\ldots,x_n\in P$ existieren mit $P=\langle x_1,\ldots,x_n\rangle$.
\end{enumerate}


\end{satz}
\begin{beweis}\spspace
\begin{enumerate}
 \item Sei $K$ die Menge aller elementarabelschen $p$-Gruppen. Diese erf\"ullt die Bedingungen in Lemma \ref{10.1}. Sei $\NNC$ die Menge aller Normalteiler von $P$ mit elementarabelscher Faktorgruppe. Setzen wir $D:=\bigcap_{N\in \NNC} N$, so ist $D\nt P$ und nach \ref{10.1} gilt: $D$ ist der kleinste Normalteiler von $P$ mit elementarabelscher Faktorgruppe. Wir zeigen $D=\Phi(P)$.\\
Ist $U<P$ eine maximale Untergruppe, so gilt $U\nt P$ und $|P/U|=p$ nach \ref{5.13} und \ref{5.14}.
Daher ist $U\in \NNC$ und es folgt $D\leq \Phi(P)$.
Da $P/D$ elementarabelsch ist, ist $\Phi(P/D)=\langle 1 \rangle$. Sind $U_i/D$ f\"ur $i\in I$ die maximalen Untergruppen von $P/D$, so gilt $1=\bigcap_{i\in I}(U_i/D)=(\bigcap_{i\in I}U_i)/D$ und daher $\bigcap_{i\in I}U_i=D$. Nach dem Korrespondenzsatz ist $U_i/D$ in $P/D$ genau dann maximal, wenn $U_i$ in $P$ maximal ist. Weil $D$ in allen maximalen Untergruppen von $P$ enthalten ist, folgt $D=\bigcap_{i\in I}U_i=\Phi(P)$.
\item Aus $P=\langle x_1,\ldots,x_n\rangle$ folgt $\overline{P}:=P/\Phi(P)=\langle x_1\Phi(P),\ldots,x_n\Phi(P)\rangle$. Nach \ref{10.2.1}. ist $\overline{P}$ elementarabelsch und kann daher als Vektorraum \"uber $\ZZ_p$ aufgefasst werden. Da $|\overline{P}|=p^n$, ist dieser Vektorraum $n$-dimensional und wir erhalten $n\leq m$. Andererseits besitzt der Vektorraum eine Basis $y_1\Phi(P),\ldots,y_n\Phi(P)$ aus $n$ Elementen, also ist $P=\langle y_1,\ldots,y_n,\Phi(P)$. Aus Lemma \ref{9.2} folgt $P=\langle y_1,\ldots,y_n\rangle$.
\end{enumerate}
\end{beweis}

\begin{folgerung}\label{10.3}\index{zyklische Gr.}
 Sei $P$ eine $p$-Gruppe. Es gilt 
\begin{enumerate}
 \item \label{10.3.1}Ist $P/\Phi(P)$ zyklisch, so auch $P$.
 \item Ist $P/P'$ zyklisch, so auch $P$.
 \item Besitzt $P$ nur eine maximale Untergruppe, so ist $P$ zyklisch.
\end{enumerate}

\end{folgerung}

\begin{beweis}\spspace
 \begin{enumerate}
 \item Nach \ref{10.2}.\ref{10.2.1} ist $|P/\Phi(P)|=p$. Mit \ref{10.2}.\ref{10.2.2} folgt, dass $P$ von einem Element erzeugt wird.
 \item Weil $P$ nilpotent ist, gilt $P'\leq \Phi(P)$ nach \ref{9.7}. Nach dem 2. Isomorphiesatz ist $P/\Phi(P)\cong (P/P')/(\Phi(P)/P')$, also ist $P/\Phi(P)$ als Faktorgruppe einer zyklischen Gruppe zyklisch und mit \ref{10.3.1}. folgt die Behauptung.
 \item Ist $U< P$ die einzige maximale Untergruppe von $P$, so gilt $\Phi(P)=U$ und $|P/\Phi(P)|=|P/U|=p$. Also ist $P/\Phi(P)$ zyklisch und mit \ref{10.3.1}. folgt die Behauptung.
\end{enumerate}

\end{beweis}

Bei $p$-Gruppen spielen die folgenden Untergruppen eine wichtige Rolle:
\begin{definition}[$\Omega_i(P),\Agemo^i(P)$]
\index{Omega}
\index{Agemo}
Sei $P$ eine $p$-Gruppe und $i\in \NN_0$. Wir definieren
\begin{enumerate}
 \item $\Omega_i(P):=\langle x\in P\mid x^{p^i}=1\rangle$ das Erzeugnis der Elemente, deren Ordnung $p^i$ teilt
 \item $\Agemo^i(P):=\langle x^{p^i}\mid x\in P\rangle$ das Erzeugnis der $p^i$-ten Potenzen der Elemente von $P$. Diese zweite Gruppe nennen wir \emph{Agemo}.
\end{enumerate}

 
\end{definition}
\begin{bemerkung}
 Die Gruppen $\Omega_i(P)$ und $\Agemo^i(P)$ haben folgende offensichtliche Eigenschaften:
\begin{enumerate}
 \item $\langle 1\rangle=\Omega_0(P)\leq \Omega_1(P)\leq \Omega_2(P)\leq \ldots$\\
       $P=\Agemo^0(P)\geq \Agemo^1(P)\geq \Agemo^2(P)\geq\ldots$
\item $\Omega_i(P)\Char P, \Agemo^i(P)\Char P$
\item Ist $P$ abelsch, so gilt $\Omega_i(P)=\lbrace x\in P\mid x^{p^i}=1\rbrace$ und $\Agemo^i(P)=\lbrace x^{p^i}\mid x\in P\rbrace$.
\end{enumerate}

\end{bemerkung}

\begin{satz}
 Sei $P$ eine $p$-Gruppe. Dann gilt
\begin{enumerate}
 \item \label{10.6.1}$\Phi(P)=P'\Agemo^1(P)$.
 \item Ist $p=2$, so gilt $\Phi(P)=\Agemo^1(P)$.
 \item F\"ur $U\leq P$ ist $\Phi(U)\leq \Phi(P)$. F\"ur $N\nt P$ ist $\Phi(P)N/N=\Phi(P/N)$.
\end{enumerate}

\end{satz}

\begin{beweis}\spspace
 \begin{enumerate}
 \item Weil $p$-Gruppen nilpotent sind, gilt $P'\leq \Phi(P)$ nach \ref{9.7}. Da $P/\Phi(P)$ nach \ref{10.2} elementarabelsch ist, gilt $x^p\in \Phi(P)$ f\"ur alle $x\in P$ und daher $\Agemo^1(P)\leq \Phi(P)$. Also ist $P'\Agemo^1(P)\leq \Phi(P)$. Wegen $P'\leq P'\Agemo^1(P)$ ist $P/\bigl(P'\Agemo^1(P)\bigr)$ abelsch. Da $x^p\in \Agemo^1(P)$ f\"ur alle $x\in P$, haben alle Elemente der Faktorgruppe $P/\bigl(P'\Agemo^1(P)\bigr)$ Ordnung $\leq p$, also ist $P/\bigl(P'\Agemo^1(P)\bigr)$ elementarabelsch. Mit dem Burnsideschen Basissatz \ref{10.2}.\ref{10.2.2} folgt $\Phi(P)\leq P'\Agemo^1(P)$.
 \item Wegen $\lbrack x,y\rbrack=x^{-1}y^{-1}xy=x^{-1}\underbrace{y^{-2}xx^{-2}xy}_{y^{-1}}xy=(y^x)^{-2}x^{-2}(xy)^{-2}$ ist $P'\leq \Agemo^1(P)$, wenn $p=2$. Mit \ref{10.6.1}. folgt $\Phi(P)=\Agemo^1(P)$.
\item folgt aus \ref{10.6.1}., denn f\"ur $U<P$ ist $U'\leq P'$ und $\Agemo^1(U)\leq\Agemo^1(P)$. F\"ur $N\nt P$ ist $\bigl(P'\Agemo^1(P)N\bigr)/N=(P/N)'\Agemo^1(P/N)$.
\end{enumerate}

\end{beweis}
 In einem Spezialfall lassen sich die Gruppen $\Omega_1(P)$ bzw. $\Omega_2(P)$ (f\"ur $p=2$) genauer bestimmen. Diese Aussage ben\"otigen wir in \ref{10.8}, um die Quaternionengruppe zu charakterisieren.
\begin{satz}\label{10.7}
 Sei $P$ eine $p$-Gruppe.
\begin{enumerate}
 \item Ist $p>2$ und $P/\Zen(P)$ abelsch, so ist $\Omega_1(P)=\lbrace x\in P\mid x^p=1\rbrace$.
 \item \label{10.7.1}Ist $P/\Zen(P)$ elementarabelsch, so gilt f\"ur $x,y\in P$
$$(xy)^p=x^py^p,\qquad \text{falls}\quad p\neq 2,$$
$$(xy)^4=x^4y^4,\qquad \text{falls}\quad p=2.$$
F\"ur $p=2$ ist dann also $\Omega_2(P)=\lbrace x\in P\mid x^4=1\rbrace$.
\end{enumerate}

\end{satz}
\begin{beweis}\spspace
 \begin{enumerate}
  \item Da $P/\Zen(P)$ abelsch ist, gilt $P'\leq \Zen(P)$ nach \ref{2.15}.\ref{2.15.4}. Also kommutiert $\lbrack x,y\rbrack$ mit $x$ und $y$ und mit $z:=\lbrack y,x\rbrack\in P$ gilt $(xy)^i=x^iy^iz^{\frac{1}{2}i(i-1)}$ f\"ur alle $i\in \NN_0$. F\"ur $x,y\in \Omega_1(P)$ ist $x^p=y^p=1$ und nach \ref{2.13}.\ref{2.13.8} daher $z^p=\lbrack y,x\rbrack^p=\lbrack y,x^p\rbrack=\lbrack y,1\rbrack =1$. Weil $p$ ungerade ist, gilt $p\mid\frac{1}{2}p(p-1)$, also folgt $(xy)^p=x^py^pz^{\frac{1}{2}p(p-1)}=1$ f\"ur $x,y\in \Omega_1(P)$. Damit ist $\Omega_1(P)=\lbrace x\in P\mid x^p=1\rbrace$.
 \item Ist $P/\Zen(P)$ elementarabelsch, so ist $x^p\in \Zen(P)$ f\"ur alle $x\in P$. Setzen wir nun wieder $z:=\lbrack y,x\rbrack$, so folgt $z^p=1$ und damit $(xy)^p=x^py^p$ f\"ur $p>2$ wie in \ref{10.7.1}, da wieder $z\in P'\leq \Zen(P)$.
F\"ur $p=2$ ist (wieder nach \ref{2.13}) $z^6=\lbrack y,x\rbrack^6=\lbrack y,x^6\rbrack = \lbrack y,\underbrace{(x^2)^3}_{\in \Zen(P)}\rbrack=1$, da $x^2\in \Zen(P)$. Mit \ref{2.13}.\ref{2.13.9} folgt $(xy)^4=x^4y^4z^{\frac{1}{2}\cdot 4(4-1)}=x^4y^4$ und damit $\Omega_2(P) = \lbrace x\in P\mid x^4 =1\rbrace$.
 \end{enumerate}

\end{beweis}
In den folgenden S\"atzen werden wir die $p$-Gruppen der Ordnung $p^n$ bestimmen, die nur eine Untergruppe der Ordnung $p^s, 1\leq s\leq n-1$, besitzen. Eine solche Gruppe ist uns schon aus der \"Ubung bekannt.
\begin{satz}\label{10.8}
\index{Quaternionengruppe}
 Die Quaternionengruppe der Ordnung 8 ist die einzige Gruppe, die nur eine Untergruppe der Ordnung $p$ besitzt, aber mindestens zwei zyklische Untergruppen vom Index $p$.
\end{satz}
\begin{beweis}
 Sei $P$ eine $p$-Gruppe wie im Satz und seien $\langle x_1\rangle$ und $\langle x_2\rangle$ zwei verschiedene Untergruppen vom Index $p$. Da $\langle x_1\rangle$ und $\langle x_2\rangle$ maximal sind und $x_2\not\in \langle x_1\rangle$, ist $P=\langle x_1,x_2\rangle$. Nach dem Burnsideschen Basissatz \ref{10.2} folgt $P/\Phi(P)\cong\ZZ_p\times\ZZ_p$ (da $P$ nicht zyklisch). Weiter ist $\Phi(P)\leq \langle x_1\rangle\cap\langle x_2\rangle\leq\Zen(P)$. Da $P/\Zen(P)$ nicht zyklisch sein kann (sonst w\"are $P$ zyklisch nach \ref{2.2}), folgt $\Zen(P)=\Phi(P)$. Insbesondere ist $P/\Zen(P)$ elementarabelsch. Wir k\"onnen daher \ref{10.7} anwenden. Danach ist die Abbildung $$\varphi:P\to P, \quad x\mapsto x^p\quad (\text{bzw.}\quad x\mapsto x^4 \quad\text{f\"ur}\quad p=2)$$ ein Endomorphismus von $P$ mit $\Img\varphi\leq\Zen(P)$. Insbesondere ist $\Img \varphi$ eine abelsche Untergruppe von $P$. Sei nun $M$ eine beliebige abelsche Untergruppe von $P$. Dann ist $M$ zyklisch.\\
ACHTUNG: HIER FEHLT EINE ZEILE, MEINE VORLAGE IST UNLESERLICH [ToDo]\\
Nach dem Hauptsatz \"uber endliche abelsche Gruppen ist $M$ daher zyklisch, denn direkte Produkte mehrerer zyklischer Gruppen $> \langle 1\rangle$ besitzen mehrere Untergruppen der Ordnung $p$.
Damit ist $\Img\varphi$ zyklisch, nach dem Homomorphiesatz also auch $P/\Ker\varphi\cong \Img \varphi$. Also ist $\Ker\varphi\nleq\Phi(P)$, denn sonst w\"are $\ZZ_p\times \ZZ_P\cong P/\Phi(P)\stackrel{\text{2.Isom.}}{\cong}\bigl(P/\Ker\varphi\bigr)/\bigl(\Phi(P)/\Ker\varphi\bigr)$ als Faktorgruppe einer zyklischen Gruppe zyklisch, ein Widerspruch.
Also gibt es ein $x\in \Ker\varphi\backslash\Phi(P)$.
W\"are $p\neq 2$, so w\"are $\ord x=p$ und $\langle x\rangle$ daher die einzige minimale Untergruppe von $P$. Dann w\"are allerdings $\Phi(P)=\langle 1\rangle$, denn sonst m\"usste ja auch $\Phi(P)$ diese minimale Untergruppe enthalten. Wegen $P/\Phi(P)\cong \ZZ_p\times\ZZ_p$ w\"are dann $P\cong\ZZ_p\times \ZZ_p$, ein Widerspruch.
Damit muss $p=2$ und $\ord x=4$ sein. Wir betrachten $M:=\langle x, \Phi(P)\rangle$. Wegen $\Phi(P)=\Zen(P)$ ist $M$ abelsch und daher zyklisch (siehe oben). Da $x\not\in\Phi(P)$ und $M$ eine zyklische $p$-Gruppe ist, folgt $M=\langle x\rangle$ und also $\mathopen{|}M\mathclose{|}=4$. Da $x\not\in \Phi(P)$ und $\Phi(P)\neq\langle 1\rangle$ (sonst w\"are $P\cong \ZZ_2\times \ZZ_2$), erhalten wir $|\Phi(P)|=2$. Wegen $|P/\Phi(P)|=4$ ist dann $|P|=8$. Da $P$ weder abelsch noch isomorph zu $\Di_4$ sein kann, muss $P\cong \Quat$ isomorph zur Quaternionengruppe der Ordnung $8$ sein. $\Quat$ erf\"ullt tats\"achlich die Voraussetzungen.
\end{beweis}


L\"asst man die Bedingung \"uber die Untergruppen vom Index $p$ weg, so ergeben sich noch weitere Gruppen, die wir zun\"achst definieren m\"ussen.
\begin{definition}[verallgemeinerte Quaternionengruppe]
 \index{Quaternionengruppe!verallgemeinerte}
Sei $n\geq 3, A:=\langle a\mid a^{2^{n-1}}=1\rangle\cong \ZZ_{2^{n-1}}$ und $B:=\langle b\mid b^4=1\rangle\cong \ZZ_4$. Durch 
$$\varphi:B\to Aut A,\quad b\longmapsto(a^j\mapsto a^{-j})$$ wird ein Homomorphismus mit $\Ker\varphi=\langle b^2\rangle$ definiert. Im semidirekten Produkt $A\stackrel{\varphi}{\rtimes}B$ ist $C:=\langle a^{2^{n-2}}, b^2\langle$ daher eine elementarabelsche Untergruppe mit $C\leq \Zen(A\stackrel{\varphi}{\rtimes}B)$ und $|C|=4$. Wir setzen $N:=\langle a^{2^{n-2}}\cdot b^2\rangle < C$. Die Gruppe $\Quat_n:=A\stackrel{\varphi}{\rtimes}B/N$ der Ordnung $2^n$ hei\ss{}t \emph{verallgemeinerte Quaternionengruppe} ($\Quat_3$ ist die bekannte Quaternionengruppe der Ordnung $8$).
\end{definition}

\begin{satz}
 \index{Quaternionengruppe!verallgemeinerte}
F\"ur die verallgemeinerte Quaternionengruppe $\Quat_n$ der Ordnung $2^n$ gilt:
\begin{enumerate}
 \item $\Quat_n=\langle x,y\mid x^{2^{n-1}}=1, y^2=x^{2^{n-2}}, y^{-1}xy=x^{-1}\rangle$.
 \item $\Quat_n=\lbrace 1, x,\ldots, x^{2^{n-1}-1},y,yx,\ldots,yx^{2^{n-1}-1}\rbrace$.
 \item \label{10.10.3}$(yx^i)^2=x^{2^{n-2}}$ f\"ur $i\in \NN_0$.
 \item $\langle x\rangle$ ist f\"ur $n\geq 4$ charakteristische Untergruppe von $\Quat_n$ vom Index $2$.
 \item \label{10.10.5}$\Zen(\Quat_n)=\lbrace 1, x^{2^{n-2}}\rbrace$ ist die einzige Untergruppe der Ordnung $2$ von $\Quat_n$.
\end{enumerate}

\end{satz}

\begin{beweis}\spspace
 \begin{enumerate}
  \item \label{10.10.1}Wir setzen $x:=aN$ und $y:=bN$, dann ist $\Quat_n=\langle x,y\rangle$ und es gilt $\ord x=2^{n-1}, \ord y = 4, x^{2^{n-2}}=a^{2^{n-2}}N=b^2N=y^2$ und $y^{-1}xy=x^{-1}$. $\Quat_n$ wird also von Elementen mit entsprechenden Relationen erzeugt. Da $xy=yx^{-1}$, lassen sich alle Elemente von $\Quat_n$ in der Form $y^ix^j$ mit $i\in \lbrace 0,1\rbrace$ (wegen $y^2\in \langle x\rangle$) und $j\in \lbrace 0,1,\ldots, 2^{n-1}-1\rbrace$ schreiben. Also kann $\langle x,y\rangle$ auch nicht mehr als $2^n$ Elemente enthalten und es folgt $\Quat_n=\langle x,y\mid\ldots\rangle$.
 \item wurde in \ref{10.10.1} mitbewiesen
 \item $(yx^i)^2=yx^iyx^i=y^2y^{-1}x^iyx^i=y^2x^{-i}x^i=y^2=x^{2^{n-2}}$.
 \item Nach \ref{10.10.3} haben alle Elemente in $\Quat_n\backslash\langle x\rangle$ Ordnung $4$. Also ist $\langle x\rangle$ die einzige zyklische Untergruppe in $\Quat_n$ vom Index $2$, wenn $n\geq 4$. Damit ist $\langle x\rangle$ charakteristisch in $\Quat_n$.
 \item Nach \ref{10.10.3} haben die Elemente der Form $yx$ alle Ordnung $4$. Daher ist $\langle x^{2^{n-2}}$ die einzige Untergruppe von Ordnung $2$. Aus $yx^{i+j}=yx^ix^j=x^jyx^i=yy^{-1}x^jyx^i=yx^{-j}x^i=yx{i-j} \Longleftrightarrow x^j=x^{-j} \Longleftrightarrow x^{2j}=1 \Longleftrightarrow j=2^{n-2} \text{oder} j=2^{n-1}$ folgt auch $\Zen(\Quat_n)=\langle x^{2^{n-2}}$.
 \end{enumerate}

\end{beweis}

Die Eigenschaft \ref{10.10.5} charakterisiert $\Quat_n$ fast vollst\"andig, wie wir in \ref{10.12} zeigen werden. Daf\"ur ben\"otigen wir noch ein 

\begin{lemma}\label{10.11}
 Sei $P$ eine nichtabelsche $p$-Gruppe und $A\nt P$ ein maximaler abelscher Normalteiler von $P$. Dann gilt $\Cen_P(A)=A$ und $\Zen(P)\leq A$.
\end{lemma}

\begin{beweis}
Angenommen, es ist $A<\Cen_P(A)$. Wegen $A\nt P$ gilt $\Cen_P(A)\nt P$ nach \ref{2.4}. Nach dem Korrespondenzsatz gilt daher $\Cen_P(A)/A \nt P/A$. In der nilpotenten Gruppe $P/A$ hat jeder nichttriviale Normalteiler nach \ref{8.5} einen nichttrivialen Schnitt mit dem Zentrum $\Zen(P/A)$, also gibt es ein $x\in\Cen_P(A)\backslash A$ mit $\overline{x}:=xA\in \Zen(P/A)$. Wegen $x\in \Cen_P(A)$ ist $\langle A,x\rangle$ abelsch. Da $\overline{x}\in\Zen(P/A)$, gilt $\langle \overline{x}\nt P/A$. Nach dem Korrespondenzsatz ist die zu $\langle \overline{x}\rangle$ entsprechende Untergruppe $\langle A,x\rangle\leq P$ von $P$ dann normal in $P$, ein Widerspruch zur Maximalit\"at von $A$.
 
\end{beweis}

\begin{satz}\label{10.12}
 \index{Quaternionengruppe!verallgemeinerte}
 \index{zyklische Gr.}
 Eine $p$-Gruppe $P$ mit genau einer minimalen Untergruppe ist entweder zyklisch oder eine verallgemeinerte Quaternionengruppe.
\end{satz}
\begin{beweis}
Wir f\"uhren den Beweis durch Induktion nach $|P|$. Sei $P$ eine nicht zyklische (und damit nicht abelsche) $p$-Gruppe mit nur einer minimalen Untergruppe. Dann besitzt $P$ mindestens zwei maximale Untergruppen vom Index $p$, denn sonst w\"are $P/\Phi(P)$ zyklisch und damit  auch $P$ selbst nach \ref{10.3}.\\
Nach Induktionsvoraussetzung sind die beiden maximalen Untergruppen entweder zyklisch oder verallgemeinerte Quaternionengruppen. Sind beide zyklisch, so ist $P$ nach \ref{10.8} eine Quaternionengruppe der Ordnung $8$. Sonst enth\"alt $P$ eine verallgemeinerte Quaternionengruppe, das hei\ss{}t, es ist $p=2$.\\
Sei $X$ ein maximaler abelscher Normalteiler von $P$ und sei $|X|=2^{n-1}$. Da $X$ wie $P$ nur eine minimale Untergruppe besitzt, ist $X$ nach dem Hauptsatz \"uber endliche abelsche Gruppen zyklisch. Sei also $X=\langle x\rangle$. Der weitere Beweis verl\"auft in zwei Schritten.
\begin{enumerate}
 \item \label{10.12.1}Ist $y\in P\backslash X$ mit $y^2\in X$, so ist $\langle x,y\rangle$ eine verallgemeinerte Quaternionengruppe\\
HIER FEHLT EINE ZEILE [ToDo]\\
Wegen $y^2\in X, y \not\in x$ liegt genau die H\"alfte der Elemente von $\langle y\rangle$ in $X$, das hei\ss{}t, f\"ur $V:=\langle y\rangle \cap X$ gilt $|V|=\frac{|\langle y\rangle|}{2}$. W\"are $V=X$, so w\"are $\Cen_P(X)\leq \langle y \rangle >X$, ein Widerspruch zu \ref{10.11}. Also ist $V<X$ und in der zyklischen Gruppe $X$ gibt es eine Untergruppe $X_1\leq X$ mit $|X_1|=2|V|$. Nat\"urlich ist dann $V<X_1$. F\"ur die Gruppe $X_1\cdot\langle y\rangle$ gilt daher nach \ref{1.1} $|X_1\cdot\langle y\rangle|=\frac{|X_1|\cdot|\langle y\rangle|}{|V|}=2|\langle y\rangle|$. $V$ enth\"alt auch die H\"alfte der Elemente von $\langle y \rangle$, also gilt auch $|X_1\langle y\rangle|=2|X_1|$. Damit enth\"alt die Gruppe $X_1\langle y\rangle$ zwei maximale zyklische Untergruppen (n\"amlich $X_1$ und $\langle y \rangle$) und genau eine minimale Untergruppe (da $P$ nur eine besitzt). Nach \ref{10.8} ist $X_1\langle y\rangle$ daher eine Quaternionengruppe der Ordnung $8$. Folglich ist $\ord y =4$ und somit $y^2=x^{2^{n-2}}$, denn $x^{2^{n-2}}$ ist das einzige Element der Ordnung $2$ von $P$. Wegen $X\nt P$ ist $(yx)^2=yxyx=y^2\underbrace{(y^{-1}xy)}_{\in X}x\in y^2X=X$ und analog zum Beweis f\"ur $y$ folgt $(yx)^2=x^{2^{n-2}}$. Also ist $y^{-1}xy=y^{-2}yxyxx^{-1}=y^{-2}(yx)^2x^{-1}=x^{-2^{n-2}}x^{2^{n-2}}x^{-1}$. Damit ist $\langle x,y\rangle$ eine Gruppe mit Ordnung mindestens $2^n$, deren Erzeuger die Relationen $x^{2^{n-1}}=1, y^2=x^{2^{n-2}}, y^{-1}xy=x^{-1}$ erf\"ullen. Damit ist $\langle x, y\rangle$ eine verallgemeinerte Quaternionengruppe der Ordnung $2^n$.
\item $P=\langle x, y\rangle$. Da $X$ ein maximaler abelscher Normalteiler von $P$ ist, gilt $\Cen_P(X)=X$ nach \ref{10.11}. Weiter ist $\Norm_P(X)=P$, also ist $\Norm_P(X)/\Cen_P(X)=P/X=:A$. Nach \ref{2.4} ist $A$ isomorph zu einer Untergruppe von $\Aut X\cong \ZZ_{2^{n-1}}^*\stackrel{\ref{4.12}}{=}\langle \overline{-1} \rangle\times\langle \overline{5}\rangle$, das hei\ss{}t, es gibt einen Monomorphismus $\varphi:A\to \Aut X$. Wie im Beweis von \ref{2.4} gesehen wird $\varphi$ gegeben durch $\varphi(a)=(x\longmapsto x^a)$. In \ref{10.12.1} haben wir gezeigt: ist $\ord a=2$, so ist $\varphi(a)=(x\longmapsto x^{-1})$. In der $2$-Gruppe $A=P/X>1$ gibt es nat\"urlich Elemente der Ordnung $2$, das hei\ss{}t, es gibt Elemente $y\in P\backslash X$ mit $y^2\in X$. Weil alle diese Elemente auf $\overline{-1}$ abgebildet werden und $\varphi$ injektiv ist, ist $\langle \overline{-1}\rangle$ die einzige Untergruppe von $\varphi(A)$ mit Ordnung $2$. Also ist $\varphi(A)$ zyklisch und enth\"alt $\langle \overline{-1}\rangle$. Die einzige zyklische Untergruppe von $\langle\overline{-1}\rangle\times\langle \overline{5} \rangle$, die $\langle\overline{-1}\rangle$ enh\"alt, ist $\langle \overline{-1}\rangle$ selbst. Also folgt $\varphi(A)=\langle\overline{-1}\rangle$, und damit ist $|A|=2$ und also $\lbrack P:X\rbrack=2$. Daraus ergibt sich $P=\langle x,y\rangle$.
\end{enumerate}

 
\end{beweis}


\begin{folgerung}\label{10.13}
 Ist in der $p$-Gruppe $P$ jede abelsche Untergruppe zyklisch, so ist $P$ selbst zyklisch oder eine verallgemeinerte Quaternionengruppe.
\end{folgerung}
\begin{beweis}
Sei $U\leq P$ mit $|U|=p$. Wegen $\Zen(P)\nt G$ ist $U\Zen(P)\leq P$. Offensichtlich ist $U\Zen(P)$ abelsch und daher nach Voraussetzung zyklisch. $U$ ist eine minimale Untergruppe von $U\Zen(P)$, und - da $U\Zen(P)$ zyklisch ist - die einzige. Damit folgt $U\leq \Zen(P)$, denn sonst bes\"a\ss{}e $\Zen(P)$ eine weitere minimale Untergruppe. Also liegen alle Untergruppen von $P$ mit Ordnung $p$ in $\Zen(P)$. Nach Voraussetzung ist $\Zen(P)$ zyklisch und besitzt daher nur eine Untergruppe der Ordnung $p$. Also gibt es in $P$ nur eine einzige minimale Untergruppe. Mit Satz \ref{10.12} folgt die Behauptung.
 
\end{beweis}

\begin{satz}\label{10.14}
 Sei $P$ eine $p$-Gruppe mit $|P|=p^n$. F\"ur $0\leq s\leq n$ sei $r_s$ die Anzahl der Untergruppen von $P$ mit Ordnung $p^s$. Ist $r_s=1$ f\"ur ein $s$ mit $1<s<n$, so ist $P$ zyklisch.
\end{satz}
\begin{beweis}
Sei $U$ die einzige Untergruppe von Ordnung $p^s$ von $P$. Nach Satz \ref{5.14} gibt es eine Untergruppe $V$ der Ordnung $p^{s+1}$ mit $U<V$. $V$ besitzt dann nur eine maximale Untergruppe und ist daher nach Satz \ref{10.3} zyklisch. Damit ist auch $U$ als Untergruppe von $V$ zyklisch.\\
Durch mehrfache Anwendung von Satz \ref{5.14} folgt, dass jede Untergruppe der Ordnung $p$ und jede der Ordnung $p^2$ in einer Untergruppe der Ordnung $p^s$, also hier in $U$, enthalten ist. $U$ besitzt als zyklische Untergruppe nur je eine Untergruppe von Ordnung $p$ und $p^2$. Damit gibt es auch in $P$ nur je eine Untergruppe mit Ordnung $p$ und $p^2$. Mit Satz \ref{10.12} folgt: ist $p>2$, so ist $P$ zyklisch, f\"ur $p=2$ ist $P$ zyklisch oder eine verallgemeinerte Quaternionengruppe. Diese letztere $\Quat_n=\langle x,y\mid x^{2^{n-1}}=1, y^2=x^{2^{n-2}}, y^{-1}xy=x^{-1}\rangle$ enth\"alt jedoch zwei zyklische Untergruppen der Ordnung $4$, n\"amlich $\langle x^{2^{n-2}}\rangle$ und $\langle y\rangle$. Also ist $P$ zyklisch.
 
\end{beweis}


\begin{satz}\label{10.15}
 Sei $|P|=p^n$. F\"ur ein $s$ mit $1<s\leq n$ sei jede Untergruppe der Ordnung $p^s$ von $P$ zyklisch. Ist $p^s\neq 4$, so ist $P$ zyklisch. Ist $p^s=4$, so ist $P$ zyklisch oder eine verallgemeinerte Quaternionengruppe.
\end{satz}

\begin{beweis}
Wegen $s\geq 2$ sind alle Untergruppen von $P$ von Ordnung $p^2$ zyklisch, da sie nach \ref{5.14} in einer Untergruppe der Ordnung $p^s$ enthalten sind. Sei $N\leq \Zen(P)$ mit $|N|=p$. Gibt es eine weitere Untergruppe $U<P$ mit $|U|=p$, so ist wegen $N\nt P$ zun\"achst $NU\leq P$ und aus $N\leq \Zen(P)$ folgt $NU\cong N\times U\cong \ZZ_p\times \ZZ_p$, ein Widerspruch. Somit ist $N$ die einzige minimale Untergruppe von $P$ und mit \ref{10.12} folgt: $P$ ist zyklisch oder eine verallgemeinerte Quaternionengruppe. F\"ur $p>2$ sind wir also fertig.\\
F\"ur $s\geq 3$ hat die verallgemeinerte Quaternionengruppe $\Quat_n=\langle x,y\mid x^{2^{n-1}}=1, y^2=x^{2^{n-2}}, y^{-1}xy=x^{-1}\rangle$ der Ordnung $2^n$ aber auch nichtzyklische Untergruppen der Ordnung $2^s$, n\"amlich $U=\langle x^{2^{n-s}}, y\rangle$. Die Erzeuger von $U$ erf\"ullen die Relationen einer verallgemeinerten Quaternionengruppe, das hei\ss{}t, es ist $U\cong \Quat_s$. Also kann nur im Fall $p^s=4$ die Gruppe $P$ auch nichtzyklisch sein.
 
\end{beweis}

Als n\"achstes wollen wir die $p$-Gruppen der Ordnung $p^{n+1}$ klassifizieren, die eine zyklische Untergruppe der Ordnung $p^n$ besitzen. Daf\"ur ben\"otigen wir folgendes

\begin{lemma}
 Sei $P$ eine nichtabelsche $p$-Gruppe mit einer zyklischen maximalen Untergruppe $H$. Gilt $1\neq x^p\in H$ f\"ur alle $x\in P\backslash H$, so ist $p=2$ und $P$ eine verallgemeinerte Quaternionengruppe.
\end{lemma}

\begin{beweis}
Sei $H=\langle h\rangle$ mit $\ord h=p^n$ und sei $z:=h^{p^{n-1}}$. Nach Voraussetzung ist $\ord x>p$ f\"ur alle $x\in P\backslash H$. Ist $U<P$ mit $|U|=p$, so gilt folglich $U\leq H$. Die zyklische Gruppe $H$ enth\"alt aber nur eine Untergruppe der Ordnung $p$, n\"amlich $\langle z\rangle$. Daher ist $U=\langle z\rangle$, das hei\ss{}t, $\langle z\rangle$ ist die einzige minimale Untergruppe von $P$. Mit \ref{10.12} folgt die Behauptung.
 \end{beweis}

\begin{satz}\label{10.17}
\index{$p$-Gruppe!modulare}
\index{Semidiedergr.}
\index{Quasidiedergruppe|see{Semidiedergr.}}
 Sei $P$ eine nichtabelsche $p$-Gruppe der Ordnung $p^{n+1}$, die eine zyklische Untergruppe $H=\langle h\rangle$ der Ordnung $p^n$ enth\"alt. Dann gilt:
\begin{enumerate}
 \item \label{10.17.1}Ist $p\neq 2$, so ist $P$ isomorph zur sogenannten \emph{modularen $p$-Gruppe} $\Mod_{n+1}(p)=\langle h, a\mid h^{p^n}=a^p=1, a^{-1}ha=h^{1+p^{n-1}}\rangle$, also zu einem semidirekten Produkt $\ZZ_{p^n}\rtimes\ZZ_p$.
 \item Ist $p=2$ und $n\geq 3$, so ist $P$ isomorph zu einer der folgenden Gruppen:
  \begin{enumerate}
   \item \label{10.17.2.a}einer verallgemeinertern Quaternionengruppe $$\Quat_{n+1}=\langle h, a\mid h^{2^n}=1, a^2=h^{2^{n-1}}, a^{-1}ha=h^{-1}\rangle$$.
   \item \label{10.17.2.b}einer Diedergruppe $$\Di_{2^n}=\langle h,a\mid h^{2^n}=a^2=1, a^{-1}ha=h^{-1}\rangle$$.
   \item \label{10.17.2.c}einer sogenannten \emph{Semidiedergruppe} (manchmal auch \emph{Quasidiedergruppe}) $$\SD_{n+1}=\langle h,a\mid h^{2^n}=a^2=1, a^{-1}ha=h^{-1+2^{n-1}}\rangle$$
   \item\label{10.17.2.d} einer sogenannten modularen $2$-Gruppe $$\Mod_{n+1}(2)=\langle h, a\mid h^{2^n}=a^2=1, a^{-1}ha=h^{1+2^{n-1}}\rangle$$.
   \end{enumerate}
   F\"ur $p=2$ und $n=2$ treten nur die F\"alle \ref{10.17.2.a} und \ref{10.17.2.b} auf. Die Gruppen in \ref{10.17.2.b}, \ref{10.17.2.c} und \ref{10.17.2.d} sind semidirekte Produkte der Form $\ZZ_{2^n}\rtimes \ZZ_2$.
\end{enumerate}

\end{satz}

\begin{beweis}
 Besitzt die Untergruppe $H$ kein Komplement in $P$, so gilt $x^p\neq 1$ f\"ur alle $x\in P\backslash H$ und $x^p\in H$ wegen $\lbrack P:H\rbrack=p$. Nach Lemma \ref{10.16} ist dann $p=2$ und $P$ eine verallgemeinerte Quaternionengruppe. In allen \"ubrigen F\"allen ist $P$ ein semidirektes Produkt der Form $H\rtimes \langle a\rangle$, wobei $a\in P\backslash H$ Ordnung $p$ hat.\\
HIER FEHLT EINE ZEILE [ToDo]
\\
Nach Folgerung \ref{4.17} gibt es daher bis auf Isomorphie genau ein nichttriviales semidirektes Produkt $H\rtimes \langle a\rangle$, n\"amlich das in \ref{10.17.1} angegebene.\\
F\"ur $p=2$ und $n\geq 3$ ist $\Aut H\cong\ZZ_{2^n}^*\stackrel{\ref{4.12}}{\cong}\ZZ_2\times\ZZ_2$, also gibt es in $\Aut H$ genau drei verschiedene Elemente der Ordnung $2$. Offensichtlich haben wir $\overline{-1},\overline{-1+2^{n-1}}$ und $\overline{1+2^{n-1}}$ mit Ordnung $2$ in $\ZZ_{2^n}^*$. Also sind die in \ref{10.17.2.b}, \ref{10.17.2.c} und \ref{10.17.2.d} aufgez\"ahlten Gruppen alle M\"oglichkeiten (f\"ur $n=2$ ist $\ZZ_4^*=\ZZ_2$, also gibt es in diesem Fall nur eine Gruppe, die $\Di_4$).\\
Wir m\"ussen noch zeigen, dass diese drei Gruppen paarweise nicht isomorph sind. Sei $z:=h^{2^{n-1}}$. Das Element $z$ ist das einzige Element mit Ordnung $2$ in $H\nt P$ und daher in $\Zen(P)$ enthalten. F\"ur $i\in \NN$ ist $a^{-1}h^ia=h^{-i}z^i$ im Fall \ref{10.17.2.c}, $a^{-1}h^ia=h^{i}z^i$ im Fall \ref{10.17.2.d}. Daher ist $\Zen(P)=\langle z\rangle$ im Fall \ref{10.17.2.c}, $\Zen(P)=\langle h^2\rangle$ im Fall \ref{10.17.2.d}. Da auch $\Zen(\Di_{2^n})=\langle z\rangle$, ist $\Mod_{n+1}(2)$ zu keiner der beiden anderen Gruppen isomorph. In der Diedergruppe haben alle Elemente aus $P\backslash H$ Ordnung $2$, also gibt es in $\Di_{2^n}$ genau $2^n+1$ Involutionen (Elemente der Ordnung $2$). In $\SD_{n+1}$ ist $(ha)^2=haha=hh^{-1+2^{n-1}}=h^{2^{n-1}}=z\neq 1$, das hei\ss{}t, $ha\in \SD_{n+1}\backslash H$ hat Ordnung $4$. Folglich gibt es in $\SD_{n+1}$ h\"ochstens $2^n-1+1=2^n$ Elemente der Ordnung $2$, das hei\ss{}t, $\Di_{2^n}$ und $\SD_{n+1}$ sind nicht isomorph.
\end{beweis}


Im Beweis von \ref{10.17} haben wir bereits einige Eigenschaften der Gruppen gezeigt, die wir f\"ur \ref{10.19} ben\"otigen:

\begin{satz}\label{10.18}
 Sei $P$ eine Dieder-, Semidieder- oder eine verallgemeinerte Quaternionengruppe der Ordnung $2^n$. Dann gilt (mit den Bezeichnungen aus \ref{10.17}):
\begin{enumerate}
 \item $\Zen(P)=\langle h^{2^{n-2}}$, also insbesondere $|\Zen(P)|=2$.
 \item $P'=\Phi(P)=\langle h^2\rangle$, also insbesondere $|P/P'|=4$.
 \item $\SD_n$ besitzt genau drei maximale Untergruppen. Diese sind isomorph zu $\ZZ_{2^{n-1}}, \Di_{2^{n-2}},\Quat_{2n}$.
 \item Alle abelschen Untergruppen von $P$ mit Ordnung $8$ sind zyklisch.
 \item $\Zen\bigl(\Mod_n(p)\bigr)=\langle h^p\rangle$, also insbesondere $|\Zen\bigl(\Mod_n(p)\bigr)|=p^{n-2}$.
 
\end{enumerate}
 [ToDo] HIER FEHLT WAHRSCHEINLICH EIN TEIL DES SATZES (evtl. innerhalb der Aufzaehlung)
\end{satz}

[ToDo] HIER FEHLT DER REST DES KAPITELS 10

%$$|-x|=|+x|$$ $$\mathopen{|}-x\mathclose{|}=\mathopen{|}+x\mathclose{|}$$
%$$|\Phi(P)|=|P/\Phi(P)|$$ $$\mathopen{|}\Phi(P)\mathclose{|}=\mathopen{|}P/\Phi(P)\mathclose{|}$$

%$\vert$, $\mid$


\section{section 11}\label{11}
bla		

\printindex
\end{document}