\section{Einf\"uhrung}
Folgendes wird als bekannt vorausgesetzt:
\begin{itemize}
 \item Definitionen: Gruppe, Untergruppe, Normalteiler, Faktorgruppe, Ordnung
 \index{Gruppe}
 \index{Untergruppe}
 \index{Normalteiler}
 \index{Faktorgruppe}
 \index{Ordnung}

 \item Untergruppenkriterien
 \item Homomorphiesatz und Isomorphies\"atze:
 \index{Homorphiesatz}
  \begin{satz}[Homomorphiesatz]
   \label{homomorphiesatz}
   $G, H$ Gruppen, $\varphi: G \to H$ Homomorphismus. Dann gilt: 
   \begin{equation*} 
    G/ H \cong \Img \varphi
   \end{equation*}
  \end{satz}
  \begin{satz}[1. Isomorphiesatz]
  \index{Isomorphies\"atze!1. Isomorphiesatz}
  \label{isomorphiesatz1}
   Sei $H\leq G, N \nt G$. Dann gilt:
   \begin{enumerate}
    \item $HN \leq G$
    \item $N \nt HN$
    \item $H \cap N \nt H$
    \item Der kanonische Homomorphismus
     \begin{equation*}
      H/H\cap N \to HN/N, \quad h\l(H\cap N\r) \mapsto hN
     \end{equation*}
     ist ein Isomorphismus.
   \end{enumerate}
  \end{satz}
  \begin{satz}[2. Isomorphiesatz]
  \index{Isomorphies\"atze!2. Isomorphiesatz}
  \label{isomorphiesatz2}
   Seien $M, N \nt G, M\leq N$. Dann gilt:
   \begin{enumerate}
    \item $N/M \nt G/M$
    \item Der kanonische Homomorphismus
     \begin{equation*}
      \l(G/M\r)/\l(N/M\r) \to G/N, \quad \l(gM\r)\l(N/M\r) \mapsto gN
     \end{equation*}
     ist ein Isomorphismus.
   \end{enumerate}
  \end{satz}
 \item Der Satz von Lagrange:
 \index{Lagrange!Satz von}
  \begin{satz}[Satz von Lagrange]
   Die Ordnung der Untergruppe teilt stets die Ordnung der Gruppe.
  \end{satz}
 \item Definition von zyklischen Gruppen, die Gruppen $\ZZ_n$.
 \index{Gruppe!zyklische}
 \index{Gruppe!$\ZZ_n$}
 \item Hauptsatz \"uber endliche abelsche Gruppen:
 %\index{Hauptsatz \"uber endliche abelsche Gruppen}
  \begin{satz}[Hauptsatz \"uber endliche abelsche Gruppen]
  \label{hauptsatz_ueber_endliche_abelsche_gruppen}
   Jede endliche abelsche Gruppe ist direktes Produkt von Gruppen von Primzahlpotenzordnung, das hei\ss{}t es gibt (nicht notwendig verschiedene) Primzahlen $p_1, \ldots, p_k$, so dass
   \begin{equation*}
    G \cong \ZZ_{p_1^{a_1}} \times \ldots \ZZ_{p_k^{a_k}}
   \end{equation*}
  \end{satz}
 \item Eigenschaften zyklischer Gruppen, zum Beispiel:
  \begin{equation*}
   \ZZ_{nm} \cong \ZZ_n \times \ZZ_m \Leftrightarrow \ggT\l(n, m\r)=1
  \end{equation*}
 \item Definition des direkten Produkts von Gruppen
 \item Definition der symmetrischen Gruppe $\Symm_n$, alternierende Gruppe $\Alt_n$
  \begin{equation*}
   \sign: \Symm_n \to \l\{1, -1 \r\}
  \end{equation*}
 \item Rechnen mit Zykeln in $\Symm_n$, Zerlegung in elementfremde Zykeln.
\end{itemize}
